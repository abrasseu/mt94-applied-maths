Forme quadratique

Rn to R
gradient // dev lim

Rn to Rm
différentiabilité ssi exists J telle que

\begin{theoreme}[Fonction k-lipschitzienne]
$f : E \to \R$ est $k$-lipschitzienne
$$
	\exists k \tq \forall (x,y)\in E^{2}, \abs{f(x)-f(y)} \leq k\abs{x-y}
$$
\end{theoreme}



	
\section{Linéaire et non-linéaire}

		\begin{definition}[Linéarité]
			Une application $f:E \to F$ est linéaire si $\forall x,y\in E\quad \forall \lambda ,\mu \in K$ :
			\begin{equation}
				f(\lambda x + \mu y) = \lambda f(x) + \mu f(y)
			\end{equation}
			On note alors $f\in L(E,F)$ 
		\end{definition}

		% TODO : Cas de la linéarité
		Dans le cas de problèmes linéaires, on a : $Ax=b$
		Avec Scilab : $A\backslash b = x$ résout le système avec la méthode de Gauss.

		

		% TODO : Morphisme, injectivité surj bij, noyau image

			% $$
			% \begin{cases}
			% 	a_{11}x_1 + a_{12}x_2 + \ldots  a_{1n}x_n =  b_1	\\
			% 	a_{21}x_1 + a_{22}x_2 + \ldots  a_{2n}x_n =  b_2	\\

			% 	a_{n1}x_1 + a_{n2}x_2 + \ldots  a_{nn}x_n =  b_n
			% \end{cases}
			% $$

		\begin{definition}[Différentiabilité]
			Soit $f: \R^n \to \R^n$ et $x_o \in \R^n$.
			\\$f$ est différentiable en $x_0$ si $\exists J_f \in \M_{n,n} \tq \forall h \in \R^n$ :
			$$
				f(x_0 +h) = f(x_0) + J_f h + \norm{h} \epsilon(h)
			$$
			où $lim_{h \to 0} \epsilon(h) = 0$ et la norme utilisée est la norme euclidienne.
			La matrice Jacobienne de $f$ en $x_0$ est $J_f(x_0)$ telle que : $(J)_{ij} = \dfdp{f_i}{x_j}(x_0)$.
		\end{definition}


		\bigskip
		\begin{theoreme}
			\begin{enumerate}
				\item $f : [a;b] \to [a;b]$ alors \quad $\exists! x^* \tq f(x^*)=x^*$
				\item Si $f$ est k-lipschitzienne alors \quad $\exists K \in ]0;1[ \tq \forall x,y :\quad \abs{f(x)-f(y)} \leq K\abs{x-y}$
			\end{enumerate}
		\end{theoreme}
