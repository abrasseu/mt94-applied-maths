
% ============================================================
% 			Colors
% ============================================================
	\usepackage{xcolor}
	\definecolor{code-bg}{rgb}{0.99,0.99,1}
	\definecolor{bordeaux}{rgb}{0.34,0.06,0.07}
	\definecolor{bleu}{rgb}{0.14,0.35,0.77}
	\definecolor{vert}{rgb}{0.2,0.65,0.12}
	\definecolor{bleu-vert}{rgb}{0.14,0.35,0.27}


% ============================================================
% 			Liens/Références
% ============================================================
	\hypersetup{
		% backref=true,			% Pour biblio
		% pagebackref=true,
		% hyperindex=true,
		colorlinks=false,
		breaklinks=true,
		urlcolor= blue,
		linkcolor= blue,
		% bookmarks=true,			% Signets
		bookmarksopen=true,
		pdfauthor={Alexandre Brasseur},	% Infos PDF
		pdfsubject={MT94}
		pdftitle={MT94 - Cahier d'intégration},
		pdfkeywords={Mathématiques, Maths, Ingénieur, Engineer, Engineering, Méthodes, Numérique}
	}


% ============================================================
% 			Header/Footer
% ============================================================	
	
	\renewcommand{\chaptermark}[1]{\markboth{\upshape\thechapter. ~ \textsc{#1}}{}}
	\lhead{MT94}
	% \chead{}
	\rhead{\leftmark}
	% \lfoot{}
	% \cfoot{}
	% \rfoot{\thepage}
	\renewcommand{\headrulewidth}{0.01pt}


% ============================================================
% 			Refonte
% ============================================================

	% - Avant-propos
	\addto\captionsfrench{\renewcommand{\abstractname}{Avant-propos\vspace{.3cm}\hrule\vspace{.5cm}}}
	\renewcommand{\abstracttextfont}{\normalfont\large}
	% - Mise en page
	\renewcommand{\abstractnamefont}{\huge\bfseries}
	\newcommand{\sdl}{\smallskip}
	\renewcommand{\bigskip}{\vspace{.8cm}}

% ============================================================
% 			Sections
% ============================================================
	\setcounter{secnumdepth}{4}
	\renewcommand{\thechapter}{\Roman{chapter}}
	\renewcommand{\thesection}{\Roman{section})}
	\renewcommand{\thesubsection}{\arabic{subsection})}
	\renewcommand{\thesubsubsection}{\alph{subsubsection})}
	\renewcommand{\theparagraph}{\roman{paragraph})}

	% \titleformat{hcommandi}[hshapei]{hformati}{hlabeli}{hsepi}{hbefore-codei}[hafter-codei]
	\titleformat{\chapter}
		[display]		% style : hang, display, runin, leftmargin, ... 
		{\bfseries}		% changement de fontenuméro + titre 
		{\huge\textsc{\chaptertitlename}~\thechapter}		% numéro 
		{20pt}		% espace entre le numéro et le titre 
		{\Huge}		% changement de fonte du titre
		[\hrule]	% after code		
	\titlespacing*{\chapter}		% 
		{0pt}		% retrait à gauche 
		{40pt}		% espace avant 
		{50pt}		% espace après 
		[0pt]		% retrait à droite

	\titleformat{\section}[hang]
		{\normalfont\Large\bfseries}		% fonte numéro + titre 
		{\thesection}		% numéro 
		{1em}				% espace entre le numéro et le titre 
		{}					% fonte titre 
		% [\hrule]
	\titlespacing*{\section} 
		{0pt}		% retrait à gauche 
		{3.5ex plus 1ex minus .2ex}		% espace avant 
		{2.3ex plus .2ex}		% espace après 
		[0pt]		% retrait à droite

	\titleformat{\subsection}[hang]
		{\normalfont\large\bfseries}		% fonte numéro + titre 
		{\thesubsection}		% numéro 
		{1em}		% espace entre le numéro et le titre 
		{}		% fonte titre 
	\titlespacing*{\subsection} 
		{0pt}		% retrait à gauche 
		{3.25ex plus 1ex minus .2ex}		% espace avant 
		{1.5ex plus .2ex}		% espace après 
		[0pt]		% retrait à droite

	\titleformat{\subsubsection} 
		[hang]		% style : hang, display, runin, leftmargin, ... 
		{\normalfont\normalsize\bfseries}		% fonte numéro + titre 
		{\thesubsubsection}		% numéro 
		{1em}		% espace entre le numéro et le titre 
		{}		% fonte titre 
	\titlespacing*{\subsubsection} 
		{0pt}		% retrait à gauche 
		{3.25ex plus 1ex minus .2ex}		% espace avant 
		{1.5ex plus .2ex}		% espace après 
		[0pt]		% retrait à droite

	\titleformat{\paragraph} 
		[hang]		% style : hang, display, runin, leftmargin, ... 
		{\normalfont\normalsize\bfseries}		% fonte numéro + titre 
		{\theparagraph}		% numéro 
		{1em}		% espace entre le numéro et le titre 
		{}		% fonte titre 
		[]		% après le titre, p.ex. "\@addpunct{.}" de amsmath 
	\titlespacing*{\paragraph} 
		{0pt}		% retrait à gauche 
		{3.25ex plus 1ex minus .2ex}		% espace avant 
		{5pt}		% espace après

	\titleformat{\subparagraph} 
		[runin]		% style : hang, display, runin, leftmargin, ... 
		{\normalfont\normalsize\bfseries}		% fonte numéro + titre 
		{\thesubparagraph}		% numéro 
		{1em}		% espace entre le numéro et le titre 
		{}		% fonte titre 
		[]		% après le titre, p.ex. "\@addpunct{.}" de amsmath 
	\titlespacing*{\subparagraph} 
		{\parindent}		% retrait à gauche 
		{3.25ex plus 1ex minus .2ex}		% espace avant 
		{1em}		% espace après

% \titleformat{\section}[frame] 
% {\normalfont} 
% {\filright\footnotesize\enspace SECTION \thesection\enspace} 
% {8pt} 
% {\Large\bfseries\filcenter}

% ============================================================
% 			Maths
% ============================================================
	\setcounter{MaxMatrixCols}{12}

	% - Commandes
	\newcommand{\R}{\mathbb{R}}
	\newcommand{\N}{\mathbb{N}}
	\newcommand{\Z}{\mathbb{Z}}
	\newcommand{\C}{\mathbb{C}}
	\newcommand{\proba}{\mathbb{P}\,}
	\newcommand{\Cont}{\mathcal{C}}
	\newcommand{\M}{\mathcal{M}}

	\newcommand{\abs}[1]{\lvert #1 \rvert}
	\newcommand{\norm}[1]{\lVert #1 \rVert}
	\newcommand{\scal}[2]{\langle #1, #2 \rangle}
	\newcommand{\dfdp}[2]{\frac{\partial #1}{\partial #2}}
	\newcommand{\dfdpp}[3]{\frac{\partial^{#3} #1}{\partial #2^{#3}}}
	%\newcommand{\interv}[4]{\mathopen{#1}#2\mathpunct{};#3\mathclose{#4}}

	\newcommand{\limto}[2]{\xrightarrow[#1 \to #2]{}}
	\newcommand{\dint}[1]{~\text{d}#1}
	\newcommand{\ie}{\textit{i.e.} }
	\newcommand{\tq}{\text{ tel que }}

	\newcommand{\grad}{~\nabla}
	\newcommand{\Aire}{\text{Aire}~}
	\newcommand{\vect}[1]{\text{Vect}\,\{#1\}}
	\newcommand{\Ker}{\text{Ker~}}
	\newcommand{\Image}{\text{Im~}}
	\newcommand{\rang}{\text{rang~}}
	
	\newcommandx{\suite}[3][1=n,3=\in\N]{(#2_{#1})_{#1#3}}

	% - Styles de théorèmes
	\newtheoremstyle{th}
		{10pt}{10pt}	% espace après 
		{}				% police du corps du théorème 
		{\parindent}	% indentation (vide pour rien, \parindent) 
		{}{}			% police + ponctuation après le théorème 
		{\newline}		% après le titre du théorème (espace ou \newline) 
		{\textsc{\bfseries\thmname{#1} \,\thmnumber{#2} :}\thmnote{ #3}}
	\newtheoremstyle{thShort}
		{10pt}{10pt}	% espace après 
		{}				% police du corps du théorème 
		{\parindent}	% indentation (vide pour rien, \parindent) 
		{}{}			% police + ponctuation après le théorème 
		{ }		% après le titre du théorème (espace ou \newline) 
		{{\bfseries\thmname{#1} \thmnumber{#2} :}\thmnote{ #3} }

	\newtheoremstyle{note}
		{10pt}{10pt}	% espace après 
		{}				% police du corps du théorème 
		{\parindent}	% indentation (vide pour rien, \parindent) 
		{}{}			% police + ponctuation après le théorème 
		{\newline}		% après le titre du théorème (espace ou \newline) 
		{{\itshape\thmname{#1} \thmnumber{#2} :}\thmnote{ #3}}
	\newtheoremstyle{noteShort}
		{10pt}{10pt}	% espace après 
		{}				% police du corps du théorème 
		{\parindent}	% indentation (vide pour rien, \parindent) 
		{}{}			% police + ponctuation après le théorème 
		{ }		% après le titre du théorème (espace ou \newline) 
		{{\itshape\thmname{#1} \thmnumber{#2} :}\thmnote{ #3} }

	% - Théorèmes
	\theoremstyle{th}
		\newtheorem{definition}{Définition}[chapter]
		\newtheorem{prop}{Propriété}[chapter]
		\newtheorem{theoreme}{Théorème}[chapter]
		\newtheorem{lemme}[theoreme]{Lemme}
		\newtheorem*{ex}{Exemple}
		
	\theoremstyle{thShort}
		\newtheorem{propShort}{Propriété}[chapter]
		\newtheorem{definitionShort}{Définition}[chapter]
		\newtheorem*{exShort}{Exemple}
		

	\theoremstyle{note}
		\newtheorem*{note}{Remarque}
		\newtheorem*{preuve}{Preuve}

	\theoremstyle{noteShort}
		\newtheorem*{noteShort}{Remarque}
		\newtheorem*{preuveShort}{Preuve}
	


% ============================================================
% 			Minted
% ============================================================
	% - Numérotation des lignes
	\renewcommand{\theFancyVerbLine}{\roboto
	\textcolor[rgb]{0.15,0.15,0.15}{\scriptsize
	{\arabic{FancyVerbLine}}}}

	% - Scilab custom code :		alias = \scicode{"filepath.sce"}
	\newmintedfile[scicode]{scilab}{
		bgcolor=code-bg,
		linenos=true,
		fontfamily=txtt,			% txtt, pcr
		fontsize=\footnotesize,
		numberblanklines=true,
		numbersep=5pt,
		gobble=0,
		frame=leftline,
		framerule=0.2pt,
		framesep=4mm,
		tabsize=4,
		obeytabs=false,
		numberfirstline=true,
		stepnumber=5,
		numbersep=3mm,
		breaklines=true,
		samepage=false,
		texcl=false,
		%label=Code Scilab,
	}
	% \usemintedstyle{colorful}		% colorful, borland, friendly
	\surroundwithmdframed{minted}

	\renewcommand\listingscaption{Code Source}
	\renewcommand\listoflistingscaption{Liste des codes sources Scilab}

	% - Algorithme
	\SetAlgoSkip{bigskip}			% Espace vertical avant/après l'algorithme
	%\SetCustomAlgoRuledWidth{10pt}
	%\SetAlCapHSkip{}
	\SetAlgoCaptionLayout{centerline}
	\SetAlgoInsideSkip{smallskip}
	\setlength{\algomargin}{3em}
	\setlength{\interspacetitleruled}{3pt}

	% - Code inline
	% \newcommand{\code}[1]{\colorbox{gray!10}{\texttt{#1}}}
	\newcommand{\code}[1]{\mintinline{scilab}{#1}}

	% - Accès fichier scilab
	\newcommand{\tdA}{"../Scilab/TD1-Pb_Non_Lineaires/"}
	\newcommand{\tdB}{"../Scilab/TD2-Fractales/"}
	\newcommand{\tdC}{"../Scilab/TD3-Equations_Differentielles/"}
	\newcommand{\tdD}{"../Scilab/TD4-Valeurs_Propres/"}
	\newcommand{\tdE}{"../Scilab/TD5-Optimisation/"}
	\newcommand{\tdF}{"../Scilab/TD6-Equations_Differentielles_Partielles/"}
	\newcommand{\img}{"Exports/"}
