\documentclass[a4paper,10pt,titlepage,openany]{report}

% - Fonts et encodage
\usepackage[utf8]{inputenc}
\usepackage[T1]{fontenc}
\usepackage[francais]{babel}
\usepackage{lmodern}        	% Pour l'affichage des polices

% - Mise en page
\usepackage{layout}
\usepackage[top=2cm, bottom=2cm, left=2.5cm, right=2.5cm]{geometry}
\usepackage{float}
\usepackage{hyperref}			% Liens sommaire
\usepackage{titlesec}
\usepackage{fancyhdr}
\pagestyle{fancy}
\usepackage{mdframed}
\usepackage{abstract}

% - Maths
\usepackage{amsmath}
\usepackage{amsfonts}
\usepackage{amssymb}
\usepackage{amsthm}				% Pour les théorèmes
\usepackage{mathtools}			% Pour les acollades ++
\usepackage{xargs}
%\usepackage[hyperref,amsmath,thmarks,thref]{ntheorem}

% - Graphics et Tableaux
\usepackage{array}
\usepackage{multirow}
\usepackage{graphicx}
\usepackage{wrapfig}
\usepackage{caption}
\usepackage{subcaption}
\captionsetup{belowskip=6pt,aboveskip=-6pt}
\captionsetup[table]{belowskip=6pt,aboveskip=6pt}
\captionsetup[figure]{belowskip=6pt,aboveskip=6pt}


%\usepackage{txfonts}

% - Minted
\usepackage[chapter]{minted}	% Pour le code (numéroté // chapter)
\usepackage[french,frenchkw,algoruled,algochapter,longend]{algorithm2e}

% - Infos
\title{Cahier d'intégration MT94}
\author{Alexandre Brasseur}
\date{Printemps 2017}


% ============================================================
% 			Fonts
% ============================================================
	\usepackage[rm]{roboto}
	%\usepackage[math]{kurier}
	\renewcommand{\rmdefault}{bch}		% bch, ppl,                
	\renewcommand{\familydefault}{\rmdefault}
	\usepackage{fourier}


% ============================================================
% 			Colors
% ============================================================
	\usepackage{xcolor}
	\definecolor{code-bg}{rgb}{0.99,0.99,1}
	\definecolor{bordeaux}{rgb}{0.34,0.06,0.07}
	\definecolor{bleu}{rgb}{0.14,0.35,0.77}
	\definecolor{vert}{rgb}{0.2,0.65,0.12}
	\definecolor{bleu-vert}{rgb}{0.14,0.35,0.27}


% ============================================================
% 			Liens/Références
% ============================================================
	\hypersetup{
		% backref=true,			% Pour biblio
		% pagebackref=true,
		% hyperindex=true,
		colorlinks=false,
		breaklinks=true,
		urlcolor= blue,
		linkcolor= blue,
		% bookmarks=true,			% Signets
		bookmarksopen=true,
		pdfauthor={Alexandre Brasseur},	% Infos PDF
		pdfsubject={MT94}
		pdftitle={MT94 - Cahier d'intégration},
		pdfkeywords={Mathématiques, Maths, Ingénieur, Engineer, Engineering, Méthodes, Numérique}
	}


% ============================================================
% 			Header/Footer
% ============================================================	
	
	\renewcommand{\chaptermark}[1]{\markboth{\upshape\thechapter. ~ \textsc{#1}}{}}
	\lhead{MT94}
	% \chead{}
	\rhead{\leftmark}
	% \lfoot{}
	% \cfoot{}
	% \rfoot{\thepage}
	\renewcommand{\headrulewidth}{0.01pt}


% ============================================================
% 			Refonte
% ============================================================

	% - Avant-propos
	\addto\captionsfrench{\renewcommand{\abstractname}{Avant-propos\vspace{.3cm}\hrule\vspace{.5cm}}}
	\renewcommand{\abstracttextfont}{\normalfont\large}
	% - Mise en page
	\renewcommand{\abstractnamefont}{\huge\bfseries}
	\newcommand{\sdl}{\smallskip}
	\renewcommand{\bigskip}{\vspace{.8cm}}

% ============================================================
% 			Sections
% ============================================================
	\setcounter{secnumdepth}{4}
	\renewcommand{\thechapter}{\Roman{chapter}}
	\renewcommand{\thesection}{\Roman{section})}
	\renewcommand{\thesubsection}{\arabic{subsection})}
	\renewcommand{\thesubsubsection}{\alph{subsubsection})}
	\renewcommand{\theparagraph}{\roman{paragraph})}

	% \titleformat{hcommandi}[hshapei]{hformati}{hlabeli}{hsepi}{hbefore-codei}[hafter-codei]
	\titleformat{\chapter}
		[display]		% style : hang, display, runin, leftmargin, ... 
		{\bfseries}		% changement de fontenuméro + titre 
		{\huge\textsc{\chaptertitlename}~\thechapter}		% numéro 
		{20pt}		% espace entre le numéro et le titre 
		{\Huge}		% changement de fonte du titre
		[\hrule]	% after code		
	\titlespacing*{\chapter}		% 
		{0pt}		% retrait à gauche 
		{40pt}		% espace avant 
		{50pt}		% espace après 
		[0pt]		% retrait à droite

	\titleformat{\section}[hang]
		{\normalfont\Large\bfseries}		% fonte numéro + titre 
		{\thesection}		% numéro 
		{1em}				% espace entre le numéro et le titre 
		{}					% fonte titre 
		% [\hrule]
	\titlespacing*{\section} 
		{0pt}		% retrait à gauche 
		{3.5ex plus 1ex minus .2ex}		% espace avant 
		{2.3ex plus .2ex}		% espace après 
		[0pt]		% retrait à droite

	\titleformat{\subsection}[hang]
		{\normalfont\large\bfseries}		% fonte numéro + titre 
		{\thesubsection}		% numéro 
		{1em}		% espace entre le numéro et le titre 
		{}		% fonte titre 
	\titlespacing*{\subsection} 
		{0pt}		% retrait à gauche 
		{3.25ex plus 1ex minus .2ex}		% espace avant 
		{1.5ex plus .2ex}		% espace après 
		[0pt]		% retrait à droite

	\titleformat{\subsubsection} 
		[hang]		% style : hang, display, runin, leftmargin, ... 
		{\normalfont\normalsize\bfseries}		% fonte numéro + titre 
		{\thesubsubsection}		% numéro 
		{1em}		% espace entre le numéro et le titre 
		{}		% fonte titre 
	\titlespacing*{\subsubsection} 
		{0pt}		% retrait à gauche 
		{3.25ex plus 1ex minus .2ex}		% espace avant 
		{1.5ex plus .2ex}		% espace après 
		[0pt]		% retrait à droite

	\titleformat{\paragraph} 
		[hang]		% style : hang, display, runin, leftmargin, ... 
		{\normalfont\normalsize\bfseries}		% fonte numéro + titre 
		{\theparagraph}		% numéro 
		{1em}		% espace entre le numéro et le titre 
		{}		% fonte titre 
		[]		% après le titre, p.ex. "\@addpunct{.}" de amsmath 
	\titlespacing*{\paragraph} 
		{0pt}		% retrait à gauche 
		{3.25ex plus 1ex minus .2ex}		% espace avant 
		{5pt}		% espace après

	\titleformat{\subparagraph} 
		[runin]		% style : hang, display, runin, leftmargin, ... 
		{\normalfont\normalsize\bfseries}		% fonte numéro + titre 
		{\thesubparagraph}		% numéro 
		{1em}		% espace entre le numéro et le titre 
		{}		% fonte titre 
		[]		% après le titre, p.ex. "\@addpunct{.}" de amsmath 
	\titlespacing*{\subparagraph} 
		{\parindent}		% retrait à gauche 
		{3.25ex plus 1ex minus .2ex}		% espace avant 
		{1em}		% espace après

% \titleformat{\section}[frame] 
% {\normalfont} 
% {\filright\footnotesize\enspace SECTION \thesection\enspace} 
% {8pt} 
% {\Large\bfseries\filcenter}

% ============================================================
% 			Maths
% ============================================================
	\setcounter{MaxMatrixCols}{12}

	% - Commandes
	\newcommand{\R}{\mathbb{R}}
	\newcommand{\N}{\mathbb{N}}
	\newcommand{\Z}{\mathbb{Z}}
	\newcommand{\C}{\mathbb{C}}
	\newcommand{\proba}{\mathbb{P}\,}
	\newcommand{\Cont}{\mathcal{C}}
	\newcommand{\M}{\mathcal{M}}

	\newcommand{\abs}[1]{\lvert #1 \rvert}
	\newcommand{\norm}[1]{\lVert #1 \rVert}
	\newcommand{\scal}[2]{\langle #1, #2 \rangle}
	\newcommand{\dfdp}[2]{\frac{\partial #1}{\partial #2}}
	\newcommand{\dfdpp}[3]{\frac{\partial^{#3} #1}{\partial #2^{#3}}}
	%\newcommand{\interv}[4]{\mathopen{#1}#2\mathpunct{};#3\mathclose{#4}}

	\newcommand{\limto}[2]{\xrightarrow[#1 \to #2]{}}
	\newcommand{\dint}[1]{~\text{d}#1}
	\newcommand{\ie}{\textit{i.e.} }
	\newcommand{\tq}{\text{ tel que }}

	\newcommand{\grad}{~\nabla}
	\newcommand{\Aire}{\text{Aire}~}
	\newcommand{\vect}[1]{\text{Vect}\,\{#1\}}
	\newcommand{\Ker}{\text{Ker~}}
	\newcommand{\Image}{\text{Im~}}
	\newcommand{\rang}{\text{rang~}}
	
	\newcommandx{\suite}[3][1=n,3=\in\N]{(#2_{#1})_{#1#3}}

	% - Styles de théorèmes
	\newtheoremstyle{th}
		{10pt}{10pt}	% espace après 
		{}				% police du corps du théorème 
		{\parindent}	% indentation (vide pour rien, \parindent) 
		{}{}			% police + ponctuation après le théorème 
		{\newline}		% après le titre du théorème (espace ou \newline) 
		{\textsc{\bfseries\thmname{#1} \,\thmnumber{#2} :}\thmnote{ #3}}
	\newtheoremstyle{thShort}
		{10pt}{10pt}	% espace après 
		{}				% police du corps du théorème 
		{\parindent}	% indentation (vide pour rien, \parindent) 
		{}{}			% police + ponctuation après le théorème 
		{ }		% après le titre du théorème (espace ou \newline) 
		{{\bfseries\thmname{#1} \thmnumber{#2} :}\thmnote{ #3} }

	\newtheoremstyle{note}
		{10pt}{10pt}	% espace après 
		{}				% police du corps du théorème 
		{\parindent}	% indentation (vide pour rien, \parindent) 
		{}{}			% police + ponctuation après le théorème 
		{\newline}		% après le titre du théorème (espace ou \newline) 
		{{\itshape\thmname{#1} \thmnumber{#2} :}\thmnote{ #3}}
	\newtheoremstyle{noteShort}
		{10pt}{10pt}	% espace après 
		{}				% police du corps du théorème 
		{\parindent}	% indentation (vide pour rien, \parindent) 
		{}{}			% police + ponctuation après le théorème 
		{ }		% après le titre du théorème (espace ou \newline) 
		{{\itshape\thmname{#1} \thmnumber{#2} :}\thmnote{ #3} }

	% - Théorèmes
	\theoremstyle{th}
		\newtheorem{definition}{Définition}[chapter]
		\newtheorem{prop}{Propriété}[chapter]
		\newtheorem{theoreme}{Théorème}[chapter]
		\newtheorem{lemme}[theoreme]{Lemme}
		\newtheorem*{ex}{Exemple}
		
	\theoremstyle{thShort}
		\newtheorem{propShort}{Propriété}[chapter]
		\newtheorem{definitionShort}{Définition}[chapter]
		\newtheorem*{exShort}{Exemple}
		

	\theoremstyle{note}
		\newtheorem*{note}{Remarque}
		\newtheorem*{preuve}{Preuve}

	\theoremstyle{noteShort}
		\newtheorem*{noteShort}{Remarque}
		\newtheorem*{preuveShort}{Preuve}
	


% ============================================================
% 			Minted
% ============================================================
	% - Numérotation des lignes
	\renewcommand{\theFancyVerbLine}{\roboto
	\textcolor[rgb]{0.15,0.15,0.15}{\scriptsize
	{\arabic{FancyVerbLine}}}}

	% - Scilab custom code :		alias = \scicode{"filepath.sce"}
	\newmintedfile[scicode]{scilab}{
		bgcolor=code-bg,
		linenos=true,
		fontfamily=txtt,			% txtt, pcr
		fontsize=\footnotesize,
		numberblanklines=true,
		numbersep=5pt,
		gobble=0,
		frame=leftline,
		framerule=0.2pt,
		framesep=4mm,
		tabsize=4,
		obeytabs=false,
		numberfirstline=true,
		stepnumber=5,
		numbersep=3mm,
		breaklines=true,
		samepage=false,
		texcl=false,
		%label=Code Scilab,
	}
	% \usemintedstyle{colorful}		% colorful, borland, friendly
	\surroundwithmdframed{minted}

	\renewcommand\listingscaption{Code Source}
	\renewcommand\listoflistingscaption{Liste des codes sources Scilab}

	% - Algorithme
	\SetAlgoSkip{bigskip}			% Espace vertical avant/après l'algorithme
	%\SetCustomAlgoRuledWidth{10pt}
	%\SetAlCapHSkip{}
	\SetAlgoCaptionLayout{centerline}
	\SetAlgoInsideSkip{smallskip}
	\setlength{\algomargin}{3em}
	\setlength{\interspacetitleruled}{3pt}

	% - Code inline
	% \newcommand{\code}[1]{\colorbox{gray!10}{\texttt{#1}}}
	\newcommand{\code}[1]{\mintinline{scilab}{#1}}

	% - Accès fichier scilab
	\newcommand{\tdA}{"../Scilab/TD1-Pb_Non_Lineaires/"}
	\newcommand{\tdB}{"../Scilab/TD2-Fractales/"}
	\newcommand{\tdC}{"../Scilab/TD3-Equations_Differentielles/"}
	\newcommand{\tdD}{"../Scilab/TD4-Valeurs_Propres/"}
	\newcommand{\tdE}{"../Scilab/TD5-Optimisation/"}
	\newcommand{\tdF}{"../Scilab/TD6-Equations_Differentielles_Partielles/"}
	\newcommand{\img}{"Exports/"}


% ============================================================

% ============================================================
% 			Debut du document
% ============================================================
\begin{document}

	% == PAGE DE GARDE == %
	\begin{titlepage}
	% \fontfamily{phv}\selectfont
	\fontfamily{bch}\selectfont 
	{
		\begin{wrapfigure}{r}{0cm}
			\includegraphics[width=.2\linewidth]{\tdA\img logo_UTC.eps}
		\end{wrapfigure}
		\hspace*{\stretch{1}}

		% \begin{flushleft} 
		% 	\Large
		% 	Printemps 2017
		% \end{flushleft}
	}

	\vspace*{\stretch{1}} 

	\begin{center} 
		\textbf{\Huge MT94 - Cahier d'intégration}
		\hrule
		\vspace{.5cm}

		\LARGE
		Introduction aux Mathématiques Appliquées
		% \vspace{.2cm}

		Printemps 2017

		\vspace{1cm}

		{Alexandre Brasseur}
	\end{center}

	% \vspace*{\stretch{2}} 
	% \centering
	% \includegraphics[width=\linewidth]{\tdA\img julia_set.jpg}	
	\vspace*{\stretch{2}} 

	\begin{center} 
		\large
		% Printemps 2017
	\end{center}
\end{titlepage}
	\thispagestyle{empty}
	
	% == TABLE DES MATIERES == %
	\tableofcontents
	\label{sommaire}
	\thispagestyle{empty}		% évite redondance numérotation


	% == RESUME/MOTIVATIONS == %
	\begin{abstract}
		Ce cahier d'intégration a pour vocation d'exposer tout ce que j'ai pu apprendre dans l'UV MT94.
%		Bien sûr, il est perfectible car on peut toujours faire mieux. Cependant je suis assez fier du résultat et de ce que j'ai appris en mathématiques mais aussi en Scilab et en Latex.
		Cette introduction aux mathématiques appliquées est le juste mélange entre les théories mathématiques vues en cours, leurs applications en Travaux Pratiques avec le logiciel de calcul numérique Scilab et le compte-rendu de ces connaissances dans ce rapport en \LaTeX. On apprend donc beaucoup au cours du semestre et ce qui fait que MT94 est une très bonne UV.
		
		\medskip
		Premièrement, elle permet d'appliquer toutes les connaissances mathématiques apprises depuis MT90 jusqu'à SY01. On comprend enfin comment donner un sens à toutes ses formules et comment elles servent à résoudre des problèmes. On a un aperçu du côté pratique des mathématiques. C'est un peu comme se servir d'un tournevis la première fois après l'avoir étudié pendant des années, le tournevis est cependant très complexe. On s'approche ainsi du travail de l'ingénieur et cela est plaisant.

		\smallskip
		Deuxièmement, on n'aperçoit certes qu'une partie des mathématiques appliquées, mais on a envie de continuer. Explorer entièrement tous les outils, les méthodes et leurs applications vus dans cette matière prendrait je pense plus d'un semestre. Cependant une fois lancée, cette envie de trouver de nouvelles applications, d'optimiser des algorithmes et de découvrir une nouvelle facette des mathématiques, ne s'arrête pas. Ce rapport n'est pas complet et ne le sera probablement jamais, tout comme les sciences mathématiques. Je n'ai pas la prétention d'y rapporter toutes les connaissances sur le sujet mais de continuer à y intégrer les miennes.

		\smallskip
		Troisièmement, cette UV est bien enseignée. Stéphane Mottelet et Djalil Kateb sont passionnés de mathématiques et cela se voit. Je les remercie pour avoir renforcé en moi cet intérêt pour cette science.	

		Enfin, l'ensemble des codes Scilab que j'ai créé sont disponibles à cette adresse : \\ \url{https://abrasseu@gitlab.utc.fr/abrasseu/MT94.git}. Bonne lecture !

%		\smallskip
%		Ainsi ce rapport est divisé entre 3 parties
	\end{abstract}
	
	% == CHAPITRES== %
	%\part{Mathématiques Appliquées}

	%\chapter{Chapitre de test}

\section{Introduction}
	Lorem ipsum dolor sit amet, consectetur adipisicing elit, sed do eiusmod
	tempor incididunt ut labore et dolore magna aliqua. Ut enim ad minim veniam,
	quis nostrud exercitation ullamco laboris nisi ut aliquip ex ea commodo
	consequat. Duis aute irure dolor in reprehenderit in voluptate velit esse
	cillum dolore eu fugiat nulla pariatur. Excepteur sint occaecat cupidatat non
	proident, sunt in culpa qui officia deserunt mollit anim id est laborum.


\section{Section}
	
		Salut 

	\subsection{Théorèmes}
		
		Lorem ipsum dolor sit amet, consectetur adipisicing elit, sed do eiusmod
		tempor incididunt ut labore et dolore magna aliqua. Ut enim ad minim veniam,
		quis nostrud exercitation ullamco laboris nisi ut aliquip ex ea commodo.

		\begin{theoreme}
			theoreme
			Lorem ipsum dolor sit amet, consectetur adipisicing elit, sed do eiusmod
			$$
				f(t) = ke^{5t} - 4
			$$
			proident, sunt in culpa qui officia deserunt mollit anim id est laborum.
		\end{theoreme}

		\begin{preuve}
			preuve
			Lorem ipsum dolor sit amet, consectetur adipisicing elit, sed do eiusmod
			tempor incididunt ut labore et dolore magna aliqua. Ut enim ad minim ven		
		\end{preuve}

		\begin{ex}
			Lorem ipsum dolor sit amet, consectetur adipisicing elit, sed do eiusmod
			tempor incididunt ut labore et dolore magna aliqua. Ut enim ad minim veniam,
			quis nostrud exercitation ullamco laboris nisi ut aliquip ex ea commodo
		\end{ex}

		\begin{definition}
			definition
			Lorem ipsum dolor sit amet, consectetur adipisicing elit, sed do eiusmod
			$$
				f(t) = ke^{5t} - 4
			$$
			proident, sunt in culpa qui officia deserunt mollit anim id est laborum.
		\end{definition}

		\begin{prop}
			prop
			Lorem ipsum dolor sit amet, consectetur adipisicing elit, sed do eiusmod
			$$
				f(t) = ke^{5t} - 4
			$$
			proident, sunt in culpa qui officia deserunt mollit anim id est laborum.
		\end{prop}

		\begin{note}
			note
			Lorem ipsum dolor sit amet, consectetur adipisicing elit, sed do eiusmod
			$$
				f(t) = ke^{5t} - 4
			$$
			proident, sunt in culpa qui officia deserunt mollit anim id est laborum.
		\end{note}

		\begin{noteShort}
			noteShort
			Lorem ipsum dolor sit amet, consectetur adipisicing elit, sed do eiusmod
			$$
				f(t) = ke^{5t} - 4
			$$
			proident, sunt in culpa qui officia deserunt mollit anim id est laborum.
		\end{noteShort}

	\subsection{Construction statistique}

		\subsubsection{Paragraphes}

			Lorem ipsum dolor sit amet, consectetur adipisicing elit, sed do eiusmod
			tempor incididunt ut labore et dolore magna aliqua. Ut enim ad minim veniam,
			quis nostrud exercitation ullamco laboris nisi ut aliquip ex ea commodo
			consequat. Duis aute irure dolor in reprehenderit in voluptate velit esse
			cillum dolore eu fugiat nulla pariatur. Excepteur sint occaecat cupidatat non
			proident, sunt in culpa qui officia deserunt mollit anim id est laborum.

			SPACE Lorem ipsum dolor sit amet, consectetur adipisicing elit, sed do eiusmod
			tempor incididunt ut labore et dolore magna aliqua. Ut enim ad minim veniam,
			quis nostrud exercitation ullamco laboris nisi ut aliquip ex ea commodo.

			DOUBLE SPACE tempor incididunt ut labore et dolore magna aliqua. Ut enim ad minim veniam,
			tempor incididunt ut labore et dolore magna aliqua. Ut enim ad minim veniam,
			\\
			DOUBLE SLASH tempor incididunt ut labore et dolore magna aliqua. Ut enim ad minim veniam,



		\subsubsection{Code}

			tempor incididunt ut labore et dolore magna aliqua. Ut enim ad minim veniam,
			tempor incididunt ut labore et dolore magna aliqua. Ut enim ad minim veniam,

			\begin{listing}[H]
				\scicode{\tdA 1.0-Toutes_Methodes.sce}
				\caption{Methodes}
			\end{listing}

			tempor incididunt ut labore et dolore magna aliqua. Ut enim ad minim veniam,

			\begin{figure}[H]
				\centering
				\includegraphics[width=\linewidth]{\tdA\img 2-Trajectoire_Cercle.eps}
				\caption{Trajectoire Cercle}
			\end{figure}

			tempor incididunt ut labore et dolore magna aliqua. Ut enim ad minim veniam,
			tempor incididunt ut labore et dolore magna aliqua. Ut enim ad minim veniam,

			\begin{table}[H]
				\centering
				\begin{tabular}{|l|c|r|}
					\hline
					1	& 2	& 3	\\	\hline
					1	& 2	& 3	\\	\hline\hline
					$\frac{1}{2}$	& waaaaaaaaa	& ooooooooooooooooooooooooooooooh	\\	\hline
				\end{tabular}
				\caption{Tableadzdadz}
			\end{table}

\chapter{Ohhh}
Lorem ipsum dolor sit amet, consectetur adipisicing elit, sed do eiusmod
\section{Waaaaaa}
	tempor incididunt ut labore et dolore magna aliqua. Ut enim ad minim veniam,

	\subsection{Wooooooo}
		quis nostrud exercitation ullamco laboris nisi ut aliquip ex ea commodo
		\subsubsection{Wiiiiiiii}
			consequat. Duis aute irure dolor in reprehenderit in voluptate velit esse
			\paragraph{Weeeeee}
				cillum dolore eu fugiat nulla pariatur. Excepteur sint occaecat cupidatat non
				\subparagraph{Wouuuuu}
					proident, sunt in culpa qui officia deserunt mollit anim id est laborum.

	
	\chapter{Problèmes non-linéaires}
\label{ch-1}

% \section{Introduction}

	Dans ce chapitre, l'objectif est de résoudre des équations, d'annuler des fonctions, d'abord dans $\R$, puis dans $\R^n$, c'est-à-dire trouver $x^*$ tel que $f(x^*)=0$.
	De nombreux problèmes peuvent être modélisés par des équations (cas de $f$ à valeurs dans $\R$) voire par des systèmes dynamiques (cas de $f$ à valeurs dans $\R^n$). En réalité, ces modèles sont bien souvent non-linéaires et les résoudre de manière analytique est donc impossible. On a donc recours à des méthodes numériques qui, par l'utilisation d'algorithmes spécifiques, permettent d'obtenir une solution approchée correcte.
	
	Dans cette partie nous traitons les problèmes de résolution. Nous verrons les problèmes d'optimisation dans la seconde partie des problèmes non-linéaires.

\section{Méthodes numériques de résolution dans \texorpdfstring{$\R$}{R}}

	\subsection{Ordre de convergence théorique}

		L'ordre de convergence d'une méthode de résolution numérique correspond à la vitesse à laquelle elle converge vers la solution $x^*$.

		\begin{definition}
			\label{def-1-convergence}
			Une méthode est d'ordre de convergence $\alpha$ si et seulement si $\exists \alpha \tq$ :
			\begin{equation}
				\label{eq-1-convergence}
				\frac{\abs{x_{k+1} - x^*}}{\abs{x_k - x^*}^\alpha} = C \qquad \text{avec } C \in \R
			\end{equation}
		\end{definition}

		Nous utiliserons cette définition pour trouver l'ordre des méthodes présentées dans ce chapitre, cependant nous voudrons représenter graphiquement ces ordres. On a donc :
		\begin{align*}
			\eqref{eq-1-convergence}
			\implies & \abs{x_{k+1} - x^*} = C \times \abs{x_k - x^*}^\alpha				\\
			\implies & \ln \abs{x_{k+1} - x^*} = \ln C + \alpha \times \ln \abs{x_k - x^*}	\\
			\implies & \ln \abs{x_{k+1} - x^*} = F(\abs{x_k - x^*})
		\end{align*}

		On peut donc représenter la convergence d'une méthode par une droite $F$ définie comme :
		\begin{equation}
			\label{eq-1-convLin}
			F(X) = \ln C + \alpha X \qquad \text{avec l'écart } X = \abs{x_k - x^*}
		\end{equation}

		Ainsi une méthode sera plus rapide et donc plus efficace qu'une autre si son coefficient $\alpha$ est supérieur; et secondairement on peut comparer les constantes $C$.



	\subsection{Méthode de la dichotomie}
		
		\subsubsection{Idée}
			La dichotomie, aussi appelée bissection, provient du grec \textit{dikhotomia} qui représente une division en deux parties. Cette méthode consiste à se rapprocher de la solution par divisions successives par deux de l'intervalle de départ contenant la solution.

			On choisit d'abord d'un intervalle $[a,b]$ contenant la solution $x^*$. 
			Le choix de l'intervalle de départ est important car si la solution n'y est pas comprise, l'algorithme ne converge pas et s'il est trop grand, la convergence sera plus longue.
			% TODO : choix intervalle de départ
			
			Pour tout entier $n$, on définit deux suites $\suite{a}$ et $\suite{b}$ correspondant aux bornes de l'intervalle à l'itération $n$ avec $a_0 = a$ et $b_0 = b$; ainsi que la suite $(x_n)$ telle que :
			$$ x_n = \frac{a_n + b_n}{2} $$
			
			A chaque itération, on compare le signe de $f(a_n)$ à celui de $f(x_n)$ et on réduit l'intervalle de moitié en affectant $x_n$ à $a_{n+1}$ ou $b_{n+1}$, ainsi $[a_{n+1}, b_{n+1}]$ se resserre sur $x^*$ et $x_{n+1}$ se rapproche de la solution.

			Pour que cette méthode soit applicable, il suffit que $f$ soit continue pour pouvoir trouver des valeurs intermédiaires comme cela est garanti par le théorème des valeurs intermédiaires.


		\subsubsection{Convergence}
			Par construction, on divise l'intervalle par deux à chaque itération donc on a :
			$$	(b_n - a_n) =  \left(\frac{1}{2}\right)^n (b_0 - a_0)	$$

			% Ainsi l'écart entre la solution approchée et la solution théorique est :
			$$	\lvert x_n-x^* \rvert \leq \frac{1}{2} (b_n-a_n) = (\frac{1}{2})^{n+1} (b_0-a_0) \limto{n}{\infty} 0	$$
			On remarque bien que l'erreur est réduite de manière quasi-linéaire.
			% TODO Quasi linéaire ? + Preuve
		

	\subsection{Méthode du point fixe}

		\subsubsection{Idée}
			On cherche $g : \R\to\R$ telle que $f(x)=0\iff g(x)=x$.
			Ainsi on ramène le problème d'annulation de la fonction $f$ a la recherche d'une point fixe pour la fonction $g$.
			\begin{exShort}
				$f(x)=x^2-2$
				$$
					x^2-2 = 0 \iff 
					g(x)=x \iff 
					g(x)=\frac{x+2}{x+1}
				$$
			\end{exShort}
			% TODO :Comment trouver g	????????

			De plus, il faut que $g$ soit dérivable et que sa dérivée au point fixe ne dépasse pas $1$ en norme.
			Alors la convergence de cette méthode est garantie par le théorème suivant :

			\begin{theoreme}
				\label{th-1-gConv}
				Soit $g : \R\to\R$ dérivable et admettant un point fixe $x^*$ telle que $\lvert g'(x^*) \rvert < 1 $ alors $\exists [a;b]$ tel que $x^* \in [a;b]$ et la suite $(x_n)_{n\geq\N}$ définie comme suit, converge vers $x^*$
				$$
					\begin{cases}
						x_{n+1}=g(x_n) \\
						x_0 \in [a;b]
					\end{cases}
				$$
			\end{theoreme}

			%\begin{preuve}
				% TODO : Avec th des accroissement fini ??
			%\end{preuve}

			En revanche il y a un problème : pour que cette méthode convergence assez rapidement $x_0$ doit être proche de $x^*$, il faut donc bien le choisir.
			% TODO : Choix ????????!!!!!!!!!!!


		\subsubsection{Convergence}
			Si $\lvert g'(x^*) \rvert < 1$ alors
			$$ \frac{\lvert x_{n+1} - x^* \rvert}{\lvert x_n - x^* \rvert} < \lvert g'(x^*) \rvert $$
			Et donc :
			$$ \lvert x_{n+1} - x^* \rvert < \lvert g'(x^*) \rvert \lvert x_n - x^* \rvert $$
			Ainsi la méthode du point fixe est d'ordre 1. Par récurrence :
			$$ \lvert x_{n+1} - x^* \rvert < \lvert g'(x^*) \rvert^n \lvert x_0 - x^* \rvert \limto{n}{\infty} 0$$
			Elle converge bien vers $x^*$ car $\lvert g'(x^*) \rvert < 1$.
			De plus la convergence de cette méthode est d'ordre 1.

	\subsection{Méthode de Newton}
	
		\subsubsection{Idée}
			Cette méthode est plus puissante mais il faut que $f$ soit deux fois continûment dérivable ($\ie f \in \Cont^2$).
			Le but est d'approcher la solution grâce à la tangente de la courbe.
			On fait un développement de Taylor-Young de $f$ en $x$.
			Soit $x_0 \in \R$ :
			$$
				f(x) = \underbrace{f(x_0) + f'(x_0)(x-x_0)}_{T_{x_0}} + \frac{(x-x_0)^2}{2}f''(x_0 +\theta(x-x_0))
			$$
			\\
			On cherche à rapprocher $x_0$ de $x$ dont l'image est la solution désirée, cette aproximation est réalisée quand la différence entre $f(x)$ et $f(x_0)$ est négligeable, c'est-à-dire quand la tangente $T_{x_0}$ s'annule.
			On définit $x_1$ tel que $T_{x_0}(x_1)=0$ donc si $f'(x_0) \neq 0 $ on a :
			$$
				x_1 = x_0 - \frac{f(x_0)}{f'(x_0)}
			$$
			Par récurrence on a alors :
			\begin{equation}
				\label{eq-1-newton}
				x_{n+1} = x_n - \frac{f(x_n)}{f'(x_n)}
			\end{equation}

			L'algorithme de la méthode est alors le suivant :

			\begin{algorithm}[H]
				\caption{Méthode de Newton}
				\KwData{$x_0 \in \R$ donné}
				\KwResult{$x^* \in \R$ la solution approchée à $\epsilon$ près}
				\While{$\abs{f(x_n)} > \epsilon$ and $f'(x_n)\neq0$}
				{
					$x_{n+1} = x_n - \frac{f(x_n)}{f'(x_n)}$\;
				}
			\end{algorithm}

			Nous avons :
			$$
				\eqref{eq-1-newton} \iff x_{n+1} = g(x_n) \text{ avec } g(x) = x- \frac{f(x)}{f'(x)}
			$$
			\begin{align*}
				g'(x) 				&= 1 - \frac{\left(f'(x)\right)^2 - f(x)f''(x)}{\left(f'(x)\right)^2}	\\
				\implies g'(x^*) 	&= 1 - \frac{\left(f'(x^*)\right)^2}{\left(f'(x^*)\right)^2} = 0
			\end{align*}
			$\abs{g'(x^*)} <1$ donc d'après le théorème \eqref{th-1-gConv} la méthode converge bien.

		\subsubsection{Convergence}
			On suppose $f\in \Cont^3$. On a :
			$$
				x_{n+1} - x^* = g(x_n) - g(^*)
			$$
			On effectue un développement de Taylor-Young : 
			$$
				g(x_n) = g(x^*) +(x_n - x^*)g'(x^*) + \frac{(x_n - x^*)}{2} g''(\xi)
			$$
			Avec $\xi$ compris entre $x_n$ et $x^*$.
			$$
				\abs{x_{n+1} - x^*} \leq C\abs{x_n - x^*}^2
			$$
			Avec $C = \frac{1}{2} \max g''(\xi)$.
			La méthode de Newton est donc d'ordre quadratique, ce qui la rend donc plus efficace.
			Concrètement cela signifie que le nombre de décimales significatives double à chaque itération.


	\subsection{Méthode de la sécante}

		\subsubsection{Idée}

			Cette méthode ressemble à celle de Newton car elle la reprend en approchant $f'(x)$ par le taux de variation. Ici la seule hypothèse est que

			\begin{algorithm}[H]
			\caption{Méthode de la sécante}
				\KwData{$x_0$ et $x_1 \in \R$ donnés}
				\KwResult{$x^* \in \R$ la solution approchée à $\epsilon$ près}
				\While{($\abs{f(x_n)}>\epsilon$)}
				{
					$x_{n+1} = x_n - \frac{f(x_n)}{f(x_n) - f(x_{n-1})}(x_n - x_{n-1})$\;
				}
			\end{algorithm}


		\subsubsection{Convergence}
			La convergence de cette méthode est d'ordre le nombre d'or $\phi =\frac{1+\sqrt{5}}{2}$.


	\subsection{Comparaison des méthodes}

		On compare expérimentalement les méthodes en cherchant à annuler $f(x) = x^2 - 2$. La solution recherchée ici est seulement $\sqrt{2}$. En effet, les différentes méthodes ont leurs paramètres centrés sur la solution poisitive, ils convergeront donc vers celle-ci.

		On a les codes Scilab suivants pour les différentes méthodes :

		\begin{listing}[H]
		\label{code-1-dichotomie}
			\scicode{\tdA 1.1-Dichotomie.sce}
			\caption{Méthode de la dichotomie}
		\end{listing}

		\begin{listing}[H]
		\label{code-1-pointFixe}
			\scicode{\tdA 1.2-Point_Fixe.sce}
			\caption{Méthode du Point Fixe}
		\end{listing}

		\begin{listing}[H]
		\label{code-1-newton}
			\scicode{\tdA 1.3-Newton.sce}
			\caption{Méthode de Newton dans $\R$}
		\end{listing}

		\begin{listing}[H]
		\label{code-1-secante}
			\scicode{\tdA 1.4-Secante.sce}
			\caption{Méthode de la sécante}
		\end{listing}

		Les méthodes sont définies comme des fonctions dont les paramères communs à toutes les méthodes sont :
		\begin{itemize}
			\item \code{f} la fonction à annuler
			\item \code{df} (dérivée de $f$) ou \code{g} voire rien selon la méthode
			\item \code{x0} ou \code{a} et \code{b} les conditions de départ spécifique à la méthode
			\item \code{ITE_MAX} le nombre maximal d'itérations autorisées avant d'arrêter la méthode
			\item \code{EPS} la tolérance pour considérer que la fonction a été annulée
		\end{itemize}

		En retour, on obtient :
		\begin{itemize}
			\item\code{x} le vecteur colonne des solutions avec $x_k$ l'approximation à l'itération $k$
			\item\code{i} l'itération a laquelle l'approximation est comprise dans le seuil de tolérance
		\end{itemize}

		Enfin on compare ces méthodes :
		\begin{listing}[H]
		\label{code-1-comparaison}
			\scicode{\tdA 1.0-Toutes_Methodes.sce}
			\caption{Comparaison des méthodes}
		\end{listing}

		La démarche est la suivante :
		on récupère les résulats des méthodes, on calcule l'écart par rapport à la solution analytique $\code{solution} = \sqrt{2}$, puis les ordres de convergence $\alpha$ par régression linéaire sur le logarithme de l'écart absolu avec \code{reglin}. On affiche ensuite l'évolution des approximations et des écarts en fonction des itérations \ref{img-1-result2} et l'ordre des méthodes \ref{img-1-ordre}.

		\begin{figure}[H]
			\centering
			\includegraphics[width=\linewidth, trim=3cm 2cm 3cm 2cm, clip]{\tdA\img 1-Toutes_Methodes_Approximation.eps}
			\caption{Résultats pour $f(x) = x^2 -2$}
			\label{img-1-result2}
		\end{figure}
		Graphiquement on voit que les méthodes convergent assez vite vers la solution.

		\begin{figure}[H]
			\centering
			\includegraphics[width=0.6\linewidth, trim=2cm 1cm 2cm 1cm, clip]{\tdA\img 1-Toutes_Methodes_Ordre.eps}
			\caption{Représentation graphique des ordres de convergence des méthodes}
			\label{img-1-ordre}
		\end{figure}

		On retrouve bien les même ordres de convergence pour chaque méthode.
		Ainsi il est préférable d'utiliser la méthode de Newton. L'outil \code{fsolve(x0, f, Jf)} du Scilab est pratique aussi. Il s'inspire d'une méthode complexe et peu documentée, la méthode hybride de Powell. Il suffit de renseigner $x_0 \in \R^n$ et $f$, la matrice Jacobienne $J_f$ est optionnelle mais permet d'éviter à l'outil de devoir l'approcher et donc d'être plus rapide. Nous comparerons la méthode de Newton à \code{fsolve} dans l'application \eqref{code-1-GPS}.

\section{Méthodes numériques dans \texorpdfstring{$\R^n$}{Rn}}

	\subsection{Rappels}
	% TODO : mettre dans maths ?
		\begin{definition}[Différentiabilité]
			Soit $f: \R^n \to \R^n$ et $x_o \in \R^n$.
			\\$f$ est différentiable en $x_0$ si $\exists J_f \in \M_{n,n} \tq \forall h \in \R^n$ :
			$$
				f(x_0 +h) = f(x_0) + J_f h + \norm{h} \epsilon(h)
			$$
			où $lim_{h \to 0} \epsilon(h) = 0$ et la norme utilisée est la norme euclidienne.
			La matrice Jacobienne de $f$ en $x_0$ est $J_f(x_0)$ telle que : $(J)_{ij} = \dfdp{f_i}{x_j}(x_0)$.
		\end{definition}

	\subsection{Méthode de Newton dans \texorpdfstring{$\R^n$}{Rn}}

		Une version de la méthode de Newton existe aussi pour le cas plus général d'un système dynamique dans $\R^n$. On remplace alors la dérivée de $f$ par sa matrice Jacobienne. Il faut donc que $f$ soit différentiable.

		Soit $x_0 \in \R^n$ choisi. $\forall x \in \R^n$ :
		$$
			f(x) = \underbrace{f(x_0) + f'(x_0)(x-x_0)}_{T_{x_0}} + \frac{(x-x_0)^2}{2}f''(x_0 +\theta(x-x_0))
		$$
		avec $T_{x_0}$ le plan tangent en $x$.
		On définit $x_1$ \tq 
		\begin{equation}
			\label{eq-1-planTangent}
			T_{x_0}(x_1) = 0
		\end{equation}
		ce qui forme un système d'équations linéaires.
		\begin{align*}
			\eqref{eq-1-planTangent} \iff		& J_f(x_0)\times(x_1 -x_0) = -f(x_0)		\\
									\implies 	& x_1 = x_0 - f(x_0) \times \left(J_f(x_0)\right)^{-1}
		\end{align*}
		En pratique, il est préférable de résoudre le système linéaire \eqref{eq-1-planTangent} avec la méthode de Gauss, que d'inverser la jacobienne pour calculer $x_1$.

		\begin{algorithm}[H]
		\caption{Méthode Newton dans $\R^n$}
			\KwData{$x_0 \in \R^n$ donné}
			\KwResult{$x^* \in \R^n$ la solution approchée à $\epsilon$ près}
			\While{($\norm{f(x_n)}>\epsilon$ et $J_f(x_n)$ inversible)}
			{
				$x_{n+1} = x_n - J_f(x_n) \backslash f(x_n)$\;
			}
		\end{algorithm}

		On a l'implémentation dans Scilab suivante :
		\begin{listing}[H]
			\scicode{\tdA 1.5-NewtonRn.sce}
			\caption{Méthode de Newton dans $\R^n$}
			\label{code-1-newtonRn}
		\end{listing}

		Les paramères sont :
		\begin{itemize}
			\item \code{f} la fonction à annuler
			\item \code{Jf} la Jacobienne de $f$
			\item \code{x0} $\in\M_{n1}$ les conditions initials
			\item \code{ITE_MAX} le nombre maximal d'itérations autorisées avant d'arrêter la méthode
			\item \code{EPS} la tolérance pour considérer que la fonction a été annulée
		\end{itemize}
		En retour on obtient \code{x} $\in \M_{ni}$ où $i$ est le nombre d'itérations effectuées. On obtient donc l'évolution de l'approximation à chaque colonne.


\section{Applications}

	\subsection{Cinématique inversée}
		Savoir résoudre des problèmes non-linéaires est intéressant dans de nombreux domaines, dont celui de la robotique. On a par exemple les problèmes de cinématique inversée : le but est de retrouver la bonne configuration d'un bras robotique pour qu'il atteigne un point particulier sous certaines contraintes.
		% Nous pouvons schématiser un tel problème par la figure suivante :

		% TODO : Figure bras

		On modélise ensuite la situation par le système suivant :
		\begin{equation}
		\label{eq-1-bras}
			M(\theta)=A \iff
			\begin{cases}
				l_1\cos(\theta_1) + l_2\cos(\theta_1 + \theta_2) - x_A &= 0 \\
				l_1\sin(\theta_1) + l_2\sin(\theta_1 + \theta_2) - y_A &= 0 				
			\end{cases}
		\end{equation}
		
		On pose $f : \R^2 \to \R^2 \tq$
		\begin{equation}
		\label{eq-1-fBras}
			f(\theta) = 
			\begin{pmatrix}
				l_1\cos(\theta_1) + l_2\cos(\theta_1 + \theta_2) - x_A \\
				l_1\sin(\theta_1) + l_2\sin(\theta_1 + \theta_2) - y_A			
			\end{pmatrix}
		\end{equation}

		On a équivalence entre \eqref{eq-1-bras} et $f(\theta) = 0$.

		On résout alors \eqref{eq-1-fBras} par une méthode numérique, ici celle de Newton.
		Pour cela, nous devons calculer la Jacobienne de $f$ :
		$$
			J_f(\theta) = \begin{pmatrix}
				-l_1\sin(\theta_1) - l_2\sin(\theta_1 + \theta_2)
				&	-l_2\sin(\theta_1 + \theta_2)
				\\
				l_1\cos(\theta_1) + l_2\cos(\theta_1 + \theta_2)
				&	l_2\cos(\theta_1 + \theta_2)
			\end{pmatrix}
		$$

		Nous avons le programme Scilab suivant :

		\begin{listing}[H]
			\scicode{\tdA 2.3-Cinematique_Inversee.sce}
			\caption{Cinématique inversée}
			\label{code-1-cinematiqueInversee}
		\end{listing}

		On cherche aussi à décrire une trajectoire circulaire de centre $(1,1)$ et de rayon $0.5$.
		Pour cela on calcules les \code{NB_POSITIONS} positions successives lorsque le bras atteint le point $M$, ce dernier variant autour du cercle à chaque itération.
		La fonction \code{dessine_bras(X,1)} dessine le bras avec les angles $X_1$ et $X_2$ et le cercle si le second paramètre est égale à 1. 
		On obtient les positions successives du bras articulé :

		\begin{figure}[H]
			\centering
			\includegraphics[width=0.7\linewidth, trim=1cm 2.5cm 1cm 2.5cm,clip]{\tdA\img 2-Trajectoire_Cercle.eps}
			\caption{Mouvement autour d'un cercle}
			\label{img-1-brasCercle}
		\end{figure}


		
		% TODO : finir




		% Résolution en Cartésien
		%	Les coordonnées $(x,y)$ du coude sont telles que :
		%	\begin{equation}
		%	\label{eq-1-coude}
		%		\begin{cases}
		%			x^2 + y^2 - l_1^2 = 0				& \quad \textit{Partie OA du coude}	\\
		%			(x-x_A)^2 + (y-y_A)^2 - l_2^2 = 0	& \quad \textit{Partie AM du coude}
		%		\end{cases}
		%	\end{equation}
		%	$$
		%		% \eqref{eq-1-coude}
		%		\iff g(x,y) = 
		%		\begin{pmatrix}
		%			x^2 + y^2 - l_1^2	\\
		%			(x-x_A)^2 + (y-y_A)^2 - l_2^2
		%		\end{pmatrix}
		%		= \vec0
		%	$$


	\subsection{GPS}
	\label{ch-1-gps}

		Pour se localiser dans l'espace, un GPS (\textit{Global Positioning System} ou en français \textit{Géo-Positionnement par Satellite}) résoud un problème non-linéaire dans $\R^3$. Pour fonctionner, il doit être connecté à au moins 3 satellites afin de garantir une solution unique. En connaissant la distance\footnote{Le satellite envoie un signal contenant l'heure d'émission, le GPS peut alors calculer la distance à partir de l'heure de réception et de la vitesse de l'onde.} qui les séparent et leur position dans l'espace, on forme un système d'équations non-linéaires.
		De plus des problèmes de synchronisation temporelle entre les satellites et le GPS peuvent s'ajouter pour calculer les erreurs.

		On va ici résoudre ce problème dans le cas de 3 satellites sans tenir compte de problèmes temporels.
		Soient $S_1, S_2, S_3\in\M_{31}$ tels que $S_i=(x_i,y_i,z_i)$ les positions des satellites dans l'espace, ainsi que $d\in\M_{31}$ où $d_i$ est la distance séparant le satellite $S_i$ au GPS.
		On cherche $X = (x,y,z) \in\M_{31}$ tel que $f(X) = 0$ :
		\begin{equation}
			\label{eq-1-fGPS}
			f(X) = \begin{pmatrix}
				\norm{X - S_1}^2 - d_1	\\
				\norm{X - S_2}^2 - d_2	\\
				\norm{X - S_3}^2 - d_3
			\end{pmatrix}
		\end{equation}

		On pose : $g_i(X) = \norm{X - S_i}^2$

		Pour trouver la Jacobienne sans calculer chaque dérivée partielle on effectue le développement limité suivant :

		\begin{align*}
			g_i(X+h) 	&= \norm{(X-S) +h}^2								\\
						&= \left((X-S)+h\right)^T\left((X-S)+h\right)		\\
						&= (X-S)^T(X-S) + h^T(X-S) + (X-S)^T h + h^Th		\\
						&= \underbrace{\norm{X-S}^2}_{g(X)} + \underbrace{2(X-S)^T}_{J_g(X)}h + \underbrace{\norm{h}^2}_{\text{Reste}}
		\end{align*}

		Ainsi on a la Jacobienne :
		\begin{equation}
			\label{eq-1-jGPS}
			J_g(X) = 2\times\begin{pmatrix}
								(X-S_1)^T \\
								(X-S_2)^T \\
								(X-S_3)^T
							\end{pmatrix}
		\end{equation}

		On peut donc résoudre le problème avec Scilab.
		\begin{listing}[H]
			\scicode{\tdA 2.2-GPS.sce}
			\caption{Simulation d'un GPS}
			\label{code-1-GPS}
		\end{listing}

		Ici, j'ai fait diverses comparaisons :
		\begin{itemize}
			\item entre la méthode de Newton et l'outil \code{fsolve} de Scilab \ref{tb-1-resultGPS}
			\item entre la matrice Jacobienne \eqref{eq-1-jGPS} et celle calculée par Scilab avec \code{numderivative}
		\end{itemize}

		J'ai alors obtenu les coordonnées et leur écart absolu :
		\begin{table}[H]
			\centering
			\begin{tabular}{|r|r|r|}
				\hline
				Méthode de Newton	& Macro \code{fsolve}	& Écart absolu				\\	\hline
				595.0250498015592	& 595.0250498015607		& 1.4779288903810084E-12	\\	\hline
				-4856.025050498366	& -4856.025050498369	& 2.7284841053187847E-12	\\	\hline
				4078.329999324317	& 4078.3299993243168	& 4.547473508864641E-13		\\	\hline
			\end{tabular}
			\caption{Résulats de la simulation du GPS}
			\label{tb-1-resultGPS}
		\end{table}

		On obtient une altitude par rapport au rayon moyen de la Terre de $-1.7135655$ m. On est ainsi relativement proche de la surface du globe car les données possèdent des incertitudes de même ordre. 
		% TODO : Finir, virer ou refaire le tableau


	\chapter{Fractales}
\label{ch-2}

% \section{Introduction}

	Les fractales ont été définies par Benoit Mandelbrot en 1975 dans son oeuvre \emph{Les Objets Fractals}.
	Ce sont des objets complexes que la géométrie traditionnelle peine à décrire. Le plus souvent, ce sont des figures qui se répètent à l'infini avec n'importe quel niveau de zoom : on parle d'autosimilarité à toutes les échelles.

	Les fractales peuvent être appliquées à de nombreux domaine :
	\begin{itemize}
		\item en médecine avec la modélisation d'un poumon
		\item la modélisation de structures de plantes comme le chou romanesco
		\item en finance avec la prévision de krachs boursiers par la théorie multifractale
		\item en urbanisme avec les murs antibruits
		\item en géologie avec l'étude du relief
		\item mais aussi dans les arts, par leur aspect intriguant et infini
	\end{itemize}

	Nous allons ici étudier comment générer des fractales, puis nous verrons quelques exemples.
	% Mandelbrot : un objet qui continue à présenter une structure détaillée sur un grand éventail d'échelles
	% irrégularité ?
	% Comment construire des figures irrégulières (???) telles que des courbes entières non dérivables
	% simuler l'écume des vagues
	% G. Cantor (1845-1918)
	% Van Koch (1870-1924)
	% Sierpinski (???) (1882-1969)
	% Weirstrass
	% Mandelbrot (1920-2014)
	% Au-delà de la géométrie euclidienne classique
	% Paradoxe de l'infini

\section{Outils}

	\subsection{Distance et dimension de Hausdorff}
		% Différence par rapport à la dimension topologique que l'on connait déjà.
		% Dans le cas d'une fractale, sa dimension topologique est strictement inférieure à sa dimension de Hausdorff. (???)

		On cherche ici à mesurer convenablement la distance entre deux compacts de $\R^2$.
		On définit la distance entre un point $x$ et un compact $B$ par : $ d(x,B) = \min_{b\in B}{d(x,b)} $
		Naïvement on pourrait définir la distance donc par : $d(A,B) = \max_{a\in A}{d(a,B)}$
		Mais on a alors un problème car $d(A,B)\neq d(B,A)$.
		Nous avons une solution :

		\begin{definition}%[Distance de Hausdorff]
			\label{def-2-distanceHausdorff}
			On définit la distance de Hausdorff $d_H$ entre deux compacts $A$ et $ B$ par :
			\begin{equation}
				\label{eq-2-distanceHausdorff}
				d_H(A,B) = \max \left( d(A,B), d(B,A) \right) = \max \left( \max_{a\in A}d(a,B) ;~ \max_{b\in B}d(b,A)  \right)
			\end{equation}
		\end{definition}
			
		\medskip

		De plus, on définit la dimension de Hausdorff $d$ d'un compact $K \subset \R^2$ :

		\begin{definition}[Dimension de Hausdorff]
			\label{def-2-hausdorff}
			On note $N(\epsilon)$ le nombre de carrés ou disques de longueur $\epsilon$ recouvrant un compact $K$, on a la dimension de Hausdorff de $K$ notée $\dim_H$ telle que :
			\begin{equation}
				\label{eq-2-dimHaussdorffEps}
				\dim_H = \lim_{\epsilon\to 0} \frac{\ln{N(\epsilon)}}{\ln{\frac{1}{\epsilon}}}		
			\end{equation}
		\end{definition}

		On peut aussi l'expliquer plus simplement.
		On prend un élément d'un compact $K$ et on le réduit par d'un facteur $k$ (on a le rapport de réduction $r = \frac{1}{k}$), pour recouvrir le compact initial il faut $n$ réductions. On a donc $n\times k^d = 1$ d'où la dimension :
		\begin{equation}
			\label{eq-2-dimHaussdorff}
			d = \frac{\ln n}{\ln k}
		\end{equation}

		\begin{ex}
			Pour $\R$ : on prend un segment que l'on réduit de $k$, on a besoin de $n=k$ segments réduits.	\\
			Pour $\R^2$ : on prend un carré que l'on réduit de $k$, on a besoin de $n=k^2$ carrés réduits.	\\
			Pour $\R^3$ : on prend un cube que l'on réduit de $k$, on a besoin de $n=k^3$ cubes réduits.	\\
			Par exemple si on réduit un cube d'un facteur $2$, il faut $8 = 2^3$ cubes réduits.
		\end{ex}

		% Elle est utilisée pour caractériser les fractales.
		% TODO Développer

	\subsection{Système de Fonctions Itérées (ISF)}
		Les ISF sont des algorithmes qui permettent de construire des fractales de manière itérée, évitant ainsi les méthodes récursives. Avant de les définir plus précisément nous allons voir les transformations utilisées dans ces méthodes.

		\medskip

		Soit $u:\R^2\to\R^2$ une transformation linéaire du plan et soit $A$ la matrice associée à cette application.
		Les transformations les plus courantes sont :

		\begin{table}[H]
			\centering
			\begin{tabular}{lc}
				l'homothétie de rapport $\lambda$ :
				&	$A = \begin{pmatrix}
						\lambda & 0 		\\
						0 		& \lambda
					\end{pmatrix}$
				\\
				\\
				la rotation d'angle $\theta$ et de centre l'origine :
				&	$A = R_\theta = \begin{pmatrix}
						\cos\theta	& \sin\theta	\\
						-\sin\theta	& \cos\theta
					\end{pmatrix}$
				\\
				\\
				la symétrie axiale par rapport à l'axe des ordonnée :
				&	$A = S_{Ox} = \begin{pmatrix}
						1 & 0	\\
						0 & -1
					\end{pmatrix}$
				\\
				\\
				la combinaison de rotation $\theta$ et d'homothétie $\rho$:
				&	$A = \begin{pmatrix}
				% \label{eq-2-rotHomo}			
						a & -b	\\
						b & a
					\end{pmatrix}$
				\\
				\multicolumn{2}{c}{
					\smallskip
					Où on a : \hfill
					$
						\rho = \sqrt{a^2 + b^2}
							\hfill
						\cos\theta = \frac{a}{\rho}
							\hfill
						\sin\theta = \frac{-b}{\rho}
					$
				} \\		
			\end{tabular}
			\caption{Transformations courantes}
			\label{tb-2-transf}
		\end{table}

		% TODO : schéma


		% =========================================================================

			% \begin{itemize}
			% 	\item l'homothétie de rapport $\lambda$ : \hfill
			% 	$A = \begin{pmatrix}
			% 		\lambda & 0 		\\
			% 		0 		& \lambda
			% 	\end{pmatrix}$

			% 	\item la rotation d'angle $\theta$ et de centre l'origine : \hfill
			% 	$A = R_\theta = \begin{pmatrix}
			% 		\cos\theta	& \sin\theta	\\
			% 		-\sin\theta	& \cos\theta
			% 	\end{pmatrix}$

			% 	\item la symétrie axiale par rapport à l'axe des ordonnée : \hfill
			% 	$A = S_{Ox} = \begin{pmatrix}
			% 		1 & 0	\\
			% 		0 & -1
			% 	\end{pmatrix}$

			% 	\item la combinaison de rotation $\theta$ et d'homothétie $\rho$: \hfill
			% 	$A = \begin{pmatrix}
			% 		a & -b	\\
			% 		b & a
			% 	\end{pmatrix}$

			% 	Où on a : \hfill
			% 	$$
			% 		\label{eq-2-rotHomo}			
			% 		\rho = \sqrt{a^2 + b^2}
			% 			\hfill
			% 		\cos\theta = \frac{a}{\rho}
			% 			\hfill
			% 		\sin\theta = \frac{-b}{\rho}
			% 	$$			
			% \end{itemize}
		% =========================================================================

		En revanche, la translation par $t\in\R^2$ n'est pas linéaire car $u(0) = 0 + t \neq 0$. Il s'agit d'une transformation affine.
		\begin{definition}[Transformation Affine]
			\label{def-2-transfAffine}
			Une transformation affine $T:\R^2\to\R^2$ d'un point $M(x;y)$ est la composition d'une transformation linéaire $A$ et d'une translation $t$ :
			\begin{align}
				\label{eq-2-transAffine}
				T(M) &= AM + t = (ax + by +e,\,cx + dy + f)	\\
				&= \begin{pmatrix}
					a & b \\ c & d
				\end{pmatrix}
				\binom{x}{y} + \binom{e}{f}
			\end{align}
		\end{definition}

		% TODO : Colinéarité

		\medskip

		\begin{definition}[Compact]
			\label{def-2-compact}
			Un compact de $\R^2$ est un sous-ensemble fermé et borné de $\R^2$
		\end{definition}

		\begin{definition}[Contraction]
			\label{def-2-contraction}
			Une transformation affine du plan est une contraction de facteur $r\in ]0;1[$ si l'image d'un segment est un segment de longueur inférieur (longueur initial $\times r$).
		\end{definition}

		\smallskip

		\begin{definition}[Système de Fonctions Itérées]
			\label{def-2-ifs}
			Un Système de Fonctions Itérées (ISF) est une collection finie de $n$ transformation affines $T_i$.
		\end{definition}

		\begin{definition}[Attracteur]
			\label{def-2-attracteur}
			 L’attracteur d’un IFS de ${T_1,T_2,...,T_n}$ est l’unique compact $K\in\R^2$ tel que
			 \begin{equation}
			 	\label{eq-2-attracteur}
			 	K = T(K) = \bigcup_{i=1}^n T_i(K) = T_1(K) \cup T_2(K) \cup ... \cup T_n(K)
			 \end{equation}
		\end{definition}

		\medskip

		On pose : $E_k$ le compact formé à l'itération $k$ d'un ISF à partir du compact de départ $E_0 \subset \R^2$ : $E_{k+1} = T_i(E_k)$.
		On parle d'ISF aléatoire quand la transformation $T_i$ est choisie aléatoirement avec un probabilité $p_i$ parmi les $n$ transformations possibles. Il faut que $\sum_{i=1}^n p_1 = 1$. Ce type d'ISF converge de la même façon que la version déterministe.
		% TODO : Jeu du chaos
		% Le jeu du chaos décrit par Michael Barnsley, désigne

		\begin{theoreme}[Théorème de Banach]
			\label{th-2-convergenceISF}
			Si $f: \R^2\to\R^2$	est une contraction de facteur $r \in ]0;1[ $ telle que
			$$
				d_H(f(A),f(B)) \leq r \times d_H(A,B)
			$$
			alors il existe un point fixe $K$ appelé attracteur tel que $K=f(K)$.
		\end{theoreme}

		Ce théorème nous permet alors de prouver la convergence des ISF.
		La suite $\suite[k]{E}$ converge vers l'attracteur $K$ décrit en \eqref{eq-2-attracteur} si ses transformations $T_i$ sont contractantes selon la distance de Haussdorf vue en \ref{def-2-hausdorff}.

		Comme tous les ISF que nous construirons par la suite n'auront que des transformations contractantes, alors ils convergeront vers leur point fixe qui correspond à leur fractale.


		% TODO Utile ? ou Maths ?
		% \bigskip
		% \begin{theoreme}
		% 	\begin{enumerate}
		% 		\item $f : [a;b] \to [a;b]$ alors \quad $\exists! x^* \tq f(x^*)=x^*$
		% 		\item Si $f$ est k-lipschitzienne alors \quad $\exists K \in ]0;1[ \tq \forall x,y :\quad \abs{f(x)-f(y)} \leq K\abs{x-y}$
		% 	\end{enumerate}
		% \end{theoreme}


\section{Quelques fractales}

	\subsection{Ensemble de Mandelbrot}

		
	
		% Ici chaque pixel correspond à un c à tester
		% Trouver Pi
		% N(c) = nombre d'itérations avant que z_c >2
		% c = 1/ + epsilon
		% epsilon = .01 .001 ->
		% N(c) -> Pi * 10^k ?
		% Casper ??

			% // On itére les complexes de tous les points : z(n+1) = zn^2 + z0 avec z0 = c
			L'ensemble de Mandelbrot est sûrement l'une des fractales les plus connues et les plus étudiées. Elle a été découvert par Gaston Julia et Pierre Fatou, puis repris par Benoît Mandelbrot qui lui donnera son nom. 

		\paragraph{Construction}

			On définit l'ensemble de Mandelbrot $\mathbb{M}$ de la façon suivante :
			un nombre complexe $c$ appartient à $\mathbb{M}$ si la suite \eqref{eq-2-mandelbrot} est bornée et donc ne diverge pas.
			\begin{equation}
				\label{eq-2-mandelbrot}
				\begin{cases}
					z_{n+1} = z_n^2 + c		\\
					z_0 = 0
				\end{cases}
			\end{equation}
			On considère que comme borne suffisante 2 : $\abs{z_n} < 2 ~\forall n\in\N$ car si $\abs{z_n}\geq 2$ alors $\abs{z_{n+1}}\geq 2z_n$ et la suite diverge. 

			On a l'implémentation sous Scilab suivante :
			\begin{listing}[H]
				\scicode{\tdB 1-Mandelbrot.sce}
				\caption{Ensemble de Mandelbrot}
				\label{code-2-mandelbrot}
			\end{listing}

			On crée une grille de $800\times600$ pixels, et à chaque point $(x,y)$ on associe le complexe $c = x + iy$.
			On va tester si chaque complexe correspondant ne diverge pas pour l'inclure dans l'ensemble de Mandelbrot.
			On définit la matrice complexe \code{Z} qui correspond aux coordonnées complexes de chaque point et $\code{C}=Z_0$.
			On itère ensuite chaque point dans $Z$ avec la fonction \eqref{eq-2-mandelbrot}.

			La matrice \code{InMandelbrot} permet de savoir si les complexes restent bornés : on met 1 si le point correspond n'a pas divergé après les itérations, il appartient ainsi à l'ensemble de Mandelbrot; sinon on laisse à 0. On affiche ensuite les points à 1 et on obtient :
			\begin{figure}[H]
				\centering
				\includegraphics[width=0.8\linewidth]{\tdB\img 1-Mandelbrot.png}
				\caption{Ensemble de Mandelbrot}
				\label{img-2-mandelbrot}
			\end{figure}

		\paragraph{Propriétés}
			\begin{propShort}
				L'ensemble de Mandelbrot a été démontré connexe.
			\end{propShort}

			On peut aussi retrouver les décimales de $\pi$ avec l'ensemble de Mandelbrot.
			On note $N(z)$ le nombre d'itérations nécessaires avant que le point d'affixe $z$ ne diverge de l'ensemble de Mandelbrot.
			Prenons la suite $z_i = \frac{1}{4} + 10^{-2i} ~\forall i \in \N^*$. On calcules $N(z_i)$ avec le programme suivant :

			\begin{listing}[H]
				\scicode{\tdB 1.1-Mandelbrot_Pi.sce}
				\caption{Pi et Mandelbrot}
				\label{code-2-piMandelbrot}
			\end{listing}

			On récupère les résultats suivants :

			\begin{table}[H]
				\centering
				\begin{tabular}{|c|l|l|}
					\hline
					$i$	& $z_i$				& $N(z_i)$	\\	\hline
					\hline
					$1$	& $0.26$			& $28$		\\	\hline
					$2$	& $0.2501$			& $310$		\\	\hline
					$3$	& $0.250001$		& $3138$	\\	\hline
					$4$	& $0.25000001$		& $31412$	\\	\hline
					$5$	& $0.2500000001$	& $314155$	\\	\hline
					% $6$	& $0.250000000001$	& $3141623$	\\	\hline
				\end{tabular}
				\caption{Retrouver les décimales de $\pi$ avec l'ensemble de Mandelbrot}
				\label{tb-2-piMandelbrot}
			\end{table}

			On retrouve étonnement les décimales de $\pi$ dans $N(z_i)$ quand $i$ augmente. En comparaison, on a : $\pi = 3.1415927$.
			Ainsi il s'agit d'une méthode originale pour retrouver les décimales de $\pi$ mais extrêmement lente.

	\subsection{Ensembles de Julia}

		Soit $c\in\C$ fixé, les complexes de la suite $\suite{z}$ appartiennent à l'ensemble de Julia $\mathbb{J}_c$ s'il sont bornés par 2 de la même manière que pour l'ensemble de Mandelbrot.
		\begin{equation}
			\label{eq-2-julia}
			\begin{cases}
				z_{n+1} = z_n^2 + c			\\
				z_0 \in \C
			\end{cases}
		\end{equation}	

		On a l'implémentation sous Scilab suivante :
		\begin{listing}[H]
			\scicode{\tdB 2-Julia.sce}
			\caption{Ensemble de Julia}
			\label{code-2-julia}
		\end{listing}		

		On crée de la même manière que précédemment une grille de points complexes que l'on va itérer, sauf qu'ici chaque complexe correspond à $z_0$ et $c$ est fixé.
		\code{Cv} est une matrice binaire : si $z$ est encore borné alors $\code{Cv(z)} = 1$.
		\code{Color} est la matrice de coloration de la map pixel en fonction du nombre d'itérations effectuées avant que le complexe ne diverge : on augmente \code{Color(x,y)} tant que le complexe associé est borné.

		On a choisi ici plusieurs valeurs de $c$ donnant les fractales suivantes :
		\begin{figure}[H]
			\centering
			\includegraphics[width=\linewidth]{\tdB\img 2-Julias.png}
			\caption{Ensembles de Julia}
			\label{img-2-julia}
		\end{figure}

		On observe que pour $c\in\mathbb{M}$ et que ces ensembles sont connexes.
		Il s'avère que pour chaque $c\in\mathbb{M}$, l'ensemble de Julia $\mathbb{J}_c$ est connexe.


	\subsection{Ensemble de Cantor}

		% Cantor (1845-1918) est un mathématicien allemand fondateur de la théorie des ensembles.

		% TODO Introduction avec les caractéristiques principales ?

		\paragraph{Construction}
			On construit l'ensemble de Cantor $E_n$ de profondeur $n$ de la manière suivante :

			\begin{algorithm}[H]
			\DontPrintSemicolon
			\caption{Ensemble de Cantor}
				\KwData{
					Segment $E_0 =[0,1]$\;
					\Indp\Indp$n$ la profondeur désirée\;
				}
				\KwResult{$E_n$ l'ensemble de Cantor de profondeur $n$}
				\For{$i$ allant de $0$ à $n$}
				{
					Partager chaque segment en trois parties égales\;
					Retirer la partie centrale de chaque segment\;
				}
			\end{algorithm}

			Cet algorithme décrit un ISF déterministe avec les transformations suivantes :
			\begin{align*}
				% \label{eq-2-isfCantor}
				T_1(x) = \frac{1}{3}x				\\
				T_2(x) = \frac{1}{3}x + \frac{2}{3}	\\
				E_{k+1} = T_1(E_k) \cup T_2(E_k)
			\end{align*}

			D'après le théorème \eqref{th-2-convergenceISF}, cet ISF converge bien vers l'ensemble triadique de Cantor $E_\infty$ \tq:
			\begin{equation}
				\label{eq-2-cantor}
				E_\infty = \bigcap_{i=0}^{\infty} E_i
			\end{equation}
			où $E_i$ l'ensemble obtenu à l'étape $i$.

			On implémente l'ISF déterministe sous Scilab :
			\begin{listing}[H]
				\scicode{\tdB 3-Cantor.sce}
				\caption{Ensemble de Cantor}
				\label{code-2-cantor}
			\end{listing}

			Le déterministe se fait par récursivité. On obtient le résultat suivant :
			\begin{figure}[H]
				\centering
				\includegraphics[width=.8\linewidth]{\tdB\img 3-Cantor.eps}
				\caption{Ensemble de Cantor de profondeur $n=7$}
				\label{img-2-cantor}
			\end{figure}

			% Nous allons voir les diverses propriétés de cet ensemble particulier dans la partie suivante.

		\paragraph{Propriétés}

			L'ensemble de Cantor est auto-similaire dans le sens où le local et le global se ressemblent, c'est-à-dire qu'à différentes échelles la figure paraît la même. 
			Sa structure est telle qu'elle ne peut être décrite par la géométrie traditionnelle mais seulement par la représentation triadique suivante.

			\begin{theoreme}
				On a la décomposition en base 3 suivante $\forall x \in E_\infty$ :
				\begin{align}
					\label{eq-2-cantorDecomposition}				
					\underline{x}_{10}	&= \sum_{k=1}^\infty \frac{x_i}{3^i}	\\
				\iff\underline{x}_{3} 	&= 0,x_1x_2x_3\ldots x_k\ldots
				\end{align}
				Avec $x_k = 0$ ou $2$, $\forall k$
			\end{theoreme}
			\begin{ex}
				On a la décomposition suivante :
				\begin{align*}
					\left(\frac{1}{3}\right)_3 	& = 0,02222222\ldots	\\
												& = 0 + \frac{0}{3} + \frac{2}{3^2} + \ldots + \frac{2}{3^k}	\\
												& = \frac{2}{3^2} \left(1 + \frac{2}{3} + \ldots +\frac{2}{3^{k-2}}\right)	\\
												& = \frac{2}{9} \times \frac{1}{1-\frac{1}{3}} = \frac{1}{3} = 0.1
				\end{align*}
				Donc $\frac{1}{3} \in E_\infty $
				% $$ (0,1)_{10} = (0.0999\ldots)_{10} $$
			\end{ex}

			\begin{propShort}
				$\abs{E_\infty} = 0$
			\end{propShort}
			\begin{preuve}
				On a : $\abs{E_0} = 1$, $\abs{E_1} = \frac{2}{3}$, $\abs{E_2} = \frac{4}{9} = \left(\frac{2}{3}\right)^2$, etc.
				Par récurrence on obtient : $ \abs{E_k} = \left( \frac{2}{3}\right)^k $.
				Donc $\abs{E_k} \limto{k}{\infty} 0$.
			\end{preuve}
					

			\begin{propShort}
				$E_\infty$ est indénombrable.
			\end{propShort}
			\begin{preuve}
				Montrons que l'ensemble $E_\infty$ est en bijection avec $[0;1]$ qui est indénombrable.
				$$
					\forall x \in E_\infty \text{ : } \eqref{eq-2-cantorDecomposition} \to \frac{x}{2} = 0,\frac{x_1}{2}\frac{x_2}{2}\ldots\frac{x_k}{2}\ldots \text{ avec } \frac{x_k}{2}\in [0;1]
				$$
				On associe à $\frac{x}{2}$ le nombre en base 2 : $0,y_1y_2y_3\ldots y_k\ldots$ avec $y_k \in [0;1]$
			\end{preuve}

		\paragraph{Dimension}

			La dimension de $E_\infty$ est :
			$$
				\left.\begin{aligned}
					r &= \frac{1}{3}	\quad	\\
					N &= 2	\quad
					\end{aligned}
				\right\}
				\qquad \text{à l'étape $k$ : } \quad 2^k=(3^k)^d \implies 2=3^d \implies d= \frac{\ln{2}}{\ln{3}} \in ]0;1[
			$$


	\subsection{Flocon de Von Koch}

		% Le flocon de Von Koch correspond à la même procédure que 
		% Algorithme :
		% Figure de départ : un triangle
		% Opération : chaque segment est partagé en trois segments de même longueur.
		% Le segment central est alors remplacé par deux segments égaux formant un triangle équilatéral ayant pour base le segment retiré.


		\paragraph{Construction}

			On construit le flocon de la manière suivante :

			\begin{algorithm}[H]
			\DontPrintSemicolon
			\caption{Flocon de Von Koch}
				\KwData{
					Triangle équilatéral $C_0$ de coté $1$\;
					$n$ la profondeur désirée\;
				}
				\KwResult{$C_n$ le flocon de Von Koch de profondeur $n$}
				\For{$i$ allant de $1$ à $n$}
				{
					Partager chaque segment en 3 segments égaux\;
					Remplacer le segment central par deux segments égaux formant un triangle équilatéral ayant pour base le segment remplacé\;
				}
			\end{algorithm}

% 			Il faut effectuer un homothétie de rapport $r=\frac{1}{3}$, deux rotations de $60$° puis $120$° et deux translations pour former le triangle équilatéral. En partant de , on peut retrouver les matrice $A_1$ et $A_2$.
% % \eqref{eq-2-rotHomo}
% 			Pour le premier segment, on a la tranformation $T_1(M) = A_1M + t_1$ avec :
% 			\begin{equation}
% 				\label{eq-2-tFlocon1}
% 				\begin{array}{lc}
% 					\begin{cases}
% 						\rho = \frac{1}{3}
% 						\\
% 						\theta_1 = \frac{\pi}{3}
% 					\end{cases}
% 					&
% 					\implies\begin{cases}
% 						a_1 = \rho \times \cos\theta_1 = \frac{1}{3} \times \frac{1}{2} = \frac{1}{6}
% 						\\
% 						b_1 = \rho \times \sin\theta_1 = \frac{1}{3} \times \frac{\sqrt{3}}{2} = \frac{1}{2\sqrt{3}}
% 					\end{cases}
% 					\\
% 					\\
% 					A_1 = \begin{pmatrix}
% 						\frac{1}{6}				& -\frac{1}{2\sqrt{3}} \\
% 						\frac{1}{2\sqrt{3}}		& \frac{1}{6}
% 					\end{pmatrix}
% 					&
% 					t_1 = \begin{pmatrix} \frac{1}{3}	\\ 0	\end{pmatrix}
% 				\end{array}
% 			\end{equation}

% 			Pour le second segment, $T_2(M) = A_2M + t_2$ :
% 			% TODO redo the T_1-4
% 			\begin{equation}
% 				\label{eq-2-tFlocon2}
% 				\begin{array}{lc}
% 					\begin{cases}
% 						\rho = \frac{1}{3}
% 						\\
% 						\theta_1 = \frac{2\pi}{3}
% 					\end{cases}
% 					&
% 					\implies\begin{cases}
% 						a_2 = \rho \times \cos\theta_2 = \frac{1}{3} \times -\frac{1}{2} = -\frac{1}{6}
% 						\\
% 						b_1 = \rho \times \sin\theta_2 = \frac{1}{3} \times \frac{\sqrt{3}}{2} = \frac{1}{2\sqrt{3}}
% 					\end{cases}
% 					\\
% 					\\
% 					A_2 = \begin{pmatrix}
% 						-\frac{1}{6}			& -\frac{1}{2\sqrt{3}} \\
% 						\frac{1}{2\sqrt{3}}		& -\frac{1}{6}
% 					\end{pmatrix}
% 					&
% 					t_2 = \begin{pmatrix} \frac{2}{3}	\\ 0	\end{pmatrix}
% 				\end{array}
% 			\end{equation}

			On obtient le résultat suivant : 

			\begin{figure}[H]
				\centering
				\includegraphics[width=.4\linewidth]{\tdB\img vonKoch.png}
				\caption{Flocon de Von Koch}
				\label{img-2-vonKoch}
			\end{figure}

			La courbe de Von Koch correspond à un segment du flocon.
			Elle est de longueur infinie mais contenant dans une surface d'aire finie $[0;1]\times[0;1]$.


		\paragraph{Propriétés}
	    
			\begin{propShort}
				Le flocon de Von Koch a un périmètre infinie.
			\end{propShort}
			\begin{preuve}
			% Longueur de $C^\infty$ :
			$$
				l_0 = 1 \quad \mapsto l_1 = \frac{4}{3} \quad \mapsto \ldots \mapsto l_k = \left(\frac{4}{3}\right) \underset{k\to\infty}{\to} + \infty
			$$
			\end{preuve}


	    	\begin{propShort}
	    		Le flocon de Von Koch a une aire finie.
	    	\end{propShort}
	    	\begin{preuve}
				% Aire de $C^\infty$ :
				\begin{align*}
					A_0 &\in \R	\\
					A_1 &= A_0 + 3\times\frac{A_0}{9} = A_0 + \frac{A_0}{3} 	\\
					A_2 &= A_1 + 12\times\frac{A_0}{9^2} = A_0 + \frac{A_0}{3} + \frac{2}{3^2}A_0	\\
				\end{align*}
				Par récurrence :
				\begin{align*}
				 A_k &= A_0 + \sum_{i=0}^{k}\frac{2^i}{3^{i+1}}A_0 = A_0 + \frac{A_0}{3} \times \frac{1-\left(\frac{2}{3}\right)^k}{\frac{1}{3}} = A_0 + A_0 \times \left(1-\left(\frac{2}{3}\right)^k\right) \\
				 A_\infty &= \lim_{k\to\infty} A_k = 2A_0
				\end{align*}
	    	\end{preuve}

			Le flocon de Von Koch définit donc une surface finie contenue dans une courbe de longueur infinie.

	\subsection{Triangle de Sierpinski}
		
		\paragraph{Construction}
			
			On construit cette fractale de la manière suivante :

			\begin{algorithm}[H]
			\DontPrintSemicolon
			\caption{Triangle de Sierpinski}
				\KwData{
					Triangle équilatéral $E_0$ de coté $1$\;
					$n$ la profondeur désirée\;
				}
				\KwResult{$E_n$ le triangle de Sierpinski de profondeur $n$}
				\For{chaque triangle jusqu'à la profondeur $n$}
				{
					Partager le triangle en 4 triangles égaux\;
					Retirer le triangle central\;
				}
			\end{algorithm}

			On a l'implémentation sous Scilab suivante :
			\begin{listing}[H]
				\scicode{\tdB 4-Triangle_Sierpinski.sce}
				\caption{Triangle de Sierpinski}
				\label{code-2-triangleSierpinskiRecursif}
			\end{listing}

			On obtient le résultat suivant :
			\begin{figure}[H]
				\centering
				\includegraphics[width=.6\linewidth]{\tdB\img 4-Triangle_Sierpinski_7.eps}
				% \includegraphics[width=.4\linewidth]{\tdB\img 4-Triangle_Sierpinski_9.png}
				\caption{Triangle de Sierpinski de profondeur $n=7$}
				\label{img-2-triangleSierpinksi}
			\end{figure}

			On peut aussi construire l'ISF suivant composé d'une homothétie de rapport $r=\frac{1}{2}$ et quelques translations :
			\begin{align*}
				% \label{eq-2-isfTriSierpinski}
				T_1(x,y) &= \frac{1}{2} \binom{x}{y}							\\
				T_2(x,y) &= \frac{1}{2} \binom{x}{y} + \binom{\frac{1}{2}}{0}	\\
				T_3(x,y) &= \frac{1}{2} \binom{x}{y} + \binom{\frac{1}{4}}{\frac{\sqrt{3}}{4}}
			\end{align*}

			On implémente cet ISF de type aléatoire sous Scilab :
			\begin{listing}[H]
				\scicode{\tdB 4.2-Triangle_Sierpinski_Iteratif.sce}
				\caption{Triangle de Sierpinski Itératif}
				\label{code-2-triangleSierpinskiIteratif}
			\end{listing}

			Le choix des transformations est équiprobable et la construction du triangle est itérative.
			Le résultat est le suivant :
			\begin{figure}[H]
				\centering
				\includegraphics[width=.6\linewidth]{\tdB\img 4-Triangle_Sierpinski_Ite.png}
				\caption{Triangle de Sierpinski itératif avec $n=100000$ points}
				\label{img-2-triangleSierpinksiIte}
			\end{figure}

			On observe une légère différence de rendu entre les deux figures \ref{img-2-triangleSierpinksi} et \ref{img-2-triangleSierpinksiIte} mais les deux décrivent bien la même fractale.

		\paragraph{Propriétés}
			On note $E_\infty$ le triangle de Sierpinski.

			\begin{propShort}
				L'aire du triangle de Sierpinski est nulle.
			\end{propShort}
			\begin{preuve}
				On suppose que $\Aire E_0 = 1$.
				On a :
				\begin{align*}
					\Aire E_1 &= 1 - \frac{1}{4}		\\
					\Aire E_2 &= 1 - \frac{1}{4} - 3\times\left(\frac{1}{4}\right)^2 \\
				\end{align*}
				Par récurrence, on a :
				\begin{align*}
					\Aire E_\infty
						&= 1 - \sum_{k=0}^\infty 3^k \times \frac{1}{4^{k+1}} \\
						&= 1 - \frac{1}{4} \sum_{k=0}^\infty \left(\frac{3}{4}\right)^k		\\
						&= 1 - \frac{1}{4} \times \left( \frac{1- \left( \frac{3}{4} \right)^\infty}{1 - \frac{3}{4}} \right)	\\
						&= 	0
				\end{align*}
			\end{preuve}

			% TODO : Binaire

		\paragraph{Dimension}
			% TODO : ajouter les images des triangles/carrés

			La dimension $d$ du triangle de Sierpinski $E^\infty$ est :
			$$
				\left.\begin{aligned}
					r &= \frac{1}{2}	\quad	\\
					N &= 3	\quad
					\end{aligned}
				\right\}
				\qquad 2=3^d \implies d= \frac{\ln{3}}{\ln{2}}
			$$


	\subsection{Tapis de Sierpinski}

		\paragraph{Construction}
		
			Le tapis ou napperon de Sierpinski part du même principe que le triangle mais avec un carré. On a alors un rapport de $r=1/3$ et 9 sous-carrés dont 1 seul enlevés à chaque itérations.

			\begin{algorithm}[H]
			\DontPrintSemicolon
			\caption{Tapis de Sierpinski}
				\KwData{
					Carré $E_0$ de coté $1$\;
					$n$ la profondeur désirée\;
				}
				\KwResult{$E_n$ le tapis de Sierpinski de profondeur $n$}
				\For{chaque carré jusqu'à la profondeur $n$}
				{
					Partager le triangle en 9 carrés égaux\;
					Retirer le carré central\;
				}
			\end{algorithm}

			On a l'ISF suivant décrivant une homothétie de rapport $r=\frac{1}{3}$ et quelques translations :
			\begin{equation}
				\begin{array}{lll}
					\label{eq-2-isfTapisSierpinski}
					\text{Translations :}
					& d = \begin{pmatrix} \frac{1}{3}	\\ 0				\end{pmatrix}
					& h = \begin{pmatrix} 0				\\ \frac{1}{3}		\end{pmatrix}
					\\
					\\
					\text{Transformations :}
					& T_3(M) = rM + 2d
					& T_6(M) = rM + d + 2h
					\\
					T_1(M) = rM
					& T_4(M) = rM + 2d + h
					& T_7(M) = rM + 2h
					\\
					T_2(M) = rM + d
					& T_5(M) = rM + 2d + 2h
					& T_8(M) = rM + h
				\end{array}	
			\end{equation}

			On peut donc implémenter cet ISF de manière déterministe et aléatoire avec Scilab :

			\begin{listing}[H]
				\scicode{\tdB 5-Tapis_Sierpinski.sce}
				\caption{Tapis de Sierpinski}
				\label{code-2-tapisSierpinskiRecursif}
			\end{listing}
			\begin{listing}[H]
				\scicode{\tdB 5-Tapis_Sierpinski_Iteratif.sce}
				\caption{Tapis de Sierpinski itératif}
				\label{code-2-tapisSierpinskiIteratif}
			\end{listing}

			On obtient les résultats suivants :
			\begin{figure}[H]		
				\centering
				\begin{subfigure}{.45\textwidth}
					\centering
					\includegraphics[width=.95\linewidth, trim=1cm 1cm 1cm 1cm, clip]{\tdB\img 5-Tapis_Sierpinski_5.eps}
					\caption{Version récursive de profondeur $n=5$}
					\label{img-2-tapisSierpinskiRecursif}
				\end{subfigure}
				\begin{subfigure}{.45\textwidth}
					\centering
					\includegraphics[width=\linewidth, trim=1.05cm 1.05cm 1.05cm 1.05cm, clip]{\tdB\img 5-Tapis_Sierpinski_Ite.png}
					\caption{Version itérative avec $n=100000$ points}
					\label{img-2-tapisSierpinskiIteratif}
				\end{subfigure}
				\caption{Tapis de Sierpinski}
				\label{img-2-tapisSierpinski}
			\end{figure}


		\paragraph{Dimension}
			La dimension $d$ du tapis de Sierpinski est :
			$$
				\left.\begin{aligned}
					r &= \frac{1}{3}	\quad	\\
					N &= 8	\quad
					\end{aligned}
				\right\}
				\qquad 2=3^d \implies d= \frac{\ln{8}}{\ln{3}}
			$$

			% TODO : Surface nulle

	% \subsection{Pyramide de Sierpinski}
	% \subsection{Dragon de Levy}


	\subsection{Éponge de Menger}

		L'éponge de Menger correspond à l'application du tapis de Sierpinski en 3D. On a cette fois 20 sous-cubes 3 fois plus petits que le précédent.

		% Le rapport à l'itération $k$

		J'ai d'abord crée la fonction \code{plotCube} qui dessine un cube de coté \code{r} et d'origine \code{origin}.
		\begin{listing}[H]
			\scicode{\tdB Plot_Cube.sce}
			\caption{Fonction pour dessiner un cube}
			\label{code-2-plotCube}
		\end{listing}
	


		Ensuite j'ai fait le programme suivant :
		\begin{listing}[H]
			\scicode{\tdB 6.1-Eponge_Menger_Light.sce}
			\caption{Éponge de Menger}
			\label{code-2-epongeMenger}
		\end{listing}

		Le code \ref{code-2-epongeMenger} ne peut que très légèrement être modifié. Si on veut améliorer ses performances, il faut donc améliorer la fonction \code{plotCube}.
		Cependant le code \ref{code-2-plotCube} n'est pas très optimisé, il doit y avoir d'autres façons plus efficaces pour dessiner un cube simplement. Pour une profondeur de $n=2$ le programme s'arrête en moins d'une minute mais pour $n=3$ la mémoire est très vite saturée.

		Néanmoins j'ai pu obtenir le résultat suivant :
		\begin{figure}[H]
			\centering
			\includegraphics[width=.6\linewidth, trim=2.5cm 2.5cm 2.5cm 2.5cm, clip]{\tdB\img 6-Eponge_Menger.png}
			\caption{Éponge de Menger de profondeur $n=2$}
			\label{img-2-epongeMenger}
		\end{figure}


	\subsection{Fougère de Barnsley}

		La fougère de Barnsley est l'une des premières fractales créées itérativement avec un ISF aléatoire.
		On a le programme Scilab suivant :
		\begin{listing}[H]
			\scicode{\tdB 7-Fougere.sce}
			\caption{Fougère de Barnsley}
			\label{code-2-fougere}
		\end{listing}

		Ici les 4 transformations ne sont pas équiprobables.
		On obtient :
		\begin{figure}[H]
			\centering
			\includegraphics[width=.4\linewidth, trim=3cm 3cm 3cm 3cm, clip]{\tdB\img 7-Fougere.png}
			\caption{Fougère de Barnsley avec $n=100000$ points}
			\label{img-2-fougere}
		\end{figure}	

	% \subsection{Arbre binaire symétrique}



% \section{Applications des fractales}

% 	Structure des poumons
% 	Structures des plantes
% 	Branches d'arbre
% 	Finance : prévision de krachs boursiers avec la théorie multifractale
% 	Murs antibruits

% 	\subsection{Dans le corps humain}
% 	\subsection{Dans la nature}
% 	\subsection{Dans l'imagerie de synthès}
% 	\subsection{Dans la technologie}


%

% TODO IDEAS :
% Attracteur
% Image de synthèse
% Exemples imagés
% ISF et Algo pour les 
	\chapter{Equations différentielles}
\label{ch-3}

% \section{Introduction}

	Dans ce chapitre, nous chercherons à rédoudre des équations différentielles que
	\begin{equation}
		\label{eq-3-eqDiff}
		\begin{cases}
			y'(t) = f(t, y(t)) \\
			y(t_0) = y_0
		\end{cases}
	\end{equation}
	avec $t \in I = [t_0; t_0 + T] ~ (T>0)$ et $f(t,y):~I\times\R^n \to \R^n$.
	\eqref{eq-3-eqDiff} est un problème de Cauchy.

	\bigskip

	Dans le cas des équations différentielles d'ordre $n>1$, nous allons ramener le problème a un équation différentielle d'ordre 1 dans $\R^n$. On a le système différentiel d'ordre $n$:
	\begin{equation}
		\label{eq-3-systDiffRn}
		\begin{cases}
			y(0) = a_0					\\
			y'(0) = a_1					\\
			\quad\vdots 				\\
			y^{(n-1)}(0) = a_{n-1}		\\
			y^{(n)} = f(t, y, y',...,y^{(n-1)})
		\end{cases}
	\end{equation}

	On pose alors $Y\in\R^n$ tel que : \quad
	$
		\begin{cases}
			Y' = F(t,Y)			\\
			Y(0)
		\end{cases}
	$ \quad avec :
	$$
		Y = \begin{pmatrix}
			y_1 = y(t) 	\\
			y_2 = y'_1 = y'		\\
			\vdots 	\\
			y_n = y_{n-1}' = y^{(n-1)}
		\end{pmatrix}
		\quad, \qquad
		Y(0) = \begin{pmatrix}
			a_0	\\
			a_1	\\
			\vdots 	\\
			a_{n-1}
		\end{pmatrix}
		\quad \text{et} \qquad
		F(t,Y) = \begin{pmatrix}
			y_1 	\\
			y_2		\\
			\vdots 	\\
			y_{n-1}	\\
			f(t, y1, y2, ..., y_{n-1})
		\end{pmatrix}
		$$

	\bigskip

	\begin{ex}[Vectorisation de systèmes différentielles]
		\begin{equation}
			\label{eq-3-ex3ordre}
			\begin{cases}
				y^{(3)} + ay' + cy = 0	\\
				y(0) = d_0		\\
				y'(0) = d_1		\\
				y''(0) = d_2
			\end{cases}
		\end{equation}

		On vectorise l'équation :
		$$
			Y = \begin{pmatrix}
				y_1 = y			\\
				y_2 = y'_1 = y'	\\
				y_3 = y'_2 = y''
			\end{pmatrix}
			\iff
			\begin{cases}
				y'_1 = y_2	\\
				y'_2 = y_3	\\
				y'_3 = -ay_2 - cy_1
			\end{cases}
			\iff
			Y' = f(t,Y) = \begin{pmatrix}
				0	&	1	&	0	\\	
				0	&	0	&	1	\\	
				-c	&	-a	&	0	
			\end{pmatrix}
			\begin{pmatrix}
				y_1	\\	
				y_2	\\	
				y_3		
			\end{pmatrix}
			$$
	\end{ex}

	Que ce passe-t-il si l'on n'est pas capable de résoudre analytiquement une équation différentielle ?

	\begin{ex}
		\begin{equation}
			\label{eq-3-et2}
			\begin{cases}
				y'(t) = e^{-t^2}	\\
				y(0) = 1
			\end{cases}
		\end{equation}
		
		\begin{align*}
			\frac{dy}{dt} = e^{-t^2} 	&\implies \frac{dy}{y} = e^{-t^2}				\\	
										&\implies \int \frac{dy}{y} = \int e^{-t^2} dt	\\
										&\implies \ln\abs{y} = \int e^{-t^2} dt
		\end{align*}

		Arrivé ici, on est bloqué car on ne connaît pas de primitive de $e^{-t^2}$ donc l'équation différentielle \eqref{eq-3-et2} n'admet pas de solution analytique. On a donc recours à un schéma numérique.
	\end{ex}


\section{Schémas numériques}

		Les algorithmes que nous présenterons ici ont pour but d'approcher la solution d'une équation numérique.

		On discrétise l'intervalle $I = [t_0; t_0 +T]$ en $\sigma = \{ t_0, t_1, ..., t_N \}$ avec $t_N = t_0 +T$. Pour simplifier, on supposera des subdivisions uniformes et donc le pas est : $h = h_i = \abs{t_{i+1}-t_i} ~~\forall i \in [0; N-1]$.

		On approche $y(t_i)$ par $z_i$ où :
		\begin{equation}
			\label{eq-3-schema}
			\begin{cases}
				z_{i+1} = z_i + h\Phi(t_i, z_i, h_i) \\
				z_0 = y_0
			\end{cases}
		\end{equation}
		Avec $\Phi$ la fonction d'approximation spécifique à chaque schéma.


	\subsection{Caractéristiques d'un schéma numérique}

		\subsubsection{Stabilité}
			Soient les suites $\suite[i]{u}$ et $\suite[i]{v}$ telles que :
			\begin{equation}
				\label{eq-3-uv}
				\begin{cases}
					u_{i+1} = u_i + h\Phi(t_i,u_i,h)	\\
					u_0 \text{ donné}
				\end{cases}
				\qquad
				\begin{cases}
					v_{i+1} = u_{i+1} + \epsilon_i	\\
					v_0 = u_0
				\end{cases}
			\end{equation}

			$(v_i)$ correspond au même schéma que $(u_i)$ mais avec une erreur $\epsilon_i \in \R$.

			\begin{definition}[Stabilité]
				Le schéma est dit stable si
				\begin{equation}
					\label{eq-3-stable}
					\max_{i\in  I} \abs{u_i - v_i} \leq C \left( \abs{u_0 - v_0} + \sum_{i=1}^N \abs{\epsilon_i} \right)
				\end{equation}
				où $C$ est une constante ne dépendant pas de $u_i$ et $v_i$.
			\end{definition}

			\begin{theoreme}
				Le schéma est stable si et seulement si $\exists K \tq$ :
				\begin{equation}
					\label{eq-3-stableLip}
					\norm{\Phi(t,y,h) - \Phi(t,z,h)} \leq K \norm{y-z}
				\end{equation}
				c'est-à-dire que $\Phi$ est $K-$lipschitzienne par rapport à sa 2\ieme variable.
			\end{theoreme}

			Le fait qu'un schéma soit stable signifie que la propagation d'erreur sur la donnée initiale n'influe pas le résultat final. Si $\Phi$ n'est pas lipschitzienne, l'unicité de la solution n'est pas garantie.

		\subsubsection{Consistance}

			\begin{definition}[Consistance]
				Un schéma est dit consistant si
				\begin{align}
					\label{eq-3-consistance}
					\epsilon(y) = \sum \norm{y_{i+1} - y_i - \phi{t_i, y_i, h}} \limto{h}{0} 0	\\
					\Phi(t,y,0) = f(t,y)
				\end{align}
			\end{definition}

			\begin{theoreme}
				Pour montrer que le schéma est consistant, il suffit de montrer qu'il existe une constante $C>0$ telle que :
				\begin{equation}
					\label{eq-3-consistanceSuff}
					\norm{\Phi(t,y,0) - \Phi(t,z,0)} \leq C \norm{y-z}
				\end{equation}
			\end{theoreme}
			% TODO : WTFFFFFFFFFFFFFFFFFFFf

		\subsubsection{Ordre de convergence}

			\begin{definition}[Ordre de convergence]
				Un schéma est d'ordre $p$ si $\exists C > 0 \tq $:
				\begin{equation}
					\label{eq-3-ordre}
					\epsilon(y) = \max \abs{y_i - z_i} \leq Ch^p	\\
				\end{equation}
				% TODO : Cela correspond au nombre de décimales significatives gagnées
			\end{definition}


	\subsection{Schéma d'Euler}

		On a $y' = f(t, y)$. On connaît $y_0$, comment calculer $y_1 = y(t_1)$ ? On fait un développement de Taylor :
		$$
			y_{i+1} = y(t_{i+1}) = y(t_i) + hy'(t_i) + \frac{h^2}{2}y''(\epsilon)
		$$
		On se débarasse du reste qui tend vers 0 quand $h$ tend vers 0 :
		$$
			y_{i+1} = y(t_{i+1}) \sim y_i + hy'(t_i) = y_i + hf(t_i, y_i) = z_{i+1}
		$$

		On approche donc $y_i$ par $z_i$, on obtient le schéma d'Euler :
		\begin{equation}
			\label{eq-3-euler}
			\begin{cases}
				z_{i+1} = z_i + hf(t_i, z_i)		\\
				z_0 = y_0
			\end{cases}
		\end{equation}
		Et donc
		\begin{equation}
			\label{eq-3-phiEuler}
			\Phi(t_i,z_i,h) = f(t_i,z_i)
		\end{equation}
		~
		\bigskip

		On peut aussi définir ce schéma sans développement de Taylor :
		\begin{equation}
			\label{eq-3-dev}
			y(t_{i+1}) 	= y(t_i) + \int_{t_i}^{t_{i+1}} y'(t)dt
						= y(t_i) + \int_{t_i}^{t_{i+1}} f(t,y)dt			
		\end{equation}

		Or $\int_{t_i}^{t_{i+1}} f(t,y)dt$ est inconnu, on l'approxime donc par l'aire du rectangle gauche :
		\begin{equation}
			y(t_{i+1}) \sim y(t_i) + h\times f(t_i, y_i)
		\end{equation}
		D'où le schéma d'Euler explicite \eqref{eq-3-euler} : $z_{i+1} = z_i + hf(t_i,z_i)$.

		Si on approche par l'aire du rectangle droit, on a le schéma d'Euler implicite (moins utilisé) : $z_{i+1} = z_i + hf(t_{i+1}, z_{i+1})$.

		\bigskip

		\begin{propShort}
			Le schéma d'Euler est stable et consistant.
		\end{propShort}
		\begin{preuve}
			$ \eqref{eq-3-phiEuler} \implies \norm{f(t,y) -f(t,z)} \leq K\norm{y-z} $
			si $f$ est lipschitzienne, alors le schéma est stable
			De plus par définition \eqref{eq-3-phiEuler} le schéma est consistant.
		\end{preuve}

		\begin{propShort}
			Le schéma d'Euler est d'ordre $1$.
		\end{propShort}
		\begin{preuve}
			\begin{align*}
				y_{i+1} 			&= y(t_i) + hy'_i + \frac{h^2}{2}y''(\epsilon_i)	\\
				\text {et } z_{i+1} 	&= z_i + hf(t_i, z_i)
			\end{align*}
			$$
				\text{D'où } \norm{y_{i+1} - z_{i+1}} = (y_i -z_i) + h \left( f(t_i,y_i) -f(t_i, z_i) \right) + \frac{h^2}{2}y''(\epsilon_i)
			$$
			% Uniforme ?
			On suppose $f$ $L$-lipschitzienne uniforme par rapport à sa seconde variable.
			De plus, supposons : $\exists M > 0$ tel que $\norm{y''(\epsilon_i)}\leq M ~~\forall\epsilon\in [t_0, t_0 + T] $.

			On note $e_i = y_i - z_i$, on a alors :
			\begin{align*}
				\norm{e_{i+1}}	&\leq \norm{e_i} + hK\norm{e_i} + \frac{h^2}{2}M	\\
								&\leq \underbrace{(1 + hL)}_{C} \norm{e_i} + \underbrace{\frac{h^2}{2}M}_{D}			\\
								% \text{on pose $1+hL = C$ et } \frac{h^2}{2}M = D		\\
								&\leq C \norm{e_i} + D
			\end{align*}
			$\suite[i]{e}$ est une suite arithmético-géométrique, on a alors :
			\begin{align*}
				\norm{e_{i+1}}	&\leq C^{i+1}e_0 + D\times \sum_{k=1}^i C^i			\\
								&\leq C^{i+1}e_0 + D\times \frac{C^{i+1} - 1}{C-1}	\\
			\end{align*}
			Or $z_0 = y_0 \implies e_0 = 0$. De plus :		
			\begin{align*}
				D\times \frac{C^{i+1} - 1}{C-1}	&= \frac{h^2}{2} M \frac{(1+hL)^{i+1} -1}{hL}	\\
												&= \frac{hM}{2L} \left( (1+hL)^{i+1} -1 \right)
			\end{align*}
			On utilise $(1+x)^k \leq e^{kx} ~~\forall k,x>0$. On a finalement :
			\begin{align*}
				\norm{e_{i+1}}	&\leq \frac{hM}{2L}(e^{iKH} -1) = \frac{hM}{2L}(e^{\overbrace{t_i-t_0}^{ih}} -1)	\\
								&\leq h \underbrace{\left( \frac{M}{2L}(e^T-1) \right)}_{K \in \R}				\\
								&\leq Kh
			\end{align*}
		\end{preuve}

	\subsection{Schéma du point-milieu}

		On approxime \eqref{eq-3-dev} par l'aire du rectangle du point du milieu :
		\begin{align*}
			\eqref{eq-3-dev} \iff
			y(t_{i+1}) 	&\sim y(t_i) + h\times f(t_i +\frac{h}{2}, y(t_i +\frac{h}{2}))	\\
						&\sim y(t_i) + h\times f(t_i +\frac{h}{2}, y_i + \frac{h}{2}f(t_i,y_i))
		\end{align*}

		D'où le schéma du point milieu :
		\begin{equation}
			\label{eq-3-pointMilieu}
			\begin{cases}
				z_{i+1} = z_i + h\times f(t_i +\frac{h}{2}, z_i + \frac{h}{2}f(t_i,z_i))		\\
				z_0 = y_0
			\end{cases}
		\end{equation}
		Et donc $\Phi(t_i,z_i,h) = f(t_i +\frac{h}{2}, y_i + \frac{h}{2}f(t_i,y_i))$.

		\bigskip

		% TODO ordre 1 ou 2 ?
		\begin{propShort}
			Le schéma du point milieu est d'ordre $1$.
		\end{propShort}	
		\begin{preuve}
			On part des suites $\suite[i]{u}$ et $\suite[i]{u}$ définies en \eqref{eq-3-uv}
			\begin{align*}
				u_{i+1} - v_{i+1} 			&= [u_i + h\Phi(t_i,u_i,h)] - [v_i + h\Phi(t_i,v_i,h)]		\\
				\norm{u_{i+1} - v_{i+1}} 	&\leq [u_i + h\Phi(t_i,u_i,h)] - [v_i + h\Phi(t_i,v_i,h)]	\quad\text{(inégalité triangulaire)}	\\
											&\leq \norm{u_i - v_i} + h\norm{\Phi(t_i,u_i,h) - \Phi(t_i,v_i,h)} + \norm{\epsilon_i}	\\
											% Dafuq ???
											&\leq \norm{u_i - v_i}(1+Ch) + \norm{\epsilon_i}
			\end{align*}
			On utilise le lemme de Granwall :
			Si $\abs{u_{i+1} - v_{i+1}} \leq C\abs{u_i - v_i} + \abs{\epsilon_i}$ alors $\exists K \tq $ :
			\begin{equation}
				\label{eq-2-granwall}
				\abs{u_{i+1} - v_{i+1}} \leq K \left( \abs{u_0 - v_0} + \sum_{k=1}^i \abs{\epsilon_i} \right)
			\end{equation}

			\begin{preuve}
				Par récurrence :
				Au rang $i=0$, c'est trivialement vrai.	\\
				On suppose $\abs{u_{i+1} - v_{i+1}} \leq C \left( \abs{u_0 - v_0} + \sum_{k=1}^i \abs{\epsilon_i} \right)$ vrai pour un certain rang $i$.
				Par hypothèse, on a :
				\begin{align*}
					\abs{u_{i+1} - v_{i+1}} &\leq C\abs{u_i - v_i} + \abs{\epsilon_i}		\\
											&\leq C\left( C(\abs{u_0 - v_0} + \sum_{k=1}^{i-1} \abs{\epsilon_i}) \right) + \abs{\epsilon_i}	\quad\text{(hypothèse de récurrence)}	\\
											&\leq \underbrace{\max(C^2, C\abs{\epsilon_i})}_{K} \left( \abs{u_0 - v_0} + \sum_{k=1}^i \abs{\epsilon_i} \right)	
				\end{align*}
			\end{preuve}
		\end{preuve}

		% TODO Preuve
		\begin{propShort}
			Le schéma du point milieu est stable et consistant.
		\end{propShort}


	\subsection{Schéma d'Euler-Cauchy}

		On approxime \eqref{eq-3-dev} par l'aire du trapèze :
		\begin{align*}
			\eqref{eq-3-dev} \iff
			y(t_{i+1}) 	&\sim y(t_i) + \frac{h}{2}\times \left( f(t_i, y_i) + f(t_{i+1}, y_{i+1}) \right)	\\
						&\sim y(t_i) + \frac{h}{2}\times \left( f(t_i, y_i) + f(t_{i+1}, y_i + hf(t_i,y_i)) \right)
		\end{align*}

		D'où le schéma d'Euler-Cauchy :
		\begin{equation}
			\label{eq-3-eulerCauchy}
			\begin{cases}
				z_{i+1} = z_i + \frac{h}{2}\times \left( f(t_i, z_i) + f(t_{i+1}, z_i + hf(t_i,z_i)) \right)	\\
				z_0 = y_0
			\end{cases}
		\end{equation}
		Et donc $\Phi(t_i,z_i,h) = \frac{1}{2}\times \left( f(t_i, z_i) + f(t_{i+1}, z_i + hf(t_i,z_i)) \right)$.

		\bigskip

		% TODO : Preuves
		\begin{propShort}
			Le schéma d'Euler-Cauchy est stable et consistant.
		\end{propShort}

		\begin{propShort}
			Le schéma d'Euler-Cauchy est d'ordre $2$.
		\end{propShort}


	\subsection{Schéma de Runge-Kutta}

		Enfin nous avons le schéma de Runge-Kutta :
		\begin{equation}
			\label{eq-3-rungeKutta}
			\begin{cases}
				k_1 = f(t_i, z_i)	\\
				k_2 = f(t_i + \frac{h}{2}, 	z_i + \frac{h}{2}k_1)	\\
				k_3 = f(t_i + \frac{h}{2}, 	z_i + \frac{h}{2}k_2)	\\
				k_4 = f(t_i + h,			z_i + hk_3)				\\
				z_{i+1} = z_i + \frac{1}{6}(k_1 + 2k_2 + 2k_3 + k_4)
			\end{cases}
		\end{equation}

		% TODO Preuves ?
		\begin{propShort}
			Le schéma de Runge-Kutta est stable et consistant.
		\end{propShort}

		\begin{propShort}
			Le schéma de Runge-Kutta est d'ordre $4$.
		\end{propShort}

\section{Applications}
	\subsection{Comparaison des schémas}
		
		Nous implémentons les schémas sous Scilab de la manière suivante :

		\begin{listing}[H]
			\scicode{\tdC 0-Tous_Schemas.sce}
			\caption{Schémas de résolution d'équations différentielles}
			\label{code-3-schemasED}
		\end{listing}

		Les paramètres de ces schémas sont les suivants :
		\begin{itemize}
			\item \code{y0} $\in\M_{1k}$ vecteur colonne ou réel correspondant à la situation initiale ($k$ est l'ordre de l'équation différentielle)
			\item \code{t} $\in\M_{n1}$ discrétisation de l'intervalle $[t_0; t_0+T]$
			\item \code{f(t,x)} correspond à l'équation différentielle à résoudre retourne $f(t,x) \in\M_{1k}$
		\end{itemize}
		En retour, on récupère la matrice \code{y} $\in\M_{kn}$ où chaque ligne correspond à une itération correspondant à la fonction $y(t)$ approchée à l'instant $t_i$.

		On va comparer tous ces schémas sur l'équation différentielle suivante :
		\begin{equation}
			\begin{cases}
				y' = -t\times y(t) + t	& t\in[0;4]	\\
				y(0) = y_0 = 0
			\end{cases}
		\end{equation}
		dont la solution analytique est $y(t) = 1 - e^{-\frac{t^2}{2}}$.

		On fait les comparaisons pour plusieurs discrétisations avec $n$ le nombre de subdivisions de l'intervalle $[0;4]$. Pour chaque valeur de $n$ on calcule l'erreur logarithmique absolue maximale de chaque schéma par rapport à la solution analytique.
		Ensuite on effectue une régression linéaire des erreurs afin de récupérer l'ordre des schémas.

		On obtient les résultats suivants :
		\begin{figure}[H]
			\centering
			\includegraphics[width=\linewidth]{\tdC\img 1-Comparaison.eps}
			\caption{Comparaison des schémas}
			\label{img-3-comparaison}
		\end{figure}

		Sur le graphique de la fonction, on ne distingue pas les approximations de chaque schémas car elles sont confondues. Concernant les ordres des méthodes on obtient bien les mêmes que ceux prévues précédemment.

		% TODO ode utilise la méthode d'Adams 

	\subsection{Simulation d'un pendule}

		On modélise ici un pendule de masse $M$ accroché à une tige de longueur $L$ de masse négligeable devant $M$. On note $\theta(t)$ l'angle formé par la tige et l'axe vertical passant par la fixation du pendule.
		% TODO schéma
		On a l'équation différentielle :
		\begin{equation}
			\label{eq-3-pendule}
			\begin{cases}
				\theta''(t) = -\frac{g}{L} \times\sin(\theta(t))		\\
				\theta'(0) = v_0		\\
				\theta(0) = \theta_0
			\end{cases}
		\end{equation}

		Comme l'équation est non-linéaire à cause du $\sin(\theta)$, on ne peut pas trouver de solution analytique. En revanche, on peut approcher $\sin(\theta)$ par $\theta$ dans l'hypothèse où les angles sont petits (inférieurs à 5°). Cette approximation permet alors d'avoir l'équation différentielle linéaire suivante :
		\begin{equation}
			\label{eq-3-penduleApprox}
			\begin{cases}
				\phi''(t) = -\frac{g}{L} \times\phi(t)		\\
				\phi'(0) = v_0		\\
				\phi(0) = \theta_0
			\end{cases}
		\end{equation}
		où $\phi(t)$ correspond à $\theta(t)$ avec l'approximation.
		La solution analytique est alors :
		\begin{equation}
			\label{eq-3-soluceApprox}
			\phi(t) = \theta_0 \cos \left( t\sqrt{\frac{g}{L}} \right)
		\end{equation}

		On va alors comparer les deux modèles \eqref{eq-3-pendule} et \eqref{eq-3-penduleApprox} avec différents angles $\theta_0$.
		Pour l'implémentation sous Scilab on pose : $y = \binom{\theta}{\theta'}$. On a :
		\begin{listing}[H]
			\scicode{\tdC 2-Pendules.sce}
			\caption{Modélisation de pendules}
			\label{code-3-pendules}
		\end{listing}

		Les différentes conditions intiales sont répertoriées dans la matrice \code{thetas}. Pour chaque expérience, on résout le problème original \eqref{eq-3-pendule} et celui après approximation \eqref{eq-3-penduleApprox} avec la méthode de Runge-Kutta. On obtient les résultats suivant

		\begin{figure}[H]
			\centering
			\includegraphics[width=\linewidth]{\tdC\img 2a-Pendules.eps}
			\caption{Pendules des différentes expériences}
			\label{img-3-pendules}
		\end{figure}
		\begin{figure}[H]
			\centering
			\includegraphics[width=\linewidth]{\tdC\img 2a-Graphes_Sin.eps}
			\caption{Comparaison des différentes expériences}
			\label{img-3-compPendules}
		\end{figure}

		On observe que pour les petits angles $\theta_0 = \frac{\pi}{64} = 2.8125$° et même $\frac{\pi}{16} = 11.25$° l'approxmation diffère très peu du problème original. En revanche pour des angles plus grand, on voit clairement une différence de période d'oscillation et donc un écart se creusant entre les deux solutions. Ainsi on a bien vérifié expérimentalement que l'hypothèse $sin(\theta) \sim \theta$ n'est valable que pour de petits angles. 


	\subsection{Gravitation}

		La mécanique céleste est une affaire d'équations différentielles.
		Ainsi nous pouvons simuler le mouvement de plusieurs corps céleste soumis à leur attractions gravitationnelles mutuelles.
		Dans le cas de deux corps, le problème peut être résolu analytiquement mais pour plus de deux corps cela n'est pas possible. On pourrait tenter de contourner ce problème en le ramenant à plusieurs sous-problèmes à deux corps mais la solution ainsi obtenue ne correspond pas aux trajectoires observées. Il y a entre chaque corps une influence suffisamment importante pour modifier les trajectoires à long terme.
		Dans le cas de plusieurs corps, nous devons recourir aux mêmes schémas numériques qui nous ont permis de résoudre le problème du pendule \eqref{eq-3-pendule}.

		Nous allons ici modéliser le mouvement de deux corps d'abord, puis de trois.
		On considère deux corps sphérique $C_1$ et $C_2$ de masses $m_1$ et $m_2$ et de centres :
		\begin{equation}
			\label{eq-3-centreCorps}
			u_1 = \binom{x_1}{y_1} \qquad \text{et} \qquad u_2 = \binom{x_2}{y_2}
		\end{equation}

		La force gravitationnelle exercée par le corps 2 sur le corps 1 est :
		\begin{equation}
			F_{2\to1} = G \times \frac{m_1 m_2}{\norm{u_2 - u_1}^3}(u_2 - u_1)			
		\end{equation}
		avec $G = 6.67 \times 10^{-11}$ la constante de gravitation universelle. La force exercée par le $C_1$ sur $C_2$ est égale à son opposée : $F_{1\to2} = - F_{2\to1}$.

		Avec la seconde loi de Newton, nous obtenons le système dynamique suivant :
		\begin{equation}
			\label{eq-3-gravit2}
			\begin{cases}
				u_1'' = +G m_2 \frac{u_2-u_1}{\norm{u_2-u_1}^3}		\\
				u_2'' = +G m_1 \frac{u_2-u_1}{\norm{u_2-u_1}^3}		\\
				% u_1(0) 	\qquad
				% u_1'(0)	\qquad
				% u_2(0)	\qquad
				% u_2'(0)
			\end{cases}
		\end{equation}

		On considère ce problème avec la Terre pour $C_1$ et la Lune pour $C_2$. La distance moyenne Terre-Lune est de $d_{TL} = 3.84402\times10^8$ m et la période de rotation d'environ $T = 27,55$ jours. Nous centrons le systèmes sur la Terre, nous avons les conditions intiales suivantes :
		\begin{equation}
			\label{eq-3-condIni2}
			C_1
			\begin{cases}
				m_1 = 5.975\times10^{24}			\\
				u_1(0) = (0 ; 0)					\\
				u_1'(0) = (0; 0)				
			\end{cases}			
			\qquad\qquad
			C_2
			\begin{cases}			
				m_2 = 7.35\times10^{22}				\\
				u_2(0) = (d_{TL}; 0)				\\
				u_2'(0) = (0; \frac{2\pi}{T}d_{TL}) 
			\end{cases}			
		\end{equation}

		Pour pouvoir utiliser les schémas numériques, il faut vectoriser les équations sous forme du premier ordre. On a donc $v\in\R^8$ tel que le système d'équations différentielles à résoudre correspond à $v' = f(v)$.
		\begin{equation}
			\label{eq-3-gravit2Vect}
			v = \begin{pmatrix}
				u_1 	\\
				u_1' 	\\
				u_2 	\\
				u_2'
			\end{pmatrix}
		\end{equation}

		On a l'implémentation sous Scilab suivante :

		\begin{listing}[H]
			\scicode{\tdC 3-Gravitation.sce}
			\caption{Gravitation Terre-Lune}
			\label{code-3-gravitation2}
		\end{listing}


		On représente la trajectoire de la Lune (en bleu) et de la Terre (en bleu) par rapport au centre de gravité du système. On retrouve bien la trajectoire circulaire de la Lune autour de la Terre.

		\begin{figure}[H]
			\centering
			\includegraphics[width=.4\linewidth, trim=3cm 3cm 3cm 3cm, clip]{\tdC\img 3-Gravitation.eps}
			\caption{Gravitation Terre-Lune}
			\label{img-3-gravitation2}
		\end{figure}


		% TODO : point de Lagrange


	\subsection{Attracteur de Lorenz}

		% L'attracteur de Lorenz illustre l'effet papillon : un léger chang
		% Cond Initiales
		% Chaos

		L'attracteur de Lorenz permet de modéliser de manière simple le caractère chaotique des phénomènes météorologiques. Nous avons le système différentiel suivant :

		\begin{equation}
			\label{eq-3-systLorenz}
			\begin{cases}
				\dfdp{x(t)}{t} = \sigma \bigl( y(t) - x(t) \bigr) \\
				\dfdp{y(t)}{t} = \rho \, x(t) - y(t) - x(t) \, z(t)\\
				\dfdp{z(t)}{t} = x(t) \, y(t) - \beta \, z(t) 
			\end{cases}
		\end{equation}
		avec $\sigma = 10$, $\rho = 28$, $\beta = \frac{8}{3}$.

		Nous résolvons le système par le programme suivant :

		\begin{listing}[H]
			\scicode{\tdC 4-Lorenz.sce}
			\caption{Attracteur de Lorenz}
			\label{code-3-Lorenz}
		\end{listing}

		Affichons les valeurs de $x,y,z$ dans un repère 3D :

		\begin{figure}[H]
			\centering
			\includegraphics[width=.4\linewidth, trim=3cm 3cm 3cm 3cm, clip]{\tdC\img 4-Lorenz.png}
			\caption{Attracteur de Lorenz}
			\label{img-3-Lorenz}
		\end{figure}

		On observe que la solution forme une figure semblable à des ailes de papillon. Elle a un comportement qui peut sembler périodique mais qui est en réalité chaotique et imprévisible.

		De plus, le point initial est vite attiré vers l'attracteur. On peut prendre un point plus éloigné il convergera quand même dans l'attracteur.

		\begin{figure}[H]
			\centering
			\includegraphics[width=.4\linewidth, trim=3cm 3cm 3cm 3cm, clip]{\tdC\img 4-Lorenz_loin.png}
			\caption{Attracteur de Lorenz avec une condition initiale éloignée}
			\label{img-3-LorenzLoin}
		\end{figure}



	\subsection{Systèmes Proies-Prédateurs de Lotka}

		Le modèle proies-prédacteurs de Lotka est un système différentiel modélisant de manière simplifier un écosystème d'une population de proies et d'une autre de ses prédateurs. On a le système suivant :

		\begin{equation}
			\begin{cases}
				x' = ax - bxy \\
				y' = -cy + dxy
			\end{cases}
		\end{equation}
		Avec les différentes variables et paramètres : 
		\begin{itemize}
			\item $x$ la population de proies
			\item $y$ la population de prédateurs
			\item $a$ le taux de reproduction des proies,on suppose que les proies ont un accès illimité à la nourriture, elles donc une croissance exponentielle
			\item $c$ le taux de mortalité des prédateurs
			\item $b$ et $d$ les intéractions entre les deux populations
		\end{itemize}

		On a l'implémentation suivante :
		\begin{listing}[H]
			\scicode{\tdC 5-Lotka.sce}
			\caption{Système de Lotka}
			\label{code-3-Lotka}
		\end{listing}

		On prend différentes proportions de population entre les proies et les prédateurs.
		On obtient les résultats suivants :

		\begin{figure}[H]
			\centering
			\begin{subfigure}{.45\linewidth}
				\centering
				\includegraphics[width=\linewidth, trim=1.8cm 1.2cm 1.8cm 1.2cm, clip]{\tdC\img 5-Lotka_Curves.eps}
				\caption{Courbes de Lotka}
				\label{img-3-curvesLotka}
			\end{subfigure}
			\begin{subfigure}{.45\linewidth}
				\centering
				\includegraphics[width=\linewidth, trim=1.8cm 1.8cm 1.8cm 1.8cm, clip]{\tdC\img 5-Lotka_Cycles.eps}
				\caption{Cycles de Lotka}
				\label{img-3-cyclesLotka}
			\end{subfigure}
			\caption{Systèmes proies-prédateurs de Lotka}
			\label{img-3-Lotka}
		\end{figure}

		On remarque que peut importe les populations initiales (strictement positives), l'évolution se fait de manière cyclique. On a d'abord la prolifération des proies, puis des prédateurs qui diminuent la quantité de proies, ayant moins à manger ces derniers diminuent aussi et ainsi de suite.


	\chapter{Valeurs propres}
\label{ch-4}

% \section{Introduction}
	Dans ce chapitre, nous verrons comment étendre la notion de valeurs propres à des matrices rectangulaire dans la partie \ref{ch-4-SVD} mais aussi comment calculer un classement des noeuds les plus importants d'un graphe dans la partie \ref{ch-4-PageRank}.

\section{Décomposition en Valeurs Singulières (SVD)}
\label{ch-4-SVD}

	\subsection{Méthode SVD}
		SVD signifie \emph{Singular Value Decomposition}, il s'agit d'une méthode de factorisation de matrices rectangulaires.

		On connaît déjà la décomposition de matrices carrées $A\in\M_{nn}$ :
		\begin{itemize}
			\item Si $A$ est diagonalisable alors il existe $P$ inversible et $D$ diagonale contenant les valeurs propres de $A$ tels que : $A = PDP^{-1}$.
			\item De plus si $A$ est symétrique alors $P$ est orthogonale et $A=PDP^\top$.
			\item En revanche si $A$ n'est pas diagonalisable, on peut la trigonaliser avec $T$ une matrice triangulaire telle que $A=PTP^{-1}$.
		\end{itemize}

		L'idée de la SVD est de permettre de factoriser les matrices rectangulaires en étendant la notion de valeurs propres.

		\begin{theoreme}
			\label{th-4-svd}
			On suppose $m \geq n$.
			Pour toute matrice $A \in \M_{mn}$, il existe $U \in \M_{mm}$ et $V \in \M_{nn}$ et $\Sigma \in \M_{mn}$ de la forme
			$$
				\Sigma = \begin{pmatrix}
							\sigma_1 & 0 & ... & 0			\\
							0 & \sigma_2 & \ddots & \vdots	\\
							\vdots & \ddots & \ddots & 0	\\
							0 & ... & 0 & \sigma_n			\\
							0 & \hdotsfor{2} & 0			\\
							\vdots & \hdotsfor{2} & \vdots 	\\
							0 & \hdotsfor{2} & 0
						\end{pmatrix}
			$$ telle que les $\sigma_i$ sont positifs et triés par ordre décroissant : $\sigma_1 \geq \sigma_2 \geq ... \geq \sigma_n \geq 0$.
			On a alors : $A = U\Sigma V^T$.
		\end{theoreme}

		\begin{preuve}
			$AA^T$ et $A^T A$ sont des matrices symétriques et elles admettent donc les décompositions :
			\begin{align*}
				A^T A &= VD_1V^T \in \M_n \\
				A A^T &= UD_2U^T \in \M_m
			\end{align*}
			% TODO Revérifier
			On a :
			\begin{align*}
				A^T A
					&= V \Sigma^T U^T U \Sigma V^T 	\\
					&= V \Sigma^T \Sigma V^T		\\
					&= V D_1^2 V^T
			\end{align*}

			Par identification les $\sigma_i^2$ sont les valeurs propres de $A^T A$ et $V$ est la matrice de passage associée à $A^T A$.
			De même $U$ est celle associée à $A A^T$.
			% Ainsi $\pm\sigma_i$ sont les valeurs propres de A
			Ainsi on a $A = U\Sigma V^T$.
		\end{preuve}

		Nous avons donc :
		\begin{equation}
			\label{eq-4-SVD}
			A = U\Sigma V^T \iff AV = U\Sigma
		\end{equation}

		Notons $V=[\vec v_1 ... \vec v_n]$ et $U=[\vec u_1 ... \vec u_m]$.
		On a alors :
		\begin{equation}
			\label{eq-4-vpSVD}
			A \vec v_i = \sigma_i \vec u_i
		\end{equation}
		Ce qui ressemble fortement aux propriétés des valeurs propres des matrices carrées. 

		Soit $r$ le nombre de valeurs singulières non-nulles : $\sigma_{r+1} = ... = \sigma_n = 0$.
		On a la Décomposition en Valeurs Singulières suivantes :
		\begin{equation}
			\label{eq-4-decompoSVD}
				\begin{cases}
					A \vec v_i = \sigma_i \vec u_i		& 1 \leq i \leq r	\\
					A \vec v_i = 0						& r+1 \leq i \leq n
				\end{cases}
		\end{equation}
		$A$ est la somme de matrice de rang 1.

		Nous avons donc : $\Image A = \vect{\vec v_1, ..., \vec v_r}$ et $\Ker A = \vect{\vec v_{r+1}, ..., \vec v_n}$ et $\rang A = r$.

		Sous Scilab, on utilise l'outil \code{[U, S, V] = svd(A)}. En comparaison, \code{[vp, V] = spec(S)} donne les valeurs propres $vp$ d'une matrice carré et leur vecteurs propres associés $V$.

	\subsection{Compression d'image}

		Une des applications possibles de la SVD est de compresser des images. L'idée est de supprimer des valeurs singulières négligeables afin de réduire la taille de la factorisation.
		Une image en noir et blanc de résolution $m\times n$ pixels peut être vue comme une matrice $A\in\M_{mn}$ où $a_{ij}$ correspond au niveaux de gris du pixel.

		A partir de la décomposition en valeurs singulières \eqref{eq-4-decompoSVD}, nous avons :
		$$
			A = \sum_{i=1}^r \sigma_i u_i v_i^T
		$$

		Définissons la matrice compressée à partir de la $(k+1)$-ième valeur singulière $A_k$ telle que :
		\begin{equation}
			\label{eq-4-Ak}
			A_k =  \sum_{i=1}^k \sigma_i u_i v_i^T 
		\end{equation}
		On remarque que $A_r = A$.

		On a l'écart entre l'image originale $A$ et la version compressée $A_k$:
		\begin{equation}
			\label{eq-4-ecartAk}
			\norm{A-A_k}_F^2 = \sum_{i=k+1}^m \sigma_i^2
		\end{equation}
		où $\norm{.}_F$ est la \textbf{norme de Frobenius} :
		$$
			B = (b_{ij}) \to \norm{B}_F^2 = \sum b_{ij}^2
		$$
		Sous Scilab nous utiliserons : \code{norm(A, 'fro')}.

		\medskip

		% Dans le cas de la compression d'images en noir et blanc, on remarque que l'on n'a pas besoin de tous les $\sigma_i$ car certains sont quasiment nuls.
		Pour choisir les valeurs singulières négligeables, nous définissons un seuil $\epsilon > 0$ et considérons nuls les $\sigma_i$ dont la valeur absolue est inférieure au seuil choisi avec $i>k$. On remplace alors la matrice initiale $A$ par celle compressée $A_k$ définit en \eqref{eq-4-Ak}.

		Au départ, la matrice $A$ occupe $m\times n$ cases mémoire. Après compression, il faut stocker $2k$ vecteurs $u_i$ et $v_i$ soit $k(m +n)$ espaces mémoires et les $k$ valeurs singulières considérées comme intéressantes $\sigma_i$ ($1 \leq i \leq k$). Au total, on passe à $k(m+n+1)$ cases mémoires occupées après compression. Comme $k<<m$, la compression est intéressante.

		\medskip

		Nous allons utiliser cette méthode pour compresser l'image de Lena, une photographie très utilisée en traitement d'image pour des raisons historiques (des scientifiques pressés et désireux d'utiliser une nouvelle image d'essai ont trouvé cette image dans un magazine \emph{Playboy}).

		On a l'implémentation suivante :
		\begin{listing}[H]
			\scicode{\tdD 1-Compression.sce}
			\caption{Compression par SVD}
			\label{code-4-compression}
		\end{listing}

		Après avoir effectué la SVD sur la matrice de l'image, on va choisir $k_1$ et $k_2$ tel que $\sigma_{k_1} > \code{eps}\times\sigma_1$ et $\sigma_{k_1+1} < \code{eps}\times\sigma_1$ avec $\code{eps} = 0.005$ et $k_2 = 50$.

		On obtient les matrices d'images compressées $A_{k_1}$ et $A_{k_2}$ suivantes :
		\begin{figure}[H]
			\centering
			\includegraphics[width=\linewidth, trim=1cm .6cm 1cm .6cm, clip]{\tdD\img 1-Lena.png}
			\caption{Résultats de la compression}
			\label{img-3-compressionLena}
		\end{figure}

		On ne remarque que peu de différences entre la photo originale et celle compressée avec $k_1$ mais ce n'est pas le cas pour la photo $k_2$.

		Concernant la taille prise et la compression, on a :
		\begin{table}[H]
			\centering
			\begin{tabular}{|c|c|r|c|}
				\hline
				Photo					& \multicolumn{2}{c|}{Taille occupée}	& Compression\footnotemark[1]								\\	\hline\hline
				Originale				& 	$n\times n$		& 262144 valeurs	& $1- \frac{T_\text{comp}}{T_\text{orig}} $	\\	\hline
				Compressée $k_1 = 101$	& 	$k_1(n+n+1)$	& 103525 valeurs	& $60.51 \%$								\\	\hline
				Compressée $k_2 = 50$	& 	$k_2(n+n+1)$	& 51250 valeurs		& $80.45 \%$				 				\\	\hline
			\end{tabular}
			\caption{Résultats de la compression des images}
			\label{tb-4-tailleImages}
		\end{table}
\footnotetext[1]{{Ici nous définissons le taux de compression tel que une image compressée à un taux $t$ occupe $t$ \% de la taille originelle}}

		On conclut ainsi sur l'utilité de la compression SVD : avec la compression $k_1$ on réduit la taille de l'image par plus de la moitié pour un résultat quasiment semblable.
		D'autres techniques de compression existent, la plus connue étant la compression JPEG qui utilise des méthodes plus complexes et un sous-échantillonage de l'image.

	\subsection{Débruitage}

		Souvent en télécommunication, les signaux transmis sont légèrement bruités, la SVD permet alors de restituer le signal d'origine.
		Nous allons bruiter un signal de départ $u\in\R^{256}$ tel que $u(50)=u(52)=\frac{1}{4}, u(51)=\frac{1}{2}, u(150)=1$ et le reste à $0$.

		On construit le premier signal bruité $v$ tel que chacune de ses composantes soient une moyenne pondéré des composantes voisines, on crée alors un effet de flou diffu. Soit $T$ la matrice de floutage définie par :
		\begin{equation}
			\label{eq-4-Tfloutage}
			t_{ij} = C_i \times e^{-\frac{(i-j)^2}{10}}
		\end{equation}
		avec $C_i = \sum_{j=1}^{256}$ permettant de normaliser $T$ tel que la somme de chaque ligne soit égale à 1. 

		Construisons le second signal bruité $w = v + \eta$ où $\eta$ correspond à une perturbation de $u$ par un bruit blanc de $\frac{1}{100}$ :
		% \begin{equation}
		% 	\label{eq-4-bruitageW}
		% 	w = v + \eta = v + 
		% \end{equation}

		On pourrait naïvement essayer de débruiter le signal en $\tilde u	= T^{-1}w$ mais $T$ n'est pas inversible : $\abs{T}=0$.

		En effet on appliquant la SVD \eqref{eq-4-decompoSVD} à $T$ on a en passant au déterminant :
		$$
			\det T = \det U \times \det \Sigma \times \det V
		$$
		$U$ et $V$ sont orthogonales donc leur déterminant est égale à $\pm1$ et $\Sigma$ diagonales donc on a :
		$$
			\det T = \pm \prod_{i=1}^{256} \sigma_i
		$$
		Comme certaines valeurs singulières sont très proches de $0$, elles impactent le déterminant de $T$. On va alors tronquer $T$ en $T_k$ dont le déterminant sera non nul. $T_k$ correspond ainsi à une version compressée mais inversible de $T$ grâce à la SVD.

		On a ensuite la pseudo-inverse de $T$ :
		\begin{equation}
			\label{eq-4-pseudoInverse}
			T_k^+ = \sum_{i=1}^k \frac{1}{\sigma_i}V_iU_i^T
		\end{equation}
		qui nous donne un signal débruité : $u_k = T_k^+ w$.
		% TODO Pseudo inverse

		L'outil Scilab permettant de calculer la pseudo-inverse d'une matrice est \code{pinv}.

		Pour expérimenter cette technique sur l'image Léna, nous avons le code Scilab suivant :

		\begin{listing}[H]
			\scicode{\tdD 1.2-Debruitage_Lena.sce}
			\caption{Débruitage de Léna}
			\label{code-4-debruitageLena}
		\end{listing}

		Nous floutons d'abord l'image avec $T$, puis nous la bruitons avec un bruit blanc $\eta$ pour obtenir l'image bruitée $A_b$. Nous créons la matrice restaurée $A_r$ de la manière décrite précedemment avec la pseudo-inverse de $T$. 
		Nous obtenons les résultats suivants :

		\begin{figure}[H]
			\centering
			\includegraphics[width=\linewidth, trim=0.5cm 0.5cm 0.5cm 0.5cm, clip]{\tdD\img 1b-Debruitage_Lena.png}
			\caption{Débruitage de Léna}
			\label{img-4-debruitageLena}
		\end{figure}

		La version bruitée possède un écart de la norme de Frobenius supérieure à la version débruitée, on améliore bien la qualité de l'image après le processus de débruitage. De plus ce résultat ce voit visuellement.



	\subsection{Approximation au sens des moindres carrés}

		La SVD permet aussi d'approcher la solution d'un problème linéaire.
		Nous verrons plus en détail d'autres techniques dans le chapitre 5 \eqref{ch-6}.
		On cherche ici à résoudre $Ax=b$ avec $A\in\M_{mn}, x\in\M_{n1}, b\in\M_{m1}$. On a plusieurs cas :

		\begin{itemize}
			\item $b \in \Image A$ :
				\begin{itemize}
					\item $\Ker A = 0$		\quad $\to$ $A$ est inversible, l'unique solution est $x = A^{-1}b$
					\item $\Ker A \neq 0$	\quad $\to$ il existe une infinité de solutions de la forme $x = x_\text{particulier} + x_\text{Ker}$ 
					% TODO : 
				\end{itemize}

			\item $b \not\in \Image A$ : Dans ce cas là, qui arrive souvent en pratique, on doit approcher la valeur de $x$ par $x^*$
		\end{itemize}

		On cherche donc $x^*$ l'approximation de $x$ telle que $\norm{Ax^*-b}^2$ soit le plus petit possible.

		% TODO : m ou n
		Grâce à la SVD, on a : $A=U\Sigma V^T$
		\begin{align*}
			\norm{Ax-b}^2
				&= \norm{U\Sigma V^T x-b}^2 = \norm{U\Sigma V^Tx - UU^Tb}^2		\\
				&= \norm{U(\Sigma V^T x- U^Tb)}^2 = \norm{\Sigma V^T x - U^Tb}^2 \text{\qquad car $U$ est orthogonale donc il y a isométrie}
		\end{align*}

		On pose $z = V^T x$ et $c= U^Tb$.
		On a alors : 
		\begin{align*}
			\norm{Ax-b}^2
				&= \norm{\Sigma z - c}^2	\\
				&= \norm{[\sigma_1 z_1 - c_1, ..., \sigma_r z_r -c_r, 0 - c_{r+1}, ..., 0 - c_m]^T}^2	\\
				&= \sum_{i=1}^r (\sigma_i z_i - c_i)^2 + \sum_{i=r+1}^m c_i^2
		\end{align*}

		Il suffit donc de poser $\sigma_i z_i -c_i= 0 \forall i \in \{1, ..., r\}$

		De cette façon, on a :
		$$
			z_i = \begin{cases}
				\frac{c_i}{\sigma_i}	& i \in \{ 1, ..., r \} \\
				\text{arbitraire}		& i \in \{ r+1, ..., m \}
			\end{cases}
		$$

		On a enfin le résultat approché au sens des moindres carrés : $x^* = Vz$.
		Cette méthode fonctionne aussi si $A$ est de rang plein et donne la solution unique.

		% TODO Matrice pseudo inverse

		% TODO : Approximation Scilab moindre carrés


		Dans le Big Data, on a de très grandes matrices dont les colonnes correspondent à différentes caractéristiques. La SVD permet alors de dégager les $k$ caractéristiques les plus importantes par rapport aux valeurs singulières correspondantes : les $\sigma_i$ sont triés par ordre décroissant et donc les caractéristiques aussi.


\section{Méthode de la puissance}

	Soit $A \in \M_{nn}$ admettant $\lambda_1, ..., \lambda_n$ valeurs propres telles que $\abs{\lambda_1} > \abs{\lambda_2} \geq ... \geq \abs{\lambda_n}$. On appelle $\lambda_1$ la valeur propre dominante (\ref{def-4-vpDominante}).

	Le but de cette méthode est d'approcher $\lambda_1$ et $v_1$ son vecteur propre associé.

	On cherche alors à exprimer la suite : $x_k = Ax_{k-1}$ 
	On part de $x_0 \in \R$ tel que $x_0 = \sum_{i=1}^n \alpha_i v_i$ avec $\alpha_i \neq 0 \forall i \in \{1,...,n\}$. $\{v_1, ..., v_n\}$ forme une base de vecteurs propres de A.

	On a :
	\begin{align*}	
		x_1 &= Ax_0 = A \left( \sum_{i=1}^n \alpha_i v_i \right) = \sum_{i=1}^n \alpha_i Av_i = \sum_{i=1}^n \alpha_i \lambda_i v_i		\\
		x_2 &= Ax_1 = A \left( \sum_{i=1}^n \alpha_i \lambda_i v_i \right) = \sum_{i=1}^n \alpha_i \lambda_i Av_i = \sum_{i=1}^n \alpha_i \lambda_i^2 v_i
	\end{align*}
	
	Par récurrence on obtient :
	\begin{align*}
		x_k &= Ax_{k-1} = \sum_{i=1}^n \alpha_i \lambda_i^k v_i		\\
			&= \alpha_1 \lambda_1^k v_1 + \sum_{i=2}^n \alpha_i \lambda_i^k v_i	\\
			&= \alpha_1 \lambda_1^k v_1 \left( v_1 + \sum_{i=2}^n \frac{\alpha_i}{\alpha_1} \left( \frac{\lambda_i}{\lambda_1} \right)^k v_i  \right)
	\end{align*}

	Si $k>>1$, comme $\lambda_1 > \lambda_i \forall i >1$, alors $x_k \sim \lambda_1^k \alpha_1 v_1$. Ainsi $x_k$ a quasiment la même direction que $v_1$.

	Cependant il y a un problème : si $\abs{\lambda_1}>1$ alors $\lambda_i^k$ diverge. Pour remédier à cela, nous allons normaliser les $x_i$ en $y_i$.

	\begin{algorithm}[H]
	\DontPrintSemicolon
	\caption{Méthode de la puissance}
		\KwData{
			$y_0 \in \R^n$ donné\;
			\Indp\Indp$k$ la puissance désirée
		}
		\KwResult{$y_k \in \R^n$}
		\For{$i$ allant de $1$ à $k$}
		{
			$x_i = Ay_{i-1}$\;
			$y_i = \frac{x_k}{\norm{x_k}}$\;
		}
	\end{algorithm}

	Enfin nous povons récupérer la valeur propre dominante avec cette méthode :
	$$
		y_n^T A x_n = \scal{x_n}{Ax_n} = \scal{\pm v_1, \pm Ac_1}
		\lambda_1^k = y_k^T A y_k
	$$
	% TODO : ^^ vv Correct ?
	On peut montrer que :
	$$
		\abs{\lambda^k - \lambda_1} \leq C \left( \frac{\lambda_2}{\lambda_1} \right)^k
	$$

	Le code suivant est l'implémentation de cette méthode sous Scilab :

	\begin{listing}[H]
		\scicode{\tdD 2-Methode_Puissance.sce}
		\caption{Méthode de la puissance}
		\label{code-4-methodePuissance}
	\end{listing}

	Testons cette méthode sur la matrice suivante :
	\begin{equation}
		A = \begin{pmatrix}
				0.5172	& 0.5473	& -1.224	& 0.8012 	\\
				0.5473	& 1.388		& 1.353		& -1.112 	\\
				-1.224	& 1.353		& 0.03642	& 2.893 	\\
				0.8012	& -1.112	& 2.893		& 0.05827
		\end{pmatrix}
	\end{equation}

	Nous utilisons de code suivant pour tester la méthode \ref{code-4-methodePuissance} :

	\begin{listing}[H]
		\scicode{\tdD 2.1-Puissance.sce}
		\caption{Test de la méthode de la puissance}
		\label{code-4-testPuissance}
	\end{listing}

	On obtient la valeur propre dominante $\lambda = -3.9956707$.

	Grâce à \code{spec(A)}, nous obtenons les valeurs propres de la matrice :
	$$
		-3.9956707 		\hfill
		1.0005201		\hfill
		1.9927457		\hfill
		3.0022949		
	$$
	On remarque que $\lambda$ est bien la valeur propre de valeur absolue la plus grande, elle est bien dominante.

\section{Théorème de Perron-Frobenius}

	Soit $A \in \M_{nn}$.
	\begin{definition}
		\label{def-4-matPositive}
		Si $\forall (i,j) a_{ij} \geq 0 $ alors la matrice $A$ est dite positive et on note $A\geq 0$.
		Respectivement, si l'inégalité est stricte, la matrice $A>0$ est strictement positive.
	\end{definition}

	\begin{definition}
		\label{def-4-vpDominante}
		Une valeur propre $\lambda^*$ est dite dominante si $\forall \lambda \neq \lambda^* :\quad \abs{\lambda^*} > \abs{\lambda}$
	\end{definition}

	\begin{definition}
		\label{def-4-matPrimitive}
		Une matrice $A$ est dite primitive si elle est positive et si $\exists k \in \N \tq A^k > 0$.
	\end{definition}

	\begin{definition}
		\label{def-4-matIrreductible}
		Une matrice $A$ est dite irréductible si $\forall(i,j) \exists k \in \N \tq (A^k)_{ij} > 0$.
	\end{definition}
		
	Une matrice primitive est irréductible, cependant la réciproque est fausse. 
	% TODO Contre exemple

	\begin{theoreme}[Théorème de Perron]
		\label{th-4-perron}
		Si $A$ est une matrice primitive, alors $\lambda^*$ est une valeur propre simple et dominante et il existe un unique vecteur $x \in\M_{n1}$ positif et normalisé\footnote{On définit $x$ comme normalisé si la somme de ses composantes est égale à 1 : $\sum_{i=1}^n x_i = 1$} tel que $Ax = \lambda^*x$.
	\end{theoreme}

	\begin{theoreme}[Théorème de Frobenius]
		\label{th-4-frobenius}
		Si $A$ est une matrice irréductible, alors $\lambda^*$ est une valeur propre simple et il existe un unique vecteur $x \in\M_{n1}$ positif et normalisé tel que $Ax = \lambda^*x$.
	\end{theoreme}

	La seule différence entre ces deux théorèmes est que la valeur propre n'est pas dominante si la matrice n'est que irréductible dans le cas du théorème de Frobenius \eqref{th-4-frobenius}.

	% TODO : th Perron-Frobenius
	% TODO : mat stochastique
	% TODO : chaine Markov

\section{PageRank}
\label{ch-4-PageRank}

		    
		L'algorithme \emph{PageRank} a été conçu par Larry Page et Serguey Brin, les fondateurs de Google, dans le cadre d'une thèse à Stanford en 1996.
		L'idée principale est que la popularité et l'importance d'une page web se mesure avec les liens qui lui font référence.
		D'autres paramètres rentrent bien sûr en compte et nous les verrons au fur et à mesure de l'élabolartion de l'algorithme.

		Cette idée peut être appliquée sur n'importe quel graphe orienté tel qu'un réseau ferroviaire ou bien un ensemble d'articles scientifques se faisant référence entre eux; dans chaque cas on mesure l'importance de chaque sommet.

	\subsection{Construction de la méthode}
		    
		% Problèmes : impasses x, cycles x, fermes à réf
		Nous allons ici détaillé la construction de l'algorithme \emph{PageRank} dans le cas initial d'un réseau de pages web.
		On part d'un ensemble de $n$ pages $P_i$ ($i \in I = \{1,...,n\}$) reliées entre elles ou non selon la matrice d'adjacence $A \in \M_{nn}$ :
		$$
			a_{ij} = \begin{cases}
				1	& \text{si $j$ fait référence à $i$ (on a l'arc $P_j \to P_i$)}	\\
				0	& \text{sinon}			
			\end{cases}
		$$
		Les colonnes $A_j$ présentent donc les successeurs de $P_j$. 

		% TODO : Pj -> Pi ?????????

		En modélisant l'action d'un internaute lambda sur ce réseau, le but est de prédire quelles seront les pages les plus fréquentées et ainsi de les classer.
		On définit $E_k^i$ l'événement "\emph{être à la page $P_i$ après $k$ clics}" de probabilité $p_k^i = \proba(E_k^i)$. On suppose que le choix de la page de départ est équiprobable : $p_0^i = \frac{1}{n} \quad \forall i \in I$.
		A chaque itération $k$, l'internaute change de page.

		D'après la formule des probabilités totales, on a $\forall i \in I, \forall k \in \N$:
		$$
			p_{k+1}^i = \sum_{j\in I} \proba(E_{k+1}^i|E_k^j) \times \proba(E_k^j)
		$$

		On note le vecteur de probabilité $U_k\in\M_{n1}$:
		\begin{equation}
			\label{eq-4-u}
			U_k = \begin{pmatrix}
				p_k^1	\\			
				\vdots	\\			
				p_k^n		
			\end{pmatrix}		
		\end{equation}
		qui correspond à la probabilité d'être sur chaque page à l'itération $k$.

		On pose la matrice de transfert d'importance relative $H\in\M_{nn}$ :
		\begin{equation}
			\label{eq-4-h}
			H_i = \begin{cases}
				A_i \times \frac{1}{\displaystyle \sum_{i\in I} a_{ij}}		& \text{si } A_i \neq 0	\\
				\frac{1}{n} \times e										& \text{sinon}
			\end{cases}
		\end{equation}
		Où $e \in \M_{n1}$ est le vecteur rempli de $1$.

		Dans le cas où la page $P_i$ n'a aucun lien qui pointe vers elle, $A_i$ est nulle.
		Cependant nous voulons que la somme de chaque colonne $H_i$ soit égale à 1 et éviter les bloquage sur des pages sans successeurs, c'est pourquoi nous donnons $h_{ij} = \frac{1}{n}$ quand $A_i = 0$.
		Dans le cas contraire, on a $h_{ij} = \proba(E_{k+1}^i|E_k^j)$
		
		On a donc la chaîne de Markov suivante :
		\begin{equation}
			\label{eq-4-markov}
			\begin{cases}
				U_{k+1} = HU_k					\\
				U_0 = \frac{1}{n}\times e
			\end{cases}
		\end{equation}
		
		% TODO : Chaîne de Markov

		% TODO : vrai ?? vvv
		L'importance des pages est alors indiquée par la convergence de la suite $U_k$ vers $r\in\R^n$.
		Cependant si on s'arrête ici, on ne garantit pas la convergence dans le cas de circuits.

		Pour palier à ce problème et mieux représenter l'attitude d'un internaute, on suppose qu'à n'importe quel moment ce dernier peut choisir de réinitialiser sa navigation, c'est-à-dire quitter la page courante et choisir n'importe quelle page. On note (1-$\alpha$) la probabilité d'un tel événement $R$. La probabilité de choisir n'importe quelle page après réinitialisation est équiprobable et de probabilité $\frac{1}{n}$.
		
		On a finalement la matrice Google :
		\begin{equation}
			\label{eq-4-g}
			G = \alpha H + (1 - \alpha) \times \frac{1}{n}\times e e^T
		\end{equation}

		Google utilise $\alpha = 0.85$ : cela permet d'avoir un comportement sans trop de réinitialisations de la navigation mais permettant tout de même d'éviter les problèmes empêchant la convergence.

		On pose la somme de la colonne $j$ : $c_j = \sum_{i\in I} a_{ij}$.
		En résumant nous avons donc :
		\begin{equation}
			\label{eq-4-google}
			g_{ij} = \begin{cases}
				\alpha \frac{a_{ij}}{c_j} + (1-\alpha) \frac{1}{n}ee^\top	& \text{si } c_j \neq 0 \\
				\frac{1}{n}													& \text{si } c_j = 0
			\end{cases}
		\end{equation}


		La matrice Google $G$ est primitive car $G>0$ (ses composantes sont des probabilités). Elle est ainsi construite afin de remplir les conditions du théorème de Perron \eqref{th-4-perron}.

		On a : $Ge = \sum_{i=1}^n \, \underline{G}_i = e$ et $e$ normalisé
		Ainsi $\lambda^* = 1$ est la valeur propre dominante de $G$.


		On a alors l'unique vecteur propre $r$ positif et normalisé associé à la valeur propre dominante $\lambda_1 = 1$ :
		\begin{equation}
			\label{eq-4-rank}
			r = Gr
		\end{equation}
		L'importance des pages $r\in\M_{n1}$ est ainsi donnée par $r$.


		On peut calculer le rank $r$ à partir de \eqref{eq-4-rank} avec la méthode du point fixe. Comme la matrice $G$ est énorme, il est grandement préférable de faire les opérations pour des matrices creuses. Ici nous utilison la méthode de la puissance vu en \ref{code-4-methodePuissance} sur $G$ pour récupérer le rank $R$.
		Nous avons le code Scilab suivant :

		\begin{listing}[H]
			\scicode{\tdD 3-Rank.sce}
			\caption{Rank d'une matrice}
			\label{code-4-Rank}
		\end{listing}

		Il suffit juste de créer la matrice en format csv et de la passer à la fonction pour récuperer le rank. Cette technique n'est pas adaptée pour de grandes matrices mais ici nous n'utiliserons que de petites matrices carrées de taille inférieure à 12.
		Pour adapter cette fonction à de grandes matrices, il faudrait passer une matrice creuse construites préalablement.		
		% C = full(sparse(fscanfMat(fileName)));


	\subsection{Applications}

		\subsubsection{PageRank d'un réseau de 8 pages}

			Nous allons appliquer l'algorithme du \emph{PageRank} dans le réseau de pages suivant :

			\begin{figure}[H]
				\centering
				\includegraphics[width=\linewidth]{\tdD\img 3-web1.png}
				\caption{Réseau de 8 pages}
				\label{img-4-web1}
			\end{figure}

			Nous avons la matrice d'adjacence $A$ suivante :
			$$
				A = \begin{pmatrix}
					0 & 0 & 0 & 0 & 0 & 0 & 0 & 0 \\
					1 & 0 & 1 & 0 & 0 & 0 & 0 & 0 \\
					1 & 0 & 0 & 0 & 0 & 0 & 0 & 0 \\
					0 & 1 & 1 & 0 & 0 & 0 & 0 & 0 \\
					0 & 0 & 0 & 1 & 0 & 0 & 1 & 0 \\
					0 & 0 & 0 & 1 & 0 & 0 & 0 & 1 \\
					0 & 0 & 0 & 0 & 1 & 0 & 0 & 1 \\
					0 & 0 & 0 & 0 & 1 & 1 & 1 & 0
				\end{pmatrix}
			$$

			Puis la matrice $H$ :
			$$
				H = \begin{pmatrix}
					0 		& 0 		& 0 		& 0 		& 0 		& 0 		& 0 		& 0 	\\
					1/2		& 0 		& 1/2 		& 0 		& 0 		& 0 		& 0 		& 0 	\\
					1/2		& 0 		& 0 		& 0 		& 0 		& 0 		& 0 		& 0 	\\
					0 		& 1 		& 1/2 		& 0 		& 0 		& 0 		& 0 		& 0 	\\
					0 		& 0 		& 0 		& 1/2 		& 0 		& 0 		& 1/2		& 0 	\\
					0 		& 0 		& 0 		& 1/2 		& 0 		& 0 		& 0 		& 1/2 	\\
					0 		& 0 		& 0 		& 0 		& 1/2 		& 0 		& 0 		& 1/2 	\\
					0 		& 0 		& 0 		& 0 		& 1/2 		& 1 		& 1/2 		& 0
				\end{pmatrix}
			$$
			On a bien la somme de chaque colonne $H_i$ égale à 1 : $\sum_{i=1}^8 \, h_{ij} = 1, ~ \forall j$.

			On obtient ici la matrice stochastique $G$ :
			$$
				G = 0.85 \times H + 0.15\times \frac{1}{8}
			$$
			
			Appliquons l'algorithme \ref{code-4-Rank} avec le code suivant :

			\begin{listing}[H]
				\scicode{\tdD 3.1-PageRank.sce}
				\caption{PageRank du réseau de 8 pages}
				\label{code-4-PageRank8}
			\end{listing}

			Après tri du rank nous obtenons :
			\begin{table}[H]
				\centering
				\begin{tabular}{|r|l|}
					\hline
					Rank		& Page	\\	\hline
					\hline
					0.322208	& 8		\\	\hline
					0.213505	& 7		\\	\hline
					0.182237	& 6		\\	\hline
					0.136039	& 5		\\	\hline
					0.062469	& 4		\\	\hline
					0.038074	& 2		\\	\hline
					0.026719	& 3		\\	\hline
					0.018750	& 1		\\	\hline
				\end{tabular}
				\caption{Rank du réseau de 8 pages}
				\label{tb-4-rank8}
			\end{table}

			La page $P_8$ est donc la plus importante.

		\subsubsection{Rank d'un réseau ferroviaire}


			L'algorithme peut s'appliquer à n'importe quel réseau, y compris au réseau ferroviaire suivant :

			\begin{figure}[H]
				\centering
				\includegraphics[width=.6\linewidth]{\tdD\img 3-villes.png}
				\caption{Réseau ferroviaire entre 11 villes}
				\label{img-4-villes}
			\end{figure}

			Nous avons la matrice d'adjacence $A$ suivante :
			$$
				A = \begin{pmatrix}
					0 & 1 & 0 & 0 & 0 & 0 & 0 & 0 & 0 & 0 & 0 \\
					1 & 0 & 1 & 1 & 1 & 1 & 0 & 0 & 0 & 1 & 0 \\
					0 & 1 & 0 & 0 & 0 & 0 & 0 & 0 & 0 & 0 & 0 \\
					0 & 1 & 0 & 0 & 0 & 0 & 1 & 0 & 0 & 0 & 0 \\
					0 & 1 & 0 & 0 & 0 & 1 & 0 & 1 & 0 & 0 & 0 \\
					0 & 1 & 0 & 0 & 1 & 0 & 0 & 1 & 0 & 1 & 0 \\
					0 & 0 & 0 & 1 & 0 & 0 & 0 & 0 & 0 & 1 & 1 \\
					0 & 0 & 0 & 0 & 1 & 1 & 0 & 0 & 1 & 0 & 0 \\
					0 & 0 & 0 & 0 & 0 & 0 & 0 & 1 & 0 & 0 & 0 \\
					0 & 1 & 0 & 0 & 0 & 1 & 1 & 0 & 0 & 0 & 0 \\
					0 & 0 & 0 & 0 & 0 & 0 & 1 & 0 & 0 & 0 & 0
				\end{pmatrix}
			$$
			
			Appliquons pareillement l'algorithme \ref{code-4-Rank} avec le code suivant :

			\begin{listing}[H]
				\scicode{\tdD 3.2-Villes.sce}
				\caption{Rank du réseau ferroviaire}
				\label{code-4-rankVilles}
			\end{listing}

			Après tri du rank nous obtenons :
			\begin{table}[H]
				\centering
				\begin{tabular}{|r|l|}
					\hline
					Rank		& Villes	\\	\hline
					\hline
					0.201485	&  2 - Milan		\\	\hline
					0.129563	&  6 - Bergame		\\	\hline
					0.112639	&  7 - Crémone		\\	\hline
					0.106809	&  8 - Lecco		\\	\hline
					0.101626	& 10 - Brescia		\\	\hline
					0.099975	&  5 - Côme			\\	\hline
					0.074094	&  4 - Lodi			\\	\hline
					0.045551	& 11 - Mantoue		\\	\hline
					0.043899	&  9 - Sondrio		\\	\hline
					0.042180	&  1 - Varèse		\\	\hline
					0.042180	&  3 - Pavie		\\	\hline
				\end{tabular}
				\caption{Rank du réseau de 8 pages}
				\label{tb-4-rankVilles}
			\end{table}

			Sans surprise, Milan est la ville la plus importante du réseau.
			Cet algorithme peut donc servir dans la prise de décision du nombre de trains passant sur chaque ligne par exemple.



	\chapter{Optimisation}
\label{ch-5}

% \section{Introduction}
	
		Un problème d'optimisation dans $\R^n$ peut se présenter de deux manière différentes.
		Soit nous cherchons à minimiser (respectivement maximiser) une fonction $f$, c'est-à-dire trouver $x^* \tq f(x^*) \leq f(x) ~\forall x$ (respectivement $f(x^*)\geq f(x) $).
		% Nous verrons ce type d'optimisation dans la partie .
		Soit nous cherchons à trouver un modèle $f$ qui corresponde le mieux à des données. Ces deux problèmes, bien qu'initialement différents, se résolvent quasiment de la même manière. Nous verrons cela au long du chapitre. 

\section{Optimisation}

	\subsection{Cadre statistique}
		A partir de $n$ données $(x_i, y_i)$ on cherche un modèle $f(x, \theta)$ paramétré par $\theta$.

		Ainsi $x \in \R$ est la variable indépendante, $y \in \R^n$ la variable dépendante de $x$ qui correspond aux résultats observés selon les conditions $x$, et $\theta \in \R^p$ les $p$ paramètres du modèle $f$.
		Il faut d'abord choisir un modèle pertinent et ensuite trouver les paramètres qui permettent de minimiser les écarts entre le modèle théorique et les résultats obtenus expérimentalement afin d'obtenir le modèle paramétré le plus vraisemblable face aux données.

		Comme les $y_i$ sont des mesures, il y a forcément des incertitudes : ce sont donc des variables alétoires que l'on suppose statistiquement indépendantes telles que :
		\begin{equation}
			\label{eq-5-yiEpsilon}
			y_i = f(x_i, \theta) + \epsilon_i
		\end{equation}
		
		Où toute la partie aléatoire est contenue dans la variable aléatoire $\epsilon_i$ d'espérance nulle (il y a autant de probablité de sur-estimer que de sous-estimer $y_i$) et de variance $\sigma^2$. $\epsilon$ représente ainsi les erreurs de mesures.

		On pose $g$ la densité de $\epsilon$.
		La densité de probabilité de $y_i$ est alors :
		$$
			\phi_i(y_i,\theta) = g(y - f(x_i, \theta))
		$$
		Et on a :
		$$
			E[y_i|\theta] = f(x_i,\theta)
		$$

		Les $(y_i)_{i=1..n}$ sont indépendants, la densité conjointe du vecteur $Y = (y_1, ..., y_n)$ est alors :
		\begin{equation}
			\label{eq-5-phi}
			\phi(Y, \theta) = \prod_{i=1}^n \phi_i(y_i,\theta)
		\end{equation}
		
		Ainsi la probabilité que les données expérimentales se trouvent dans un certain domaine $D\in\R^n$ est :
		$$
			\proba(Y \in D | \theta) = \int_D \phi(Y, \theta) dY
		$$

		On définit maintenant $L(\theta,Y) = \phi(Y, \theta)$ la fonction de vraisemblance (\emph{Likelihood function} en anglais, d'où le $L$). Sa différence avec $\phi$ la densité de $Y$ où $\theta$ est fixé et $Y$ est la variable aléatoire; est que cette fois $Y$ est fixé par les données obtenues expérimentalement et les paramètres $\theta$ constituent la variable, de sorte que la plus haute vraisemblance est atteinte pour $\hat\theta$ tel que :
		$$
			\hat\theta = \arg \max_{\theta \in \R^p} L(\theta,Y)
		$$

		A partir de là, on peut effectuer différentes hypothèses sur la densité de $\epsilon$ qui donneront des méthodes différentes.


	\subsection{Méthode des moindres carrés}

		Pour cette méthode, on suppose que $\epsilon$ suit une loi normale $\mathcal{N}(0, \sigma^2)$ :
		\begin{equation}
			\label{eq-5-errorLSM}
			g(\epsilon) = \frac{1}{\sqrt{2\pi\sigma^2}} \exp^{-\frac{1}{2\sigma^2} \times \epsilon^2} 
		\end{equation}
		On a donc :
		$$
			\phi_i(y_i,\theta) = \frac{1}{\sqrt{2\pi\sigma^2}} \exp^{-\frac{1}{2\sigma^2} \times \left( y - f(x_i, \theta)\right)^2} 
		$$

		A partir de \eqref{eq-5-phi}, on obtient $\phi$ puis directement $L$ :
		\begin{align*}
			L(\theta,Y) 	&= \prod_{i=1}^n \frac{1}{\sqrt{2\pi\sigma^2}} \exp^{-\frac{1}{2\sigma^2} \times \left( y - f(x_i, \theta)\right)^2}
			\\
			&= \frac{1}{\sqrt{2\pi\sigma^2}} \exp^{-\frac{1}{2\sigma^2} \times \sum_{i=1}^n \left( y - f(x_i, \theta)\right)^2}
		\end{align*}

		Maximiser $L$, et donc une exponentielle, revient à minimiser la somme des carrés.
		$$
			\hat\theta = \arg \max_{\theta \in \R^p} L(\theta,Y) = \arg \min_{\theta \in \R^p} S(\theta) 
		$$
		Ainsi nous ramenons le problème de maximisation de $L(\theta,Y)$ à la minimisation de $S(\theta)$ définie par :

		\begin{equation}
			\label{eq-5-LSM}
			S(\theta) = \sum_{i=1}^n \left( y_i - f(x_i, \theta)\right)^2 = \sum_{i=1}^n ( r_i(\theta) )^2 = \norm{r(\theta)}^2
		\end{equation}

		On appelle $r(\theta)$ le vecteur des résidus.
		En partant de l'hypothèse que la distribution de $\epsilon$ est Gaussienne \eqref{eq-5-errorLSM}, on obtient ainsi la méthode des moindres carrés \eqref{eq-5-LSM}.


	\subsection{Méthode de la moindre déviation absolue}

		Pourquoi utiliser la somme des normes au carré et non seulement la somme des normes
		On suppose ici que $\epsilon$ suit une distribution de Laplace :
		\begin{equation}
			\label{eq-5-errorLADM}
			g(\epsilon) = \frac{1}{\sqrt{2 \sigma^2}} \exp^{-\frac{\sqrt{2}}{\sigma} \times \abs{\epsilon}} 
		\end{equation}

		Ce qui nous donne :
		\begin{align*}
			L(\theta,Y) 	&= \prod_{i=1}^n \frac{1}{\sqrt{2 \sigma^2}} \exp^{-\frac{\sqrt{2}}{\sigma} \times \abs{y - f(x_i, \theta)}}
			\\
			&= \prod_{i=1}^n \frac{1}{\sqrt{2 \sigma^2}} \exp^{-\frac{\sqrt{2}}{\sigma} \times \sum_{i=1}^n \abs{y - f(x_i, \theta)}}
		\end{align*}

		Ainsi nous obtenons :
		\begin{equation}
			\label{eq-5-LADM}
			S(\theta) = \sum_{i=1}^n \abs{y - f(x_i, \theta)}
		\end{equation}

		Cependant cette hypothèse pose déjà un problème, $S$ n'est pas différentiable.
		C'est pourquoi on préfère généralement utiliser la méthode des moindres carrés \eqref{eq-5-LSM} qui fournit un bon support si l'on ne connaît pas la distribution des incertitudes $\epsilon$ pour une expérience.

		% TODO : image de comparaison


		On voit bien que la méthodes des moindres carrés et celle de la moindre déviation absolue ont des distributions différentes, ainsi les paramètres obtenus avec les même données pour chaque méthode seront différents.

	% \subsection{Évaluation des solutions}
	% 	Comme les données $y$ sont des variables aléatoires, la solution $\hat\theta$ l'est aussi.

	% 	On peut trouver l'intervalle de confiance de (la distribution de ???) $\hat\theta$ par deux méthodes :

	% 	La méthode de Monte-Carlo : % TODO

	% 	Statistiques linéairisées : % TODO




	\subsection{Sélection des modèles}
	    De plus on se peut se demander quel modèle choisir par ceux entraînés : par exemple dans le cas d'une régression polynomiale quel degré choisir ? En effet plus le degré du polynome augmente, plus l'erreur diminue.

	    Le principe est le suivant : on sépare les données en deux sets, un set de validation $V$ pour calculer l'erreur et son complémentaire $T$ qui sert à former le modèle.
	    On forme d'abord les paramètres $\theta$ avec $T$, il s'agit de la phase d'apprentissage.
	    On choisit alors le modèle réalisant l'erreur minimale sur le set de validation $S_V(\theta_k)$, on valide donc le bon modèle sur des données sur lesquelles il n'a pas été entraîné.

	    Cependant le choix des sets impacte la qualité de la validation : par exemple si on sépare les données en deux à partir d'une certaine valeur, on observera une extrapolation assez brutale qui ne coïncidera que très peut avec le set $V$.
	    Pour éviter ce genre de biais, on préfèrera utiliser chaque donnée dans les deux types de sets, en réalisant les tests sur plusieurs sets $T$ et $V$. Cette validation croisée est utile notamment quand le set $T$ n'est pas très équilibré.

\section{Problème linéaires}

	% \subsection{Linéarité}
		% Le modèle $y = f(x, \theta)$ est dit linéaire si
		% $$
		% 	y = \sum_{k=1}^p \theta_k \Phi_k(x)
		% $$


	\subsection{Régression polynomiale}
		On a $y = \sum_{k=0}^d \theta_k x^k = \theta_0 + \theta_1 x + ... + \theta_d x^d$ un modèle polynomiale de degré $d$. En utilisant la méthode des moindres carrés, on a :
		$$
			S(\theta) = \sum_{i=1}^n \left( \sum_{k=0}^d \theta_k x^k - y_i \right) = \norm{r(\theta)}^2 
		$$

		Le vecteur résiduel $r$ s'écrit alors :
		Que l'on décompose :
		\begin{equation}
			\label{eq-5-rLSM}
			r(\theta) = A\theta - y
		\end{equation}
		où
		$$
			r_i(\theta) = \begin{pmatrix}
							1 & x_i & \ldots & x_i^d 
						\end{pmatrix}
						\begin{pmatrix}
							\theta_0 \\
							\theta_1 \\
							\vdots \\
							\theta_d \\
						\end{pmatrix}
		$$
		Avec $A$ la matrice de Vandermonde suivante :
		\begin{equation}
			\label{eq-5-matVandermonde}
			A = \begin{pmatrix}
					1	& x_1	& x_1^2 & ... & x_1^d \\
					1	& x_2	& x_2^2 & ... & x_2^d \\
					\vdots	& \vdots	& \vdots & \ddots & \vdots \\
					% 1	& x_i	& x_i^2 & ... & x_i^d \\
					1	& x_n	& x_n^2 & ... & x_n^d
				\end{pmatrix}
		\end{equation}

		Cette matrice particulière est inversible si et seulement si les $x_i$ sont tous distincts.

		Ainsi on ramène le problème initial à trouver le minimum de $S$, $\hat\theta$ tel que :
		$$
			\grad S(\theta) = \grad \norm{A \hat\theta - Y}^2 = 0
		$$
		On trouve le gradient grâce au développement limité suivant :
		\begin{align*}
			S(\theta + h) 	&= \norm{A(\theta+h)-Y}^2 = \norm{A\theta-Y+Ah}^2								\\
							&= (A\theta-Y+Ah)^T (A\theta-Y+\alpha)											\\
							&= (A\theta-Y)^T(A\theta-Y) + (A\theta-Y)^T Ah + (Ah)^T (A\theta-Y) (Ah)^T Ah	\\
							&= \norm{A\theta-Y}^2 + 2(A\theta-Y)^T Ah + \norm{Ah}^2							\\
							&= S(\theta) + \grad S(\theta)^T h + \norm{Ah}^2
		\end{align*}
		Par identification $\norm{Ah}^2\limto{h}{0}0$ est le reste et on a le gradient :
		\begin{equation}
			\label{eq-5-gradLLS}
			\grad S(\theta) = 2 A^T (A\theta -y)
		\end{equation}

		La solution $\hat\theta$ est donnée par :
		\begin{align*}
			\grad S(\hat\theta) = 0	&\iff 2 A^T (A\hat\theta -y) = 0		\\
									&\iff A^T A\hat\theta - A^Ty = 0		\\
									&\iff A^T A\hat\theta = A^Ty		
		\end{align*}
		Cette solution est unique si le rang de $A$ est $p$ : $A^T A\hat\theta = A^Ty \implies A\hat\theta = y$.
		Il n'y a plus qu'à résoudre le système linéaire par la méthode de Gauss\footnote{Simplement faire $\theta = A\backslash y$ sous Scilab}.

		\begin{proof}
			$\forall \theta \in \R^p :$
			\begin{align*}
				S(\theta) = S(\hat\theta + \theta - \hat\theta)
					&= S(\hat\theta) + \grad S(\hat\theta)^T (\theta - \hat\theta) + \norm{A(\theta -\hat\theta)}^2 \\
					&= S(\hat\theta) + \norm{A(\theta -\hat\theta)}^2 \\
					&\geq S(\hat\theta)
			\end{align*}
			Donc $\hat\theta$ est solution minimale de $S$.

			\begin{align*}
				S(\hat\theta) = S(\theta) 	&\iff \norm{A(\theta - \hat\theta)}^2 = 0			\\
											&\iff A(\theta - \hat\theta) = 0 \text{\qquad or A est injective car son rang est plein}	\\
											&\iff \theta = \hat\theta
			\end{align*}
			Ainsi $\hat\theta$ est unique si $A$ est de rang $p$.
		\end{proof}

		\begin{ex}[Régression linéaire]
			On cherche $\theta_1$ et $\theta_2$ tels que :
			$$
				y = f(\theta) = \theta_1 + \theta_2 x
			$$
			
			Avec la méthode des moindres carrés, on a :
			$$
				S(\theta) = \sum_{i=1}^n \left( y_i - f(x_i, \theta) \right)^2
			$$
			Le but est alors de chercher $\theta^*$ minimisant $S$ \ie :
			$$
				\theta^* = \arg \min_{\theta \in \R^p} S(\theta)
			$$
		\end{ex}


		Nous allons voir ici l'application de la régression polynomiale sur un set de données. Nous avons le code suivant :

		\begin{listing}[H]
			\scicode{\tdE 1-Regression_Polynomiale_CV.sce}
			\caption{Régression Polynomiale}
			\label{code-5-regPoly}
		\end{listing}

		Nous allons construire plusieurs polynomes $P_0$ à $P_{10}$ du degré $p=0$ ($y=C\in\R$) au degré $p=10$ sur les couples de données $(t,y)$. Ici nous allons faire une validation croisée avec 3 sets différents, nous effectuerons ensuite une moyenne des erreurs et des coefficients sur chaque set pour chaque polynome $P_p$ afin de choisir celui avec l'erreur minimale et qui collera donc le mieux aux données.

		Pour chaque set, les coefficients des polynomes ($\theta_p$) sont calculés avec un set d'apprentissage $T$ par la méthode de Gauss. On calcule ensuite $y_\text{Poly}$ pour chaque polynome sur l'ensemble des données pour estimer l'erreur sur le set de validation.

		Nous obtenus les résultats suivants :

		\begin{figure}[H]
			\centering
			\includegraphics[width=\linewidth, trim=3cm 3cm 3cm 3cm, clip]{\tdE\img 1-Regression_Polynomiale_CV.eps}
			\caption{Résultats de la Régression Polynomiale}
			\label{img-5-regPoly}
		\end{figure}

		Le modèle le plus approprié serait alors :
		$$
			P_4(x) = 1.599 -2.521 x^1 -0.060 x^2 -1.825 x^3 + 0.266 x^4
		$$
		Cependant on remarque qu'il y a relativement peu d'écart entre $P_3$, $P_4$ et $P_5$, les trois sont donc acceptables mais nous préfèrons $P_4$.

		\begin{figure}[H]
			\centering
			\includegraphics[width=\linewidth, trim=5cm 3cm 5cm 3cm, clip]{\tdE\img 1-Modele_Optimal_P4.eps}
			\caption{Polynome Optimal $P_4$}
			\label{img-5-modeleOptimalReg}
		\end{figure}



	\subsection{Régression polynomiale d'un cercle}

		Un cas particulier de l'utilisation de la régression polynomiale est si la fonction recherchée est celle d'un cercle.
		On cherche alors à minimiser la distance  :
		\begin{equation}
			\label{eq-5-dcercle}
			d(a,b,R) = \sum_{i=1}^n \left( (x_i - a)^2 + (y_i - b)^2 -R^2 \right)^2 = \norm{r}^2
		\end{equation}

		Cependant cette forme n'est pas linéaire. On pose alors :
		\begin{align*}
			r_i &= R^2 - a^2 - b^2 + 2ax-i + 2by-i - (x_i^2 + y_i^2) \\
				&= 	\begin{pmatrix}
						2x_i & 2y_i & 1
					\end{pmatrix}
					\begin{pmatrix}
						a \\
						b \\
						R^2 -a^2 - b^2
					\end{pmatrix}
					- (x_i^2 + y_i^2)
		\end{align*}
		Cette forme est linéaire : $\theta = \begin{pmatrix} a & b & R^2 -a^2 - b^2 \end{pmatrix}^T$

		Ainsi on pose :
		$$
			A = \begin{pmatrix}
					2x_1 & 2y_1 & 1	\\
					2x_2 & 2y_2 & 1	\\
					\vdots & \vdots & \vdots \\
					2x_n & 2y_n & 1	\\
				\end{pmatrix}
			\qquad\text{ et }\qquad
			z = \begin{pmatrix}
					x_1^2 + y_1^2 \\
					x_2^2 + y_2^2 \\
					\vdots \\
					x_n^2 + y_n^2 \\
				\end{pmatrix}
		$$
		Et on obtient le problème linéaire 
		\begin{equation}
			\label{eq-5-cercleS}
			S(\theta) = d(a,b,R) = \norm{A\theta - z}^2 = \norm{r(\theta)}^2
		\end{equation}
		

		Effectuons cette régression avec le code suivant :

		\begin{listing}[H]
			\scicode{\tdE 1.1-Regression_Cercle.sce}
			\caption{Régression d'un cercle}
			\label{code-5-regCercle}
		\end{listing}

		On construit d'abord un cercle bruité qui nous servira de données 'expérimentales'. On résout ensuite le problème de moindre carré linéaire \eqref{code-5-regCercle}. On obtient le résultat suivant :

		\begin{figure}[H]
			\centering
			\includegraphics[width=.6\linewidth, trim=3cm 3cm 3cm 3cm, clip]{\tdE\img 1a-Regression_Cercle.eps}
			\caption{Régression d'un cercle}
			\label{img-5-regCercle}
		\end{figure}

		On obtient le cercle de centre $(a; b) = (1.498701; 1.496565)$ et de rayon $R=0.9999299$. On est donc très proche de la solution malgré le bruit, témoignant ainsi de l'efficacité de la méthode.


\section{Problèmes non-linéaires}

	\subsection{Erreurs possibles}
		Prenons comme example le modèle non-linéaire suivant :
		$$
			y = f(x,\theta) = e^{\theta_0 + \theta_1 x}
		$$

		On pourrait penser qu'utiliser le logarithme permettrait de se ramener au problème linéaire :
		\begin{equation}
			\label{eq-5-logTrickS}
			S_{log} = \sum_{i=1}^n \left( \ln y_i - (\theta_0 + \theta_1 x) \right)^2
		\end{equation}
		Cependant cela n'est pas correct car si $y_i - f(x_i, \theta)$ a une distribution normale, ce n'est pas le cas pour $\log y_i - \log f(x_i, \theta)$.

		On pourrait aussi essayer de calculer le gradient de $S$ et l'annuler avec la méthode de Newton mais il faudrait calculer la jacobienne de $r = y_i - f(x, \theta) - y$ et cela n'assurerait pas un minimum mais donnerait peut-être un maximum ou un point selle.

		Une solution viable est la suivante, on pose le développement limité de $\theta$ en $\theta_k$ :
		$$
			r(\theta) = r(\theta_k) + J_r(\theta)(\theta - \theta_k) + \norm{\theta - \theta_k}\epsilon(\theta - \theta_k)
		$$
		Ainsi, trouver $\theta_{k+1}$ minimisant $S_k(\theta) = \norm{r(\theta_k) +J_r(\theta_k)(\theta -\theta_k) }^2$ 
		est un problème des moindres carrés linéaires que l'on peut résoudre avec la méthode de Newton : .


		Ainsi pour résoudre des problèmes de moindre carré non-linéaires nous avons les méthodes suivantes, qui sont aussi valables pour dans le cas linéaire mais plus lourdes.


	\subsection{Méthode du gradient}

		% TODO : méthode descente, point de descente, conditions
		% Une condition nécessaire pour que $\hat x$ soit un minimum local de $f$est que le gradient de $f$ en ce point s'annule : $\grad f(\hat x) = 0$
		% Direction trouvée => choix du pas par optimisation d'une fonction à 1 var

		La méthode du gradient est une méthode de descente : on choisit une direction $d_k = -\grad f(x_k)$ et un pas $p_k$ tel que $x_{k+1} = x_k + p_k\times d_k$ 'descende' par rapport à $x_k$, c'est-à-dire $f(x_{k+1})<f(x_k)$.

		Concernant le pas, on peut choisir un pas $p_k$ constant ou bien variable selon $x_k$.
		% TODO : développer sur les formes quadratiques

		Un pas constant pose divers problèmes : s'il est mal choisi la méthode converge très lentement (pas trop petit) ou bien diverge (trop grand). En revanche nous pouvons pour chaque itération calculer un pas optimal.

		\begin{algorithm}[H]
		\DontPrintSemicolon
		\caption{Méthode du gradient}
			\KwData{$x_0 \in \R^n$ donné}
			\KwResult{$\hat x \in \R^n$ minimum local de $f$ approchée à $\epsilon$ près}
			\While{($\norm{f(x_k)}>\epsilon$ et $J_f(x_k)$ inversible)}
			{
				$x_{k+1} = x_k - \rho_k \times \grad f(x_k)$\;
			}
		\end{algorithm}
	

		% Pour l'implémenter dans Scilab, nous avons besoin de la Jacobienne de $f$, on peut aussi utiliser .. pour l'approcher.

		On a l'implémentation Scilab suivante :
		\begin{listing}[H]
			\scicode{\tdE 2-Methode_Gradient.sce}
			\caption{Méthode du Gradient}
			\label{code-5-methGrad}
		\end{listing}

		Nous allons utiliser cette méthode sur la forme quadratique suivante :
		\begin{equation}
			\label{eq-5-formeQuad}
			f(x) = \frac{1}{2} x^T A x - b^T x
		\end{equation}
		avec $A$ une matrice symétrique définie positive. On choisit :
		$$
			A = \begin{pmatrix}
				2	&	-1	\\
				-1	&	2	
			\end{pmatrix}
			\qquad
			b = \begin{pmatrix}
				1	\\
				1
			\end{pmatrix}
		$$

		\begin{listing}[H]
			\scicode{\tdE 2.1-Methode_Gradient_Quadratique.sce}
			\caption{Méthode du Gradient pour une fonction quadratique}
			\label{code-5-methGradQuad}
		\end{listing}

		Nous choissisons d'abord un pas fixe $p = 0.2$
		L'algorithme converge alors en $79$ itérations.
		\begin{figure}[H]
			\centering
			\includegraphics[width=.4\linewidth, trim=3cm 3cm 3cm 3cm, clip]{\tdE\img 2a-Gradient_Quad_Pas02.eps}
			\caption{Méthode du Gradient pour une forme quadratique avec un pas fixe}
			\label{img-5-methGradQuadFixe}
		\end{figure}

		Dans le cas de $f$ sous forme quadratique, le pas optimal est le suivant :
		% Démo
		\begin{equation}
			\label{eq-5-pasOptimal}
			p_k = \frac{\norm{\grad f}^2}{{\grad f}' \times A \times \grad f}
		\end{equation}

		Avec ce pas optimal, on s'aperçoit que la convergence est plus rapide avec désormais $18$ itérations.
		\begin{figure}[H]
			\centering
			\includegraphics[width=.4\linewidth, trim=3cm 3cm 3cm 3cm, clip]{\tdE\img 2a-Gradient_Quad_Opti.eps}
			\caption{Méthode du Gradient pour une forme quadratique avec un pas optimal}
			\label{img-5-methGradQuadOpti}
		\end{figure}

		Cependant cette méthode a des limites. Prenons la fonction de Rosenbrock :
		\begin{equation}
			\label{eq-5-fRosenbrock}
			f(x,y) = (1-x)^2 + 100(y-x^2)^2;
		\end{equation}

		Cette fonction particulière est souvent utilisée pour mettre à l'épreuve les algorithmes d'optimisation.
		On s'aperçoit facilement qu'elle admet un minimum global en $(1,1)$.

		Nous allons appliquer la méthode du gradient.
		Concernant le pas optimal pour cette fonction, on utilise la méthode de la section dorée fournie par Stéphane Mottelet qui fonctionne quelque soit la fonction à minimiser.

		\begin{listing}[H]
			\scicode{\tdE 2.2-Methode_Gradient_Rosenbrock.sce}
			\caption{Méthode du Gradient pour la fonction de Rosenbrock}
			\label{code-5-methGradRosen}
		\end{listing}
				
		Ici on ne trace pas les courbes iso-valeur à chaque itération mais l'allure de la surface (plus les couleurs tendent vers le bleu, plus la fonction est "basse").
		
		\begin{figure}[H]
			\centering
			\includegraphics[width=.6\linewidth, trim=3cm 3cm 3cm 3cm, clip]{\tdE\img 2b-Gradient_Rosenbrock.eps}
			\caption{Méthode du Gradient sur la fonction de Rosenbrock}
			\label{img-5-methGradRosen}
		\end{figure}

		Après $1000$ itérations, la solution n'a toujours pas convergé et on obtient un écart en norme de $0.2173970$ par rapport à la solution $(1,1)$.

		On s'aperçoit que les itérations successives s'approche très lentement du minimum et restent bloquées dans une "vallée".
		On trouve donc une limite à la méthode du gradient, nous avons alors d'autres méthodes plus efficaces.

	\subsection{Méthode de Gauss-Newton}

		L'idée de cette méthode est la suivante. On part de la méthode des moindre carré \eqref{eq-5-LSM} : $S(\theta) = \norm{r(\theta)}^2$ où nous cherchons à minimiser $S$. Effectuons le développement limité de $r(\theta)$ à l'itération $\theta_k$ :
		$$
			r(\theta) = r(\theta_k) + J_r(\theta_k)(\theta-\theta_k) + \norm{\theta-\theta_k} \epsilon(\theta-\theta_k)
		$$
		Ainsi il suffit de trouver $\theta_{k+1}$ minimisant :
		$$
			S_k(\theta) = \norm{r(\theta_k) + J_r(\theta_k) (\theta-\theta_k)}^2
						= \norm{r_k(\theta)}^2
		$$
		qui est un problème de moindre carré linéaire.
		On a donc :
		\begin{align*}
			\theta_{k+1} 
						&= \theta_k - J_r(\theta_k)^{-1}  r(\theta_k)	\\
						&= \theta_k - \left( J_r(\theta_k)^T J_r(\theta_k) \right)^{-1} J_r(\theta_k)^T r(\theta_k)	\\
						&= \theta_k - \frac{1}{2} \left( J_r(\theta_k)^T J_r(\theta_k)  \right)^{-1} \grad S(\theta_k)
		\end{align*}

		Le problème est quand le rang de la jacobienne en $\theta_k$ est dégénéré. Pour remédier à cela, on empêche que $\theta_{k+1}$ soit trop éloigné de $\theta_k$
		La méthode suivante permet d'éviter ce problème.

	\subsection{Méthode de Levenberg-Marquardt}

		Nous avons la méthode de Levenberg-Marquardt :
		\begin{equation}
			\label{eq-5-levenbergMarquardt}
			\theta_{k+1} = \theta_k - \frac{1}{2}\left( J_r(\theta_k)^T J_r(\theta_k) + \lambda I \right)^{-1} \grad S(\theta_k)			
		\end{equation}
		On peut choisir de régler cette méthode entre :
		\begin{itemize}
			\item la rapidité quand $\lambda \to 0$ : on se rapproche de la méthode de Gauss-Newton.
			\item la sureté $\lambda \to \infty$ : on est plus proche de la méthode du gradient.
		\end{itemize}

		Scilab implémente aussi l'outil \code{lsqrsolve} qui utilise cette méthode pour résoudre un problème non-linéaire au sens des moindres carrés. 

		Implémentons la sous Scilab de la manière suivante :

		\begin{listing}[H]
			\scicode{\tdE 3-Methode_LM.sce}
			\caption{Méthode de Levenberg-Marquardt}
			\label{code-5-methLM}
		\end{listing}

		Testons la sur la fonction de Rosenbrock \eqref{eq-5-fRosenbrock}.
		On a le code suivant :

		\begin{listing}[H]
			\scicode{\tdE 3.1-Methode_LM_Rosenbrock.sce}
			\caption{Méthode de Levenberg-Marquardt pour la fonction de Rosenbrock}
			\label{code-5-methLMRosen}
		\end{listing}

		On applique l'algorithme pour plusieurs valeur de $\lambda$ afin de voir l'impact de ce paramètre sur la vitesse de convergence de la méthode.
		On obtient les résultats suivants :

		\begin{figure}[H]
			\centering
			\includegraphics[width=\linewidth, trim=1cm 1cm 1cm 1cm, clip]{\tdE\img 3-LM_4_lambda.eps}
			\caption{Méthode de Levenberg-Marquardt avec différentes valeurs de $\lambda$}
			\label{img-5-LMRosen4lambda}
		\end{figure}

		On remarque bien que plus $\lambda$ est petit, plus la convergence est rapide, allant même jusqu'à atteindre le minimum de la fonction de Rosenbrock en $3$ itérations ! Cette méthode est donc bien plus efficace que celle du gradient.	


	% \chapter{Équations différentielles partielles}
\chapter{Séries de Fourier}
\label{ch-6}

% \section{Introduction}

	Dans ce chapitre nous aborderons les séries de Fourier. Cet outil permet d'approcher n'importe quelle fonction périodique par une somme de fonctions sinusoïdales. Historiquement le mathématicien français Josepeh Fourier (1768-1830) a construit ce concept en étudiant la propagation de la chaleur. Commençons par étudier le même problème que lui au travers de son expérience suivante.


\section{L'expérience de Fourier}

	\paragraph{Description et modélisation}
		L'expérience de Fourier sur la propagation de la chaleur dans un matériau est un des exemples historiques d'équation aux dérivées partielles.
		\smallskip

		Un anneau métallique est chauffé à blanc sur une partie, puis il est plongé dans du sable, un matériau isolant. Nous cherchons à étudier ce qu'il se passe entre la temps initial et l'infini où la chaleur devrait s'être diffusée dans tout l'anneau.

		On suppose le rayon de l'anneau suffisamment grand devant la taille de la section afin de n'avoir qu'une propagation longitudinale selon l'anneau. On repère alors la progression de la chaleur selon l'angle $\theta$ avec $\theta_0 = 0$ l'endroit où l'anneau est chauffé initialement.

		On note $u(\theta,t)$ la température de l'anneau à l'angle $\theta\in]-\pi;\pi[$ à l'instant $t\geq0$. On a l'équation aux dérivées partielles suivantes :
		% \begin{equation}
			% \label{eq-6-edpTemp}
			$$
			c \times \rho \times \dfdp{u}{t} - \lambda \dfdpp{u}{\theta}{2} = 0
			$$
		% \end{equation}
		où $c$ est la capacité calorifique, $\rho$ la masse linéique et $\lambda$ la conductivité thermique.

		En posant $d = \frac{\lambda}{c\rho}$ on a :
		\begin{equation}
			\label{eq-6-edpTempRed}
			\dfdp{u}{t} - d \dfdpp{u}{\theta}{2} = 0
		\end{equation}
		\smallskip

		Pour des raisons pratiques, on supppose une symétrie de la chaleur dans l'anneau :
		$$
			\forall\theta\in[-\pi;\pi] : ~ u(\theta,t) = u(-\theta,t)
		$$
		$u$ est alors paire et on travaillera donc sur l'intervalle $I=[0;\pi]$.
		On suppose aussi $u(\theta,t)$ dérivable par rapport à $\theta$.

	\paragraph{Conditions}
		On ajoute les conditions aux limites : $\forall t \geq 0$
		\begin{equation}
			\label{eq-6-condLimite}
			\dfdp{u}{\theta}(0,t) = \dfdp{u}{\theta}(\pi,t) = 0
		\end{equation}
		et la condition initiale :
		\begin{equation}
			\label{eq-6-condIni}
			u(\theta, 0) = f(\theta) \qquad \forall\theta\in I
		\end{equation}

	\paragraph{Construction de \texorpdfstring{$u_n$}{un}}
		On cherche $u$ sous la forme à variables séparées : \quad $u(\theta,t) = g(\theta)h(t)$.\\
		On a :
		\begin{align}
			\notag
			\eqref{eq-6-edpTempRed} \implies& g(\theta)h'(t) - dg''(\theta)h(t) = 0		\\
									\iff& \frac{g''(\theta)}{g(\theta)} = \frac{h'(t)}{d\times h(t)} = C
									\label{eq-6-const}
		\end{align}
		Les deux parts de l'équation \eqref{eq-6-const} dépendent de variables différentes donc elles sont égales à la même constante $C\in\R$.
		On trouve alors $h$ et $g$ :
		\begin{align*}
			\eqref{eq-6-const} 	&\implies h'(t) = Cdh(t)		\\
								&\implies h(t) = \gamma e^{Cdt}	\qquad \gamma \in\R
		\end{align*}
		De plus $d>0$ donc si $C>0$ alors $h(t) \limto{t}{\infty} \infty$. Or cela n'est pas plausible physiquement, ainsi $C\leq0$.

		$$
			\eqref{eq-6-const} \implies g''(\theta) - Cg(\theta) = 0
		$$
		On pose $C=-\omega^2$
		$$
			g''(\theta) + \omega^2 g(\theta) = 0 \implies g(\theta) = \alpha \cos(\omega\theta) + \beta \sin(\omega\theta)
		$$
		Comme $u$ (et donc $g$) est paire par rapport à $\theta$, $\beta=0$. De plus on a :
		\begin{align*}
			\eqref{eq-6-condLimite} &\implies -\alpha\omega \sin(\omega\pi) \times ke^{Cdt} = 0		\\
										&\implies \sin(\omega\pi) = 0	\\
										&\implies \lambda = n \in \N
		\end{align*}
		$$
			g(\theta) = \alpha \cos(n\theta) \qquad \forall n\in\N
		$$

		Ainsi :
		\begin{equation}
			\label{eq-6-un}
			u_n(\theta,t) = \alpha_n \cos(n\theta) e^{-dn^2t}
		\end{equation}
		est solution de \eqref{eq-6-edpTempRed} et des deux conditions limites \eqref{eq-6-condLimite} mais pas encore de la condition initiale \eqref{eq-6-condIni}.

	\paragraph{Construction de la Série}

		Toute combinaison linéaire de $u_n$ est solution.
		On pose alors :
		\begin{equation}
			\label{eq-6-uSerie}
			u(\theta,t) = \sum_{n\geq0} \alpha_n \cos(n\theta) e^{-dn^2t}
		\end{equation}
		vérifiant les conditions \eqref{eq-6-condLimite} et \eqref{eq-6-condIni} par construction à partir \eqref{eq-6-un}.
		$$
			\eqref{eq-6-condIni} \implies u(\theta,0) = \sum_{n\geq0} \alpha_n \cos(n\theta) = f(\theta)	\quad \forall\theta\in[0.\pi]\\
		$$
 
		D'une part, on a $\forall n > 0$ :
		\begin{align*}
			\int_0^\pi \cos(n\theta) \dint{\theta} = 0
				% TODO : expliquer l'implication
				&\implies \int_0^\pi \alpha_n\cos(n\theta) \dint{\theta} = \int_0^\pi f(\theta) \dint{\theta}		\\
				&\iff \pi\alpha_0 = \int_0^\pi f(\theta) \dint{\theta} \\
				&\iff \alpha_0 = \frac{1}{\pi}\int_0^\pi f(\theta) \dint{\theta} 
		\end{align*}
		Ainsi $\alpha_0$ est la chaleur moyenne.

		D'autre part :
		$$
			\int_0^\pi cos(n\theta) cos(k\theta) \dint{\theta} = 0 \quad\text{si } k\neq n
				\quad\text{et}\quad
			\int_0^\pi cos^2(n\theta) \dint{\theta} = \frac{\pi}{2}
		$$
		Donc :
		\begin{align*}
			\int_0^\pi \sum_{n\geq 0} cos(n\theta) cos(k\theta) \dint{\theta} = \int_0^\pi f(\theta) cos(k\theta) \dint{\theta}
				\iff &\alpha_k \int_0^\pi cos^2(k\theta) \dint{\theta} = \int_0^\pi f(\theta) cos(k\theta) \dint{\theta}	\\
				\iff &\alpha_k = \frac{2}{\pi} \int_0^\pi f(\theta) cos(k\theta) \dint{\theta}
		\end{align*}

		On a ainsi entièrement défini la série de Fourier \eqref{eq-6-uSerie}.

		Si on néglige les termes dont $n\geq 2$ (ils sont négligeables car alors $e^{-dn^2t}$ est proche de zéro), $u(\theta,t)$ peut être approché par :
		$$
			u(\theta,t) \simeq \alpha_0 + \alpha_1 \cos\theta e^{-dt}
		$$

\section{Séries de Fourier}
	
	\begin{definition}[Polynôme trigonométrique]
		Toute fonction de la forme
		\begin{equation}
			\label{eq-6-polyTrigo}
			P_n(x) = \frac{a_0}{2} + \sum_{k=1}^n a_k \cos \left( \frac{2k\pi}{T}x \right) + b_k \sin \left( \frac{2k\pi}{T}x \right)
		\end{equation}
		avec $T$ la période et la fréquence $\omega = \frac{2\pi}{T}$.
	\end{definition}		
	\begin{prop}
		Si $\suite{a}$ et $\suite{b}$ sont décroissants et tendent vers $0$ alors $P_n(x)$ converge pour tout $x$.
	\end{prop}

	\begin{definition}[Série trigonométrique]
		Toute fonction de la forme :    
		\begin{equation}
			\label{eq-6-serieTrigo}
			f(x) = \lim_{n\to\infty} P_n(x)
		\end{equation}
		On suppose que la série converge pour tout $x$.
	\end{definition}		

	\medskip

	\begin{definition}[Convergence simple]
		On dit que $P_n$ converge simplement vers $f$ si :
		$$
			\forall\epsilon >0, \forall x \in [0;T], \exists N \in\N \tq \forall n \in \N : n>N \implies \abs{P_n(x) - f(x)} < \epsilon
		$$
	\end{definition}
	\begin{definition}[Convergence uniforme]
		On dit que $P_n$ converge uniformément vers $f$ si :
		$$
			\forall\epsilon >0, \exists N \in\N \tq \forall x \in [0;T], \forall n \in \N : n>N \implies \abs{P_n(x) - f(x)} < \epsilon
		$$
		% TODO th de la convergeance dominée...
	\end{definition}

	Si $P_n$ converge vers $f$ uniformément alors on peut montrer que :
	\begin{align*}
		a_n &= \frac{2}{T} \int_0^T f(x)\cos(n\omega x) \dint{x}		\\
		b_n &= \frac{2}{T} \int_0^T f(x)\sin(n\omega x) \dint{x}
	\end{align*}
	avec la pulsation $\omega = \frac{2\pi}{T}$. L'intégrale peut être centrée sur $[-\frac{T}{2}; \frac{T}{2}]$, l'essentiel étant qu'elle soit de longueur $T$.

	\begin{definitionShort}
		Un point de discontinuité de première espèce a une limite à gauche et à droite.
	\end{definitionShort}
	$\sin(\frac{1}{x})$ possède une discontinuité de seconde espèce en $0$ (pas de limites).

	\begin{theoreme}[Théorème de Dirichlet]
		\label{th-6-dirichlet}
		Soit $f$ une fonction périodique telle que :
		\begin{itemize}
			\item les discontinuités de $f$ dans un période sont en nombre fini et de première espèce
			\item $f$ admet une dérivée à gauche et à droite
		\end{itemize}
		Alors $S(x) = P_n(x)$ converge et on a :
		\begin{equation}
			S(x) = \begin{dcases*}
				\frac{f(x^+) + f(x^-)}{2}	& si $f$ est discontinue en $x$	\\
				f(x)						& sinon
			\end{dcases*}
		\end{equation}
		avec $f(x^+)$ la limite à droite et $f(x^-)$ celle à gauche.
		La convergeance est uniforme sur tout intervalle où $f$ est continue.
	\end{theoreme}

	% TODO Filtre de Karman ???
	% TODO Fonction de carré intégrable
	% TODO : contre exemple : fonction de Weirstrass

	Pour calculer n'importe quel série de Fourier réelle dans Scilab, j'ai créé le code suivant :

	\begin{listing}[H]
		\scicode{\tdF 0.1-Exemple_Calcul_Serie.sce}
		\caption{Exemple du calcul d'une Série de Fourier}
		\label{code-6-calculSerie}
	\end{listing}

	Les différents paramètres sont :
	\begin{itemize}
		\item \code{a0} 
		\item \code{a(x,n,T)} et \code{b(x,n,T)} les fonctions permettant de calculer les coefficients $a_n$ et $b_n$, on peut aussi passer $0$ si le coefficent est nul afin d'éviter des appels inutiles
		\item \code{x} le vecteur des abscisses discrétisé pour lesquelles on calcule la série.
		\item \code{T} la période
		\item \code{NB_ITE}	le nombre d'itérations $n$ pour calculer la série
	\end{itemize}

	Cela permet de calculer n'importe quelle série simplement en renseignant ces quelques paramètres.

	\bigskip

	% \subparagraph{Cas complexe}
		Dans le cas complexe, la série est définie de la façon suivante :
		\begin{equation}
			S(x) = \sum_{n \in \Z} c_n e^{i \times n \omega x}
		\end{equation}
		Avec les coefficients complexes $c_n$ tels que :
		\begin{equation}
			c_n = \frac{1}{T} \int_{-T/2}^{T/2} f(t) e^{-i \times n \omega t} \dint{t}
		\end{equation}

		% TODO : étoffer

		
	% \subsection{Exemples A TRIER}

	% a.
	% 	Soit $f:\R\to\R$ périodique de période $T=2\pi$ vérifiant $f(x)=x ~\forall x\in]-\pi;\pi[$.
	% 	% TODO : Figure
	% 	On a alors :
	% 	\begin{align}
	% 		a_n &= \frac{1}{\pi} \int_{-\pi}^\pi x\cos(n x) \dint{x} = 0 \quad\text{car impaire}		\\
	% 		b_n &= \frac{1}{\pi} \int_{-\pi}^\pi x\sin(n x) \dint{x} = \frac{2}{\pi} \int_0^\pi x\sin(n x) \dint{x}
	% 			= \frac{2}{n\pi} (-1)^{n+1}
	% 	\end{align}
	% 	$$
	% 		\forall x \in]-\pi;\pi[ : \quad f(x) = x = \frac{2}{n\pi} \sum_{n>0} (-1)^{n+1} \frac{\sin(nx)}{n}
	% 	$$


\section{Applications}

	\subsection{Phénomène de Gibbs}
		Prenons la fonction $f$ de période $2\pi$ telle que :
		\begin{equation}
			f(x) = \begin{cases}
				-1		&	\forall x \in [-\pi;0[	\\
				1		&	\forall x \in [0; \pi[
			\end{cases}
		\end{equation}
		On a les termes $a_n$ et $b_n$ :
		\begin{align}
			a_n &= \frac{1}{\pi} \int_{-\pi}^\pi f(x)\cos(n x) \dint{x} = 0 \quad\text{car impaire}		\\
			b_n &= \frac{1}{\pi} \int_{-\pi}^\pi f(x)\sin(n x) \dint{x}
				= \frac{1}{\pi} \int_0^\pi \sin(nx) \dint{x}
				= \frac{2}{n\pi} (1 - \cos(n\pi)) = \frac{2}{n\pi}(1-(-1)^n)
		\end{align}
		Ainsi seules les termes avec $n$ impair ne sont pas nuls, on pose $n=2p+1$ et on obtien la série de Fourier correspondante : 
		\begin{equation}
			S(x) = \frac{4}{\pi} \sum_{p\geq0} \frac{\sin((2p+1)x)}{2p+1}
		\end{equation}
		Comme le prévoit le théorème de Dirichlet \eqref{th-6-dirichlet}, on n'a pas égalité entre $f(x)$ et $S(x)$ aux points de discontinuité (ici de la forme $x=k\pi$ avec $k$ impair) mais bien une moyenne des deux limites, ici $\frac{-\pi + \pi}{2} = 0$.
		Que se passe-t-il donc à ces points de discontinuité ?

		Découvrons le avec le code suivant :
		\begin{listing}[H]
			\scicode{\tdF 1-Phenomene_Gibbs.sce}
			\caption{Phénomène de Gibbs}
			\label{code-6-phenomeneGibbs}
		\end{listing}

		Ici nous allons calculer la série de Fourier d'un signal carré de période $T = 2\pi$ défini par :
		\begin{align*}
			\forall x \in [-\pi;0[ : &	f(x) = -1		\\
			\forall x \in [0;\pi] : &	f(x) = 1		\\
		\end{align*}
		% TODO Finir

		Nous obtenons le rendu suivant :

		\begin{figure}[H]
			\centering
			\includegraphics[width=.6\linewidth, trim=2cm 2cm 2cm 2cm, clip]{\tdF\img 1-Phenomene_Gibbs.eps}
			\caption{Phénomène de Gibbs}
			\label{img-6-phenomeneGibbs}
		\end{figure}

		On observe des oscillations aux voisinage des points de discontinuité. Ces erreurs d'approximation sont dues au fait que $S$ ne converge pas uniformément vers $f$ aux points de discontinuité.
		Ceci est appelé le phénomène de Gibbs.
		% Elles sont de l'ordre de 9\% de dépassement de la valeur limite de chaque coté.
		% TODO : Vérif ?

	\subsection{Régularité et décroissance des coefficients}

		Nous allons ici comparer trois séries de Fourier approchant la fonction $f$ suivante sur $[0; \frac{1}{2}]$ :
		\begin{equation}
			\label{eq-6-fComp}
			f(x) = x(x-1)
		\end{equation}

		Nous avons les trois fonctions suivantes qui vont nous servir pour les séries de Fourier :
		\begin{itemize}
			\item $f_1$ de période $T_1 = \frac{1}{2}$ : \qquad\qquad\quad~~ $\forall x\in[0;\frac{1}{2}] : \quad f_1(x) = f(x) $	
			\item $f_1$ paire de période $T_2 = 1$ : \qquad\quad\, $\forall x\in[0;\frac{1}{2}] : \quad f_2(x) = f(x) $		
			\item $f_1$ impaire de période $T_3 = 2$ : \qquad $\forall x\in[0;1] : \quad f_3(x) = f_2(x) $	
		\end{itemize}

		\bigskip
		On calcule les différents coefficents $a_n$ et $b_n ~ \forall n>0$ :
		\medskip
		\begin{itemize}
			\item Pour $f_1$ : $\omega = 4\pi$
				\\ \smallskip\qquad
				$\left\lvert \quad
				\begin{aligned}
					a_n &= 4 \int_0^\frac{1}{2} x(x-1) \cos(4\pi n x) \dint{x} = ... = \frac{1}{4\pi^2n^2}		\\
					b_n &= 4 \int_0^\frac{1}{2} x(x-1) \sin(4\pi n x) \dint{x} = ... = \frac{1}{4\pi n}			\\
					a_0 &= \frac{1}{3}
				\end{aligned}
				\right.$\smallskip	\\
				La vitesse de décroissance des coefficients de $f_1$ est $1$ (on prend le minimum des deux coefficients). 
				Cela signifie que la fonction n'est pas régulière en tous points, en effet au point de discontinuités on a $S\neq f_1$ d'après le théorème \ref{th-6-dirichlet}. Cela est dû au fait que la fonction $f_1$ est discontinue.
				\bigskip

			\item Pour $f_2$ : $\omega = 2\pi$
				\\ \smallskip\qquad
				$\left\lvert \quad
				\begin{aligned}
					a_n &= 2 \int_{-\frac{1}{2}}^\frac{1}{2} f_2(x) \cos(2\pi nx) \dint{x}
						= 4 \int_0^\frac{1}{2} x(x-1) \cos(2\pi nx) \dint{x}
						= ... = \frac{1}{\pi^2n^2}	\\
					b_n &= 0 \qquad \text{car $f_2$ est paire ??}		\\
					a_0 &= \frac{1}{3}
				\end{aligned}
				\right.$\smallskip	\\
				La vitesse de décroissance des coefficients de $f_2$ est $2$. 
				On a donc $S = f_2$ car la fonction est continue.
				\bigskip

			\item Pour $f_2$ : $\omega = \pi$
				\\ \smallskip\qquad
				$\left\lvert \quad
				\begin{aligned}
					a_n &= 0 \qquad \text{car $f_2$ est impaire ??}		\\
					b_n &= %2 \int_{-\frac{1}{2}}^\frac{1}{2} f_2(x) \cos(2\pi nx) \dint{x}
						%= 4 \int_0^\frac{1}{2} x(x-1) \cos(2\pi nx) \dint{x}
						 ... = 4 \frac{(-1)^n - 1}{\pi^3 n^3}	\\
					a_0 &= 0
				\end{aligned}
				\right.$\smallskip	\\
				La vitesse de décroissance des coefficients de $f_3$ est $3$. 
				Ici, en plus d'être continue et $S = f_3$, la fonction est également continûment dérivable.
				\bigskip
		\end{itemize}

		Ainsi le choix de la fonction périodique $f_i$ pour la série impacte la convergence de celle-ci vers $f$. Plus les coefficients décroissent vite, plus la convergence est élevée.
		
		Vérfions cela expérimentalement avec Scilab :

		\begin{listing}[H]
			\scicode{\tdF 2-Comparaison_Continuite.sce}
			\caption{Comparaison de séries}
			\label{code-6-compContinuite}
		\end{listing}

		On calcule d'abord les trois séries avec la fonction \ref{code-6-calculSerie} pour $n=400$ points, puis on affiche les séries et la fonction d'origine $f$ sur le même graphe. Ensuite on calcule l'écart entre la fonction $f$ et les différentes séries, on l'affiche.

		\begin{figure}[H]
			\centering
			\includegraphics[width=.8\linewidth, trim=2cm 0cm 2cm 0cm, clip]{\tdF\img 2-Comparaison_Continuite.eps}
			\caption{Différences entre les séries}
			\label{img-5-compContSeries}
		\end{figure}

		On remarque sur les deux graphiques que $S_3$ est bien plus proche de $f$ que $S_2$ et encore plus $S_1$. Cela confirme donc bien que plus les coefficients décroissent, plus la série converge et est proche de $f$.

	\subsection{Propagation de la chaleur}
		
		Nous allons simuler ici l'expérience de Fourier.
		Nous prenons $d=1$, la fonction de départ $f(\theta)$ est définie par $\lambda\in]0;\pi[$ :
		$$
			f(\theta) = \begin{cases}
				1	&	\text{si } x\in[\pi-\lambda; \pi+\lambda]		\\
				0	&	\text{sinon}
			\end{cases}
		$$
		On prend ici $\lambda = \frac{\pi}{2}$.

		Nous calculons les coefficients de la série de Fourier correspondante :
		\begin{align*}
			T = 2\pi \implies \omega &= \frac{2\pi}{T} = 1 \\
			f \text{ est paire donc } b_n &= 0 \\
			a_n &= \frac{1}{\pi} \int_{-\pi}^{\pi} f(x) \cos(n\omega x) \dint{x}
				= \frac{2}{\pi} \int_0^\pi f(x) \cos(n x) \dint{x}	\\
				&= \frac{2}{\pi} \int_{\pi-\lambda}^\pi \cos(n x) \dint{x}
				= \frac{2}{n\pi} \left( \sin(n\pi)-\sin(n(\pi - \lambda)) \right) \\
				&= \frac{-2}{n\pi} \sin(n(\pi-\lambda)) \\
			a_0 &= \frac{1}{2\pi} \int_0^{2\pi} f(x) \dint{x} = 2 \frac{\lambda}{\pi}
		\end{align*}
		
		Il suffit ensuite de calculer la série correspondante avec le programme suivant :

		\begin{listing}[H]
			\scicode{\tdF 3-Chaleur.sce}
			\caption{Chaleur}
			\label{code-6-chaleur}
		\end{listing}

		La série n'est alors pas définie pour $t=0$ car nous avons $u(\theta, 0) = f(\theta)$. Nous choisissons $t_0$ légèrement supérieur à 0 pour éviter le phénomène de Gibbs.


		J'ai fait diverses visualisations animées mais sur la suivante nous permet de voir l'évolution de la chaleur en fonction du temps et de $\theta$.

		\begin{figure}[H]
			\centering
			\includegraphics[width=.6\linewidth, trim=1cm 1cm 1cm 1cm, clip]{\tdF\img 3a-Chaleur.png}
			\caption{Évolution de la chaleur}
			\label{img-6-Chaleur}
		\end{figure}

		Les points blancs correspondent aux points les plus chauds et les noirs aux plus froids.

		Nous pouvons aussi visualiser la propagation de la chaleur dans un anneau avec l'animation. Prenons deux temps on observons :
		\begin{figure}[H]
			\centering
			\begin{subfigure}{.45\linewidth}
				\centering
				\includegraphics[width=\linewidth, trim=1cm 1cm 1cm 1cm, clip]{\tdF\img 3b-Chaleur_t08.png}
				\caption{Chaleur pour $t=0.5$}
			\end{subfigure}
			\begin{subfigure}{.45\linewidth}
				\centering
				\includegraphics[width=\linewidth, trim=1cm 1cm 1cm 1cm, clip]{\tdF\img 3b-Chaleur_fin.png}
				\caption{Chaleur pour $t=6$}
			\end{subfigure}
			\caption{Propagation de la chaleur dans l'anneau}
			\label{img-6-anneauChaleur}
		\end{figure}






% TODO : somme to integrale Riemann

% Théorème : 
% Si an et bn tendent vers 0 alors la série converge vers une fonction
% Si an et bn simplement bornés alors converge vers une distribution
% 
% Td Chaleur
% (an cos + bn sin)*e^azdazd tend très vite vers 0 et donc si on remplace an et bn par random on obtient quand meme une fonction assez régulière

	

	% == ANNEXE SCILAB == %
%	\part{Scilab}

%	\part{Rappels de Mathématiques}

	% == TABLES == %	
	\listoflistings
%	\listofalgorithms		% Mettre les algo avec les codes sources ?
	\listoffigures


\end{document}


% ============================================================
%			Idea
% ============================================================
% Faire algos avec complexité, comparer avec les outils Scilab, commenter
% Mettre des points à la fin des phrases
% IDEA : Part : technique de factorisation du code, outils utiles, bonnes pratiques, problèmes rencontrés